\chapter{README to LKMM's documentation directory}

\begin{Note}
  Pretty-printed \path{tools/memory-model/Documentation/README}.
\end{Note}

It has been said that successful communication requires first identifying
what your audience knows and then building a bridge from their current
knowledge to what they need to know.
Unfortunately, the expected Linux-kernel memory model (LKMM) audience
might be anywhere from noviceto expert both in kernel hacking and in
understanding LKMM.

This document therefore points out a number of places to start reading,
depending on what you know and what you would like to learn.
Please note that the documents later in this list assume that the reader
understands the material provided by documents earlier in this list.

If LKMM-specific terms lost you, \path{glossary.txt} might help you.

\begin{itemize}
  \item You are new to Linux-kernel concurrency:
	\path{simple.txt}

  \item You have some background in Linux-kernel concurrency, and would
	like an overview of the types of low-level concurrency primitives
	that the Linux kernel provides:
	\path{ordering.txt}

	Here, ``low level'' means atomic operations to single variables.

  \item You are familiar with the Linux-kernel concurrency primitives
	that you need, and just want to get started with LKMM litmus
	tests:
	\path{litmus-tests.txt}

  \item You would like to access lock-protected shared variables without
	having their corresponding locks held:
	\path{locking.txt}

  \item You are familiar with Linux-kernel concurrency, and would
	like a detailed intuitive understanding of LKMM, including
	situations involving more than two threads:
	\path{recipes.txt}

  \item You would like a detailed understanding of what your compiler can
	and cannot do to control dependencies:
	\path{control-dependencies.txt}

  \item You would like to mark concurrent normal accesses to shared
	variables so that intentional ``racy'' accesses can be properly
	documented, especially when you are responding to complaints
	from KCSAN\@:
	\path{access-marking.txt}

  \item You are familiar with Linux-kernel concurrency and the use of
	LKMM, and would like a quick reference:
	\path{cheatsheet.txt}

  \item You are familiar with Linux-kernel concurrency and the use
	of LKMM, and would like to learn about LKMM's requirements,
	rationale, and implementation:
	\path{explanation.txt} and \path{herd-representation.txt}

  \item You are interested in the publications related to LKMM, including
	hardware manuals, academic literature, standards-committee
	working papers, and LWN articles:
	\path{references.txt}
\end{itemize}


\section{Description of files}

\begin{description}[style=nextline]
  \item[\path{README}]
	This file.

  \item[\path{access-marking.txt}]
	Guidelines for marking intentionally concurrent accesses to
	shared memory.

  \item[\path{cheatsheet.txt}]
	Quick-reference guide to the Linux-kernel memory model.

  \item[\path{control-dependencies.txt}]
	Guide to preventing compiler optimizations from destroying
	your control dependencies.

  \item[\path{explanation.txt}]
	Detailed description of the memory model.

  \item[\path{glossary.txt}]
	Brief definitions of LKMM-related terms.

  \item[\path{herd-representation.txt}]
	The (abstract) representation of the Linux-kernel concurrency
	primitives in terms of events.

  \item[\path{litmus-tests.txt}]
	The format, features, capabilities, and limitations of the litmus
	tests that LKMM can evaluate.

  \item[\path{locking.txt}]
	Rules for accessing lock-protected shared variables outside of
	their corresponding critical sections.

  \item[\path{ordering.txt}]
	Overview of the Linux kernel's low-level memory-ordering
	primitives by category.

  \item[\path{recipes.txt}]
	Common memory-ordering patterns.

  \item[\path{references.txt}]
	Background information.

  \item[\path{simple.txt}]
	Starting point for someone new to Linux-kernel concurrency.
	And also a reminder of the simpler approaches to concurrency!
\end{description}
