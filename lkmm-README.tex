\chapter{README to Linux kernel memory consistency model (LKMM)}

Pretty-printing \path{tools/memory-model/README}.

\section{Introduction}

This directory contains the memory consistency model (memory model, for
short) of the Linux kernel, written in the \co{cat} language and executable
by the externally provided \co{herd7} simulator, which exhaustively explores
the state space of small litmus tests.

In addition, the \co{klitmus7} tool (also externally provided) may be used
to convert a litmus test to a Linux kernel module, which in turn allows
that litmus test to be exercised within the Linux kernel.


\section{Requirements}

Version 7.52 or higher of the \co{herd7} and \co{klitmus7} tools must be
downloaded separately:

\begin{itemize}
  \item \url{https://github.com/herd/herdtools7}
\end{itemize}

See \path{herdtools7/INSTALL.md} for installation instructions.

Note that although these tools usually provide backwards compatibility,
this is not absolutely guaranteed.

For example, a future version of \co{herd7} might not work with the model
in this release.
A compatible model will likely be made available in a later release of
Linux kernel.

If you absolutely need to run the model in this particular release,
please try using the exact version called out above.

\co{klitmus7} is independent of the model provided here.  It has its own
dependency on a target kernel release where converted code is built
and executed.  Any change in kernel APIs essential to \co{klitmus7} will
necessitate an upgrade of \co{klitmus7}.

If you find any compatibility issues in \co{klitmus7}, please inform the
memory model maintainers.

\subsection{\tco{klitmus7} Compatibility Table}

\begin{VerbatimU}
	============  ==========
	target Linux  herdtools7
	------------  ----------
	     -- 4.14  7.48 --
	4.15 -- 4.19  7.49 --
	4.20 -- 5.5   7.54 --
	5.6  -- 5.16  7.56 --
	5.17 --       7.56.1 --
	============  ==========
\end{VerbatimU}

\section{Basic usage: \tco{herd7}}

The memory model is used, in conjunction with \co{herd7}, to exhaustively
explore the state space of small litmus tests.
Documentation describing the format, features, capabilities and
limitations of these litmus tests is available in
\path{tools/memory-model/Documentation/litmus-tests.txt}.

Example litmus tests may be found in the Linux-kernel source tree:

\begin{itemize}
  \item \path{tools/memory-model/litmus-tests/}
  \item \path{Documentation/litmus-tests/}
\end{itemize}

Several thousand more example litmus tests are available here:

\begin{itemize}
  \item \url{https://github.com/paulmckrcu/litmus}
  \item \url{https://git.kernel.org/pub/scm/linux/kernel/git/paulmck/perfbook.git/tree/CodeSamples/formal/herd}
  \item \url{https://git.kernel.org/pub/scm/linux/kernel/git/paulmck/perfbook.git/tree/CodeSamples/formal/litmus}
\end{itemize}

Documentation describing litmus tests and now to use them may be found
here:

\begin{itemize}
  \item \path{tools/memory-model/Documentation/litmus-tests.txt}
\end{itemize}

The remainder of this section uses the \path{SB+fencembonceonces.litmus}
test located in the \path{tools/memory-model} directory.

To run \path{SB+fencembonceonces.litmus} against the memory model:

\begin{VerbatimU}
  $ cd $LINUX_SOURCE_TREE/tools/memory-model
  $ herd7 -conf linux-kernel.cfg litmus-tests/SB+fencembonceonces.litmus
\end{VerbatimU}

Here is the corresponding output:

\begin{VerbatimU}
  Test SB+fencembonceonces Allowed
  States 3
  0:r0=0; 1:r0=1;
  0:r0=1; 1:r0=0;
  0:r0=1; 1:r0=1;
  No
  Witnesses
  Positive: 0 Negative: 3
  Condition exists (0:r0=0 /\ 1:r0=0)
  Observation SB+fencembonceonces Never 0 3
  Time SB+fencembonceonces 0.01
  Hash=d66d99523e2cac6b06e66f4c995ebb48
\end{VerbatimU}

The \qco{Positive: 0 Negative: 3} and the \qco{Never 0 3} each indicate that
this litmus test's \qco{exists} clause can not be satisfied.

See \qco{herd7 -help} or \path{herdtools7/doc/} for more information on
running the tool itself, but please be aware that this documentation is
intended for people who work on the memory model itself, that is, people
making changes to the \path{tools/memory-model/linux-kernel.*} files.
It is not intended for people focusing on writing, understanding, and
running LKMM litmus tests.

\section{Basic usage: \tco{klitmus7}}

The \qco{klitmus7} tool converts a litmus test into a Linux kernel module,
which may then be loaded and run.

For example, to run \path{SB+fencembonceonces.litmus} against hardware:

\begin{VerbatimU}
  $ mkdir mymodules
  $ klitmus7 -o mymodules litmus-tests/SB+fencembonceonces.litmus
  $ cd mymodules ; make
  $ sudo sh run.sh
\end{VerbatimU}

The corresponding output includes:

\begin{VerbatimU}
  Test SB+fencembonceonces Allowed
  Histogram (3 states)
  644580  :>0:r0=1; 1:r0=0;
  644328  :>0:r0=0; 1:r0=1;
  711092  :>0:r0=1; 1:r0=1;
  No
  Witnesses
  Positive: 0, Negative: 2000000
  Condition exists (0:r0=0 /\ 1:r0=0) is NOT validated
  Hash=d66d99523e2cac6b06e66f4c995ebb48
  Observation SB+fencembonceonces Never 0 2000000
  Time SB+fencembonceonces 0.16
\end{VerbatimU}

The \qco{Positive: 0 Negative: 2000000} and the \qco{Never 0 2000000} indicate
that during two million trials, the state specified in this litmus
test's \qco{exists} clause was not reached.

And, as with \qco{herd7}, please see \qco{klitmus7 -help} or
\path{herdtools7/doc/} for more information.
And again, please be aware that this documentation is intended for people
who work on the memory model itself, that is, people making changes to
the \path{tools/memory-model/linux-kernel.*} files.
It is not intended for people focusing on writing, understanding, and
running LKMM litmus tests.

\section{Description of files}

\begin{description}[style=nextline]
\item[\path{Documentation/README}]
	Guide to the other documents in the \path{Documentation/} directory.

\item[\path{linux-kernel.bell}]
	Categorizes the relevant instructions, including memory
	references, memory barriers, atomic read-modify-write operations,
	lock acquisition/release, and RCU operations.

	More formally, this file (1)~lists the subtypes of the various
	event types used by the memory model and (2)~performs RCU
	read-side critical section nesting analysis.

\item[\path{linux-kernel.cat}]
	Specifies what reorderings are forbidden by memory references,
	memory barriers, atomic read-modify-write operations, and RCU\@.

	More formally, this file specifies what executions are forbidden
	by the memory model.
	Allowed executions are those which satisfy the model's ``coherence'',
	``atomic'', ``happens-before'',	``propagation'', and ``rcu'' axioms,
	which are defined in the file.

\item[\path{linux-kernel.cfg}]
	Convenience file that gathers the common-case \co{herd7} command-line
	arguments.

\item[\path{linux-kernel.def}]
	Maps from C-like syntax to \co{herd7}'s internal litmus-test
	instruction-set architecture.

\item[\path{litmus-tests}]
	Directory containing a few representative litmus tests, which
	are listed in \path{litmus-tests/README}.
        A great deal more litmus tests are available at
	\url{https://github.com/paulmckrcu/litmus}.

	By ``representative'', it means the one in the litmus-tests
	directory is:

	\begin{enumerate}
	  \item	simple, the number of threads should be relatively
		small and each thread function should be relatively
		simple.
	  \item orthogonal, there should be no two litmus tests
		describing the same aspect of the memory model.
	  \item textbook, developers can easily copy-paste-modify
		the litmus tests to use the patterns on their own
		code.
	\end{enumerate}

\item[\path{lock.cat}]
	Provides a front-end analysis of lock acquisition and release,
	for example, associating a lock acquisition with the preceding
	and following releases and checking for self-deadlock.

	More formally, this file defines a performance-enhanced scheme
	for generation of the possible reads-from and coherence order
	relations on the locking primitives.

\item[\path{README}]
	This file.

\item[\path{scripts}]
	Various scripts, see \path{scripts/README}.
\end{description}
