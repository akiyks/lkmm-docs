% .. SPDX-License-Identifier: GPL-2.0

\section{Using RCU \texttt{hlist\_nulls} to protect list and objects}
\label{sec:rcu:Using RCU hlist_nulls to protect list and objects}

This section describes how to use \co{hlist_nulls} to
protect read-mostly linked lists and
objects using \co{SLAB_TYPESAFE_BY_RCU} allocations.

Please read the basics in \path{listRCU.rst}.

\subsection{Using `\texttt{nulls}'}
\label{sec:rcu:Using nulls}

Using special makers (called \qco{nulls}) is a convenient way
to solve following problem.

Without \qco{nulls}, a typical RCU linked list managing objects which are
allocated with \co{SLAB_TYPESAFE_BY_RCU} \co{kmem_cache} can use the following
algorithms.
Following examples assume \qco{obj} is a pointer to such
objects, which is having below type.

\begin{VerbatimU}
	struct object {
		struct hlist_node obj_node;
		atomic_t refcnt;
		unsigned int key;
	};
\end{VerbatimU}

\subsubsection{Lookup algorithm}

\begin{VerbatimU}
	begin:
	rcu_read_lock();
	obj = lockless_lookup(key);
	if (obj) {
		if (!try_get_ref(obj)) { // might fail for free objects
			rcu_read_unlock();
			goto begin;
		}
		/*
		 * Because a writer could delete object, and a writer could
		 * reuse these object before the RCU grace period, we
		 * must check key after getting the reference on object
		 */
		if (obj->key != key) { // not the object we expected
			put_ref(obj);
			rcu_read_unlock();
			goto begin;
		}
	}
	rcu_read_unlock();
\end{VerbatimU}

Beware that \co{lockless_lookup(key)} cannot use traditional \co{hlist_for_each_entry_rcu()}
but a version with an additional memory barrier (\co{smp_rmb()}):

\begin{VerbatimU}
	lockless_lookup(key)
	{
		struct hlist_node *node, *next;
		for (pos = rcu_dereference((head)->first);
		     pos && ({ next = pos->next; smp_rmb(); prefetch(next); 1; }) &&
		     ({ obj = hlist_entry(pos, typeof(*obj), obj_node); 1; });
		     pos = rcu_dereference(next))
			if (obj->key == key)
				return obj;
		return NULL;
	}
\end{VerbatimU}

And note the traditional \co{hlist_for_each_entry_rcu()} misses this \co{smp_rmb()}:

\begin{VerbatimU}
	struct hlist_node *node;
	for (pos = rcu_dereference((head)->first);
	     pos && ({ prefetch(pos->next); 1; }) &&
	     ({ obj = hlist_entry(pos, typeof(*obj), obj_node); 1; });
	     pos = rcu_dereference(pos->next))
		if (obj->key == key)
			return obj;
	return NULL;
\end{VerbatimU}

Quoting Corey Minyard:

\begin{quote}
  ``If the object is moved from one list to another list in-between the
  time the hash is calculated and the next field is accessed, and the
  object has moved to the end of a new list, the traversal will not
  complete properly on the list it should have, since the object will
  be on the end of the new list and there's not a way to tell it's on a
  new list and restart the list traversal.
  I think that this can be
  solved by pre-fetching the \qco{next} field (with proper barriers) before
  checking the key.''
\end{quote}

\subsubsection{Insertion algorithm}

We need to make sure a reader cannot read the new \qco{obj->obj_node.next} value
and previous value of \qco{obj->key}.
Otherwise, an item could be deleted
from a chain, and inserted into another chain.
If new chain was empty
before the move, \qco{next} pointer is \co{NULL}, and lockless reader can not
detect the fact that it missed following items in original chain.

\begin{VerbatimU}
	/*
	 * Please note that new inserts are done at the head of list,
	 * not in the middle or end.
	 */
	obj = kmem_cache_alloc(...);
	lock_chain(); // typically a spin_lock()
	obj->key = key;
	atomic_set_release(&obj->refcnt, 1); // key before refcnt
	hlist_add_head_rcu(&obj->obj_node, list);
	unlock_chain(); // typically a spin_unlock()
\end{VerbatimU}


\subsubsection{Removal algorithm}

Nothing special here, we can use a standard RCU hlist deletion.
But thanks to \co{SLAB_TYPESAFE_BY_RCU}, beware a deleted object can be reused
very very fast (before the end of RCU grace period):

\begin{VerbatimU}
	if (put_last_reference_on(obj) {
		lock_chain(); // typically a spin_lock()
		hlist_del_init_rcu(&obj->obj_node);
		unlock_chain(); // typically a spin_unlock()
		kmem_cache_free(cachep, obj);
	}
\end{VerbatimU}


\subsection{Avoiding extra \texttt{smp\_rmb()}}

With \co{hlist_nulls} we can avoid extra \co{smp_rmb()} in \co{lockless_lookup()}.

For example, if we choose to store the slot number as the \qco{nulls}
end-of-list marker for each slot of the hash table, we can detect
a race (some writer did a delete and/or a move of an object
to another chain) checking the final \qco{nulls} value if
the lookup met the end of chain.
If final \qco{nulls} value
is not the slot number, then we must restart the lookup at
the beginning.
If the object was moved to the same chain,
then the reader doesn't care:
It might occasionally
scan the list again without harm.

Note that using \co{hlist_nulls} means the type of \qco{obj_node} field of
\qco{struct object} becomes \qco{struct hlist_nulls_node}.


\subsubsection{lookup algorithm}

\begin{VerbatimU}
	head = &table[slot];
	begin:
	rcu_read_lock();
	hlist_nulls_for_each_entry_rcu(obj, node, head, obj_node) {
		if (obj->key == key) {
			if (!try_get_ref(obj)) { // might fail for free objects
				rcu_read_unlock();
				goto begin;
			}
			if (obj->key != key) { // not the object we expected
				put_ref(obj);
				rcu_read_unlock();
				goto begin;
			}
			goto out;
		}
	}

	// If the nulls value we got at the end of this lookup is
	// not the expected one, we must restart lookup.
	// We probably met an item that was moved to another chain.
	if (get_nulls_value(node) != slot) {
		put_ref(obj);
		rcu_read_unlock();
		goto begin;
	}
	obj = NULL;

	out:
	rcu_read_unlock();
\end{VerbatimU}

\subsubsection{Insert algorithm}

Same to the above one, but uses \co{hlist_nulls_add_head_rcu()} instead of
\co{hlist_add_head_rcu()}.

\begin{VerbatimU}
	/*
	 * Please note that new inserts are done at the head of list,
	 * not in the middle or end.
	 */
	obj = kmem_cache_alloc(cachep);
	lock_chain(); // typically a spin_lock()
	obj->key = key;
	atomic_set_release(&obj->refcnt, 1); // key before refcnt
	/*
	 * insert obj in RCU way (readers might be traversing chain)
	 */
	hlist_nulls_add_head_rcu(&obj->obj_node, list);
	unlock_chain(); // typically a spin_unlock()
\end{VerbatimU}
