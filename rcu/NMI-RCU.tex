% .. _NMI_rcu_doc:

\section{Using RCU to Protect Dynamic NMI Handlers}


Although RCU is usually used to protect read-mostly data structures,
it is possible to use RCU to provide dynamic non-maskable interrupt
handlers, as well as dynamic irq handlers.
This document describes
how to do this, drawing loosely from Zwane Mwaikambo's NMI-timer
work in an old version of \path{arch/x86/kernel/traps.c}.

The relevant pieces of code are listed below, each followed by a
brief explanation:

\begin{VerbatimU}
	static int dummy_nmi_callback(struct pt_regs *regs, int cpu)
	{
		return 0;
	}
\end{VerbatimU}

The \co{dummy_nmi_callback()} function is a ``dummy'' NMI handler that does
nothing, but returns zero, thus saying that it did nothing, allowing
the NMI handler to take the default machine-specific action:

\begin{VerbatimU}
	static nmi_callback_t nmi_callback = dummy_nmi_callback;
\end{VerbatimU}

This \co{nmi_callback} variable is a global function pointer to the current
NMI handler:

\begin{VerbatimU}
	void do_nmi(struct pt_regs * regs, long error_code)
	{
		int cpu;

		nmi_enter();

		cpu = smp_processor_id();
		++nmi_count(cpu);

		if (!rcu_dereference_sched(nmi_callback)(regs, cpu))
			default_do_nmi(regs);

		nmi_exit();
	}
\end{VerbatimU}

The \co{do_nmi()} function processes each NMI\@.
It first disables preemption
in the same way that a hardware irq would, then increments the per-CPU
count of NMIs.
It then invokes the NMI handler stored in the \co{nmi_callback}
function pointer.
If this handler returns zero, \co{do_nmi()} invokes the
\co{default_do_nmi()} function to handle a machine-specific NMI\@.
Finally,
preemption is restored.

In theory, \co{rcu_dereference_sched()} is not needed, since this code runs
only on i386, which in theory does not need \co{rcu_dereference_sched()}
anyway.
However, in practice it is a good documentation aid, particularly
for anyone attempting to do something similar on Alpha or on systems
with aggressive optimizing compilers.

\QuickQuiz{
	Why might the \co{rcu_dereference_sched()} be necessary on Alpha,
	given that the code referenced by the pointer is read-only?
}\QuickQuizAnswer{
	The caller to \co{set_nmi_callback()} might well have
	initialized some data that is to be used by the new NMI
	handler.
	In this case, the \co{rcu_dereference_sched()} would
	be needed, because otherwise a CPU that received an NMI
	just after the new handler was set might see the pointer
	to the new NMI handler, but the old pre-initialized
	version of the handler's data.

	This same sad story can happen on other CPUs when using
	a compiler with aggressive pointer-value speculation
	optimizations.
	(But please don't!)

	More important, the \co{rcu_dereference_sched()} makes it
	clear to someone reading the code that the pointer is
	being protected by RCU-sched.
}\QuickQuizEnd

Back to the discussion of NMI and RCU\@:

\begin{VerbatimU}
	void set_nmi_callback(nmi_callback_t callback)
	{
		rcu_assign_pointer(nmi_callback, callback);
	}
\end{VerbatimU}

The \co{set_nmi_callback()} function registers an NMI handler.
Note that any
data that is to be used by the callback must be initialized up \emph{before}
the call to \co{set_nmi_callback()}.
On architectures that do not order
writes, the \co{rcu_assign_pointer()} ensures that the NMI handler sees the
initialized values:

\begin{VerbatimU}
	void unset_nmi_callback(void)
	{
		rcu_assign_pointer(nmi_callback, dummy_nmi_callback);
	}
\end{VerbatimU}

This function unregisters an NMI handler, restoring the original
\co{dummy_nmi_handler()}.
However, there may well be an NMI handler
currently executing on some other CPU\@.
We therefore cannot free
up any data structures used by the old NMI handler until execution
of it completes on all other CPUs.

One way to accomplish this is via \co{synchronize_rcu()}, perhaps as
follows:

\begin{VerbatimU}
	unset_nmi_callback();
	synchronize_rcu();
	kfree(my_nmi_data);
\end{VerbatimU}

This works because (as of v4.20) \co{synchronize_rcu()} blocks until all
CPUs complete any preemption-disabled segments of code that they were
executing.
Since NMI handlers disable preemption, \co{synchronize_rcu()} is guaranteed
not to return until all ongoing NMI handlers exit.
It is therefore safe
to free up the handler's data as soon as \co{synchronize_rcu()} returns.

\begin{Note}
  Important note{:} for this to work, the architecture in question must
  invoke \co{nmi_enter()} and \co{nmi_exit()} on NMI entry and exit,
  respectively.
\end{Note}
