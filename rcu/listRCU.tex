% .. _list_rcu_doc:

\section{Using RCU to Protect Read-Mostly Linked Lists}
\label{sec:rcu:Using RCU to Protect Read-Mostly Linked Lists}

One of the most common uses of RCU is protecting read-mostly linked lists
(\co{struct list_head} in \path{list.h}).
One big advantage of this approach is
that all of the required memory ordering is provided by the list macros.
This document describes several list-based RCU use cases.

When iterating a list while holding the \co{rcu_read_lock()}, writers may
modify the list.
The reader is guaranteed to see all of the elements
which were added to the list before they acquired the \co{rcu_read_lock()}
and are still on the list when they drop the \co{rcu_read_unlock()}.
Elements which are added to, or removed from the list may or may not
be seen.
If the writer calls \co{list_replace_rcu()}, the reader may see
either the old element or the new element; they will not see both,
nor will they see neither.


\subsection{Example 1: Read-mostly list: Deferred Destruction}

A widely used usecase for RCU lists in the kernel is lockless iteration over
all processes in the system.
\co{task_struct::tasks} represents the list node that
links all the processes.
The list can be traversed in parallel to any list
additions or removals.

The traversal of the list is done using \co{for_each_process()} which is defined
by the 2 macros:

\begin{VerbatimU}
	#define next_task(p) \
	        list_entry_rcu((p)->tasks.next, struct task_struct, tasks)

	#define for_each_process(p) \
	        for (p = &init_task ; (p = next_task(p)) != &init_task ; )
\end{VerbatimU}

The code traversing the list of all processes typically looks like:

\begin{VerbatimU}
	rcu_read_lock();
	for_each_process(p) {
		/* Do something with p */
	}
	rcu_read_unlock();
\end{VerbatimU}

The simplified and heavily inlined code for removing a process from a
task list is:

\begin{VerbatimU}
	void release_task(struct task_struct *p)
	{
		write_lock(&tasklist_lock);
		list_del_rcu(&p->tasks);
		write_unlock(&tasklist_lock);
		call_rcu(&p->rcu, delayed_put_task_struct);
	}
\end{VerbatimU}

When a process exits, \co{release_task()} calls \co{list_del_rcu(&p->tasks)}
via \co{__exit_signal()} and \co{__unhash_process()} under \co{tasklist_lock}
writer lock protection.
The \co{list_del_rcu()} invocation removes
the task from the list of all tasks.
The \co{tasklist_lock}
prevents concurrent list additions/removals from corrupting the
list.
Readers using \co{for_each_process()} are not protected with the
\co{tasklist_lock}.
To prevent readers from noticing changes in the list
pointers, the \co{task_struct} object is freed only after one or more
grace periods elapse, with the help of \co{call_rcu()}, which is invoked via
\co{put_task_struct_rcu_user()}.
This deferring of destruction ensures that
any readers traversing the list will see valid \co{p->tasks.next} pointers
and deletion/freeing can happen in parallel with traversal of the list.
This pattern is also called an \emph{existence lock}, since RCU refrains
from invoking the \co{delayed_put_task_struct()} callback function until
all existing readers finish, which guarantees that the \co{task_struct}
object in question will remain in existence until after the completion
of all RCU readers that might possibly have a reference to that object.


\subsection{Example 2: Read-Side Action Taken Outside of Lock: No In-Place Updates}

Some reader-writer locking use cases compute a value while holding
the read-side lock, but continue to use that value after that lock is
released.
These use cases are often good candidates for conversion
to RCU\@.
One prominent example involves network packet routing.
Because the packet-routing data tracks the state of equipment outside
of the computer, it will at times contain stale data.
Therefore, once
the route has been computed, there is no need to hold the routing table
static during transmission of the packet.
After all, you can hold the
routing table static all you want, but that won't keep the external
Internet from changing, and it is the state of the external Internet
that really matters.
In addition, routing entries are typically added
or deleted, rather than being modified in place.
This is a rare example
of the finite speed of light and the non-zero size of atoms actually
helping make synchronization be lighter weight.

A straightforward example of this type of RCU use case may be found in
the system-call auditing support.
For example, a reader-writer locked
implementation of \co{audit_filter_task()} might be as follows:

\begin{VerbatimU}[breaklines=true]
	static enum audit_state audit_filter_task(struct task_struct *tsk, char **key)
	{
		struct audit_entry *e;
		enum audit_state   state;

		read_lock(&auditsc_lock);
		/* Note: audit_filter_mutex held by caller. */
		list_for_each_entry(e, &audit_tsklist, list) {
			if (audit_filter_rules(tsk, &e->rule, NULL, &state)) {
				if (state == AUDIT_STATE_RECORD)
					*key = kstrdup(e->rule.filterkey, GFP_ATOMIC);
				read_unlock(&auditsc_lock);
				return state;
			}
		}
		read_unlock(&auditsc_lock);
		return AUDIT_BUILD_CONTEXT;
	}
\end{VerbatimU}

Here the list is searched under the lock, but the lock is dropped before
the corresponding value is returned.
By the time that this value is acted
on, the list may well have been modified.
This makes sense, since if
you are turning auditing off, it is OK to audit a few extra system calls.

This means that RCU can be easily applied to the read side, as follows:

\begin{VerbatimU}[breaklines=true]
	static enum audit_state audit_filter_task(struct task_struct *tsk, char **key)
	{
		struct audit_entry *e;
		enum audit_state   state;

		rcu_read_lock();
		/* Note: audit_filter_mutex held by caller. */
		list_for_each_entry_rcu(e, &audit_tsklist, list) {
			if (audit_filter_rules(tsk, &e->rule, NULL, &state)) {
				if (state == AUDIT_STATE_RECORD)
					*key = kstrdup(e->rule.filterkey, GFP_ATOMIC);
				rcu_read_unlock();
				return state;
			}
		}
		rcu_read_unlock();
		return AUDIT_BUILD_CONTEXT;
	}
\end{VerbatimU}

The \co{read_lock()} and \co{read_unlock()} calls have become \co{rcu_read_lock()}
and \co{rcu_read_unlock()}, respectively, and the \co{list_for_each_entry()}
has become \co{list_for_each_entry_rcu()}.
The \textbf{\co{_rcu()}} list-traversal
primitives add \co{READ_ONCE()} and diagnostic checks for incorrect use
outside of an RCU read-side critical section.

The changes to the update side are also straightforward.
A reader-writer lock
might be used as follows for deletion and insertion in these simplified
versions of \co{audit_del_rule()} and \co{audit_add_rule()}:

\begin{VerbatimU}[samepage=false]
	static inline int audit_del_rule(struct audit_rule *rule,
	                                 struct list_head *list)
	{
		struct audit_entry *e;

		write_lock(&auditsc_lock);
		list_for_each_entry(e, list, list) {
			if (!audit_compare_rule(rule, &e->rule)) {
				list_del(&e->list);
				write_unlock(&auditsc_lock);
				return 0;
			}
		}
		write_unlock(&auditsc_lock);
		return -EFAULT;	        /* No matching rule */
	}

	static inline int audit_add_rule(struct audit_entry *entry,
	                                 struct list_head *list)
	{
		write_lock(&auditsc_lock);
		if (entry->rule.flags & AUDIT_PREPEND) {
			entry->rule.flags &= ~AUDIT_PREPEND;
			list_add(&entry->list, list);
		} else {
			list_add_tail(&entry->list, list);
		}
		write_unlock(&auditsc_lock);
		return 0;
	}
\end{VerbatimU}

Following are the RCU equivalents for these two functions::

\begin{VerbatimU}[breaklines=true,samepage=false]
	static inline int audit_del_rule(struct audit_rule *rule,
	                                 struct list_head *list)
	{
		struct audit_entry *e;

		/* No need to use the _rcu iterator here, since this is the only
		 * deletion routine. */
		list_for_each_entry(e, list, list) {
			if (!audit_compare_rule(rule, &e->rule)) {
				list_del_rcu(&e->list);
				call_rcu(&e->rcu, audit_free_rule);
				return 0;
			}
		}
		return -EFAULT;         /* No matching rule */
	}

	static inline int audit_add_rule(struct audit_entry *entry,
	                                 struct list_head *list)
	{
		if (entry->rule.flags & AUDIT_PREPEND) {
			entry->rule.flags &= ~AUDIT_PREPEND;
			list_add_rcu(&entry->list, list);
		} else {
			list_add_tail_rcu(&entry->list, list);
		}
		return 0;
	}
\end{VerbatimU}

Normally, the \co{write_lock()} and \co{write_unlock()} would be replaced by a
\co{spin_lock()} and a \co{spin_unlock()}.
But in this case, all callers hold
\co{audit_filter_mutex}, so no additional locking is required.
The
\co{auditsc_lock} can therefore be eliminated, since use of RCU eliminates the
need for writers to exclude readers.

The \co{list_del()}, \co{list_add()}, and \co{list_add_tail()} primitives have been
replaced by \co{list_del_rcu()}, \co{list_add_rcu()}, and \co{list_add_tail_rcu()}.
The \textbf{\co{_rcu()}} list-manipulation primitives add memory barriers that are
needed on weakly ordered CPUs.
The \co{list_del_rcu()} primitive omits the
pointer poisoning debug-assist code that would otherwise cause concurrent
readers to fail spectacularly.

So, when readers can tolerate stale data and when entries are either added or
deleted, without in-place modification, it is very easy to use RCU\@!


\subsection{Example 3: Handling In-Place Updates}

The system-call auditing code does not update auditing rules in place.
However,
if it did, the reader-writer-locked code to do so might look as follows
(assuming only \co{field_count} is updated, otherwise, the added fields would
need to be filled in):

\begin{VerbatimU}
	static inline int audit_upd_rule(struct audit_rule *rule,
	                                 struct list_head *list,
	                                 __u32 newaction,
	                                 __u32 newfield_count)
	{
		struct audit_entry *e;
		struct audit_entry *ne;

		write_lock(&auditsc_lock);
		/* Note: audit_filter_mutex held by caller. */
		list_for_each_entry(e, list, list) {
			if (!audit_compare_rule(rule, &e->rule)) {
				e->rule.action = newaction;
				e->rule.field_count = newfield_count;
				write_unlock(&auditsc_lock);
				return 0;
			}
		}
		write_unlock(&auditsc_lock);
		return -EFAULT;	        /* No matching rule */
	}
\end{VerbatimU}

The RCU version creates a copy, updates the copy, then replaces the old
entry with the newly updated entry.
This sequence of actions, allowing
concurrent reads while making a copy to perform an update, is what gives
RCU (\emph{read-copy update}) its name.

The RCU version of \co{audit_upd_rule()} is as follows:

\begin{VerbatimU}
	static inline int audit_upd_rule(struct audit_rule *rule,
	                                 struct list_head *list,
	                                 __u32 newaction,
	                                 __u32 newfield_count)
	{
		struct audit_entry *e;
		struct audit_entry *ne;

		list_for_each_entry(e, list, list) {
			if (!audit_compare_rule(rule, &e->rule)) {
				ne = kmalloc(sizeof(*entry), GFP_ATOMIC);
				if (ne == NULL)
					return -ENOMEM;
				audit_copy_rule(&ne->rule, &e->rule);
				ne->rule.action = newaction;
				ne->rule.field_count = newfield_count;
				list_replace_rcu(&e->list, &ne->list);
				call_rcu(&e->rcu, audit_free_rule);
				return 0;
			}
		}
		return -EFAULT;	        /* No matching rule */
	}
\end{VerbatimU}

Again, this assumes that the caller holds \co{audit_filter_mutex}.
Normally, the
writer lock would become a spinlock in this sort of code.

The \co{update_lsm_rule()} does something very similar, for those who would
prefer to look at real Linux-kernel code.

Another use of this pattern can be found in the openswitch driver's
\emph{connection tracking table} code in \co{ct_limit_set()}.
The table holds connection tracking
entries and has a limit on the maximum entries.
There is one such table
per-zone and hence one \emph{limit} per zone.
The zones are mapped to their limits
through a hashtable using an RCU-managed hlist for the hash chains.
When a new
limit is set, a new limit object is allocated and \co{ct_limit_set()} is called
to replace the old limit object with the new one using \co{list_replace_rcu()}.
The old limit object is then freed after a grace period using \co{kfree_rcu()}.


\subsection{Example 4: Eliminating Stale Data}

The auditing example above tolerates stale data, as do most algorithms
that are tracking external state.
After all, given there is a delay
from the time the external state changes before Linux becomes aware
of the change, and so as noted earlier, a small quantity of additional
RCU-induced staleness is generally not a problem.

However, there are many examples where stale data cannot be tolerated.
One example in the Linux kernel is the System V IPC (see the \co{shm_lock()}
function in \path{ipc/shm.c}).
This code checks a \emph{deleted} flag under a
per-entry spinlock, and, if the \emph{deleted} flag is set, pretends that the
entry does not exist.
For this to be helpful, the search function must
return holding the per-entry spinlock, as \co{shm_lock()} does in fact do.

\QuickQuiz{
	For the deleted-flag technique to be helpful, why is it necessary
	to hold the per-entry lock while returning from the search function?
}\QuickQuizAnswer{
	If the search function drops the per-entry lock before returning,
	then the caller will be processing stale data in any case.
	If it
	is really OK to be processing stale data, then you don't need a
	\emph{deleted} flag.
	If processing stale data really is a problem,
	then you need to hold the per-entry lock across all of the code
	that uses the value that was returned.
}\QuickQuizEnd

If the system-call audit module were to ever need to reject stale data, one way
to accomplish this would be to add a \co{deleted} flag and a \co{lock} spinlock to the
\co{audit_entry} structure, and modify \co{audit_filter_task()} as follows:

\begin{VerbatimU}
	static enum audit_state audit_filter_task(struct task_struct *tsk)
	{
		struct audit_entry *e;
		enum audit_state   state;

		rcu_read_lock();
		list_for_each_entry_rcu(e, &audit_tsklist, list) {
			if (audit_filter_rules(tsk, &e->rule, NULL, &state)) {
				spin_lock(&e->lock);
				if (e->deleted) {
					spin_unlock(&e->lock);
					rcu_read_unlock();
					return AUDIT_BUILD_CONTEXT;
				}
				rcu_read_unlock();
				if (state == AUDIT_STATE_RECORD)
					*key = kstrdup(e->rule.filterkey, GFP_ATOMIC);
				return state;
			}
		}
		rcu_read_unlock();
		return AUDIT_BUILD_CONTEXT;
	}
\end{VerbatimU}

The \co{audit_del_rule()} function would need to set the \co{deleted} flag under the
spinlock as follows:

\begin{VerbatimU}
	static inline int audit_del_rule(struct audit_rule *rule,
	                                 struct list_head *list)
	{
		struct audit_entry *e;

		/* No need to use the _rcu iterator here, since this
		 * is the only deletion routine. */
		list_for_each_entry(e, list, list) {
			if (!audit_compare_rule(rule, &e->rule)) {
				spin_lock(&e->lock);
				list_del_rcu(&e->list);
				e->deleted = 1;
				spin_unlock(&e->lock);
				call_rcu(&e->rcu, audit_free_rule);
				return 0;
			}
		}
		return -EFAULT;	        /* No matching rule */
	}
\end{VerbatimU}

This too assumes that the caller holds \co{audit_filter_mutex}.

Note that this example assumes that entries are only added and deleted.
Additional mechanism is required to deal correctly with the update-in-place
performed by \co{audit_upd_rule()}.
For one thing, \co{audit_upd_rule()} would
need to hold the locks of both the old \co{audit_entry} and its replacement
while executing the \co{list_replace_rcu()}.


\subsection{Example 5: Skipping Stale Objects}

For some use cases, reader performance can be improved by skipping
stale objects during read-side list traversal, where stale objects
are those that will be removed and destroyed after one or more grace
periods.
One such example can be found in the \co{timerfd} subsystem.
When a
\co{CLOCK_REALTIME} clock is reprogrammed (for example due to setting
of the system time) then all programmed \co{timerfd}s that depend on
this clock get triggered and processes waiting on them are awakened in
advance of their scheduled expiry.
To facilitate this, all such timers
are added to an RCU-managed \co{cancel_list} when they are setup in
\co{timerfd_setup_cancel()}:

\begin{VerbatimU}
	static void timerfd_setup_cancel(struct timerfd_ctx *ctx, int flags)
	{
		spin_lock(&ctx->cancel_lock);
		if ((ctx->clockid == CLOCK_REALTIME ||
		     ctx->clockid == CLOCK_REALTIME_ALARM) &&
		    (flags & TFD_TIMER_ABSTIME) &&
		    (flags & TFD_TIMER_CANCEL_ON_SET)) {
			if (!ctx->might_cancel) {
				ctx->might_cancel = true;
				spin_lock(&cancel_lock);
				list_add_rcu(&ctx->clist, &cancel_list);
				spin_unlock(&cancel_lock);
			}
		} else {
			__timerfd_remove_cancel(ctx);
		}
		spin_unlock(&ctx->cancel_lock);
	}
\end{VerbatimU}

When a \co{timerfd} is freed (\co{fd} is closed), then the \co{might_cancel}
flag of the \co{timerfd} object is cleared, the object removed from the
\co{cancel_list} and destroyed, as shown in this simplified and inlined
version of \co{timerfd_release()}:

\begin{VerbatimU}
	int timerfd_release(struct inode *inode, struct file *file)
	{
		struct timerfd_ctx *ctx = file->private_data;

		spin_lock(&ctx->cancel_lock);
		if (ctx->might_cancel) {
			ctx->might_cancel = false;
			spin_lock(&cancel_lock);
			list_del_rcu(&ctx->clist);
			spin_unlock(&cancel_lock);
		}
		spin_unlock(&ctx->cancel_lock);

		if (isalarm(ctx))
			alarm_cancel(&ctx->t.alarm);
		else
			hrtimer_cancel(&ctx->t.tmr);
		kfree_rcu(ctx, rcu);
		return 0;
	}
\end{VerbatimU}

If the \co{CLOCK_REALTIME} clock is set, for example by a time server, the
hrtimer framework calls \co{timerfd_clock_was_set()} which walks the
\co{cancel_list} and wakes up processes waiting on the \co{timerfd}.
While iterating
the \co{cancel_list}, the \co{might_cancel} flag is consulted to skip stale
objects:

\begin{VerbatimU}
	void timerfd_clock_was_set(void)
	{
		ktime_t moffs = ktime_mono_to_real(0);
		struct timerfd_ctx *ctx;
		unsigned long flags;

		rcu_read_lock();
		list_for_each_entry_rcu(ctx, &cancel_list, clist) {
			if (!ctx->might_cancel)
				continue;
			spin_lock_irqsave(&ctx->wqh.lock, flags);
			if (ctx->moffs != moffs) {
				ctx->moffs = KTIME_MAX;
				ctx->ticks++;
				wake_up_locked_poll(&ctx->wqh, EPOLLIN);
			}
			spin_unlock_irqrestore(&ctx->wqh.lock, flags);
		}
		rcu_read_unlock();
	}
\end{VerbatimU}

The key point is that because RCU-protected traversal of the
\co{cancel_list} happens concurrently with object addition and removal,
sometimes the traversal can access an object that has been removed from
the list.
In this example, a flag is used to skip such objects.


\subsection{Summary}

Read-mostly list-based data structures that can tolerate stale data are
the most amenable to use of RCU\@.
The simplest case is where entries are
either added or deleted from the data structure (or atomically modified
in place), but non-atomic in-place modifications can be handled by making
a copy, updating the copy, then replacing the original with the copy.
If stale data cannot be tolerated, then a \emph{deleted} flag may be used
in conjunction with a per-entry spinlock in order to allow the search
function to reject newly deleted data.
