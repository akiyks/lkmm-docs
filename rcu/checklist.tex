% SPDX-License-Identifier: GPL-2.0

\section{Review checklist for RCU patches}

This document contains a checklist for producing and reviewing patches
that make use of RCU\@.
Violating any of the rules listed below will
result in the same sorts of problems that leaving out a locking primitive
would cause.
This list is based on experiences reviewing such patches
over a rather long period of time, but improvements are always welcome!

\begin{enumerate}[start=0]
\item	Is RCU being applied to a read-mostly situation?
	If the data
	structure is updated more than about 10\,\% of the time, then you
	should strongly consider some other approach, unless detailed
	performance measurements show that RCU is nonetheless the right
	tool for the job.
	Yes, RCU does reduce read-side overhead by
	increasing write-side overhead, which is exactly why normal uses
	of RCU will do much more reading than updating.

	Another exception is where performance is not an issue, and RCU
	provides a simpler implementation.
	An example of this situation
	is the dynamic NMI code in the Linux 2.6 kernel, at least on
	architectures where NMIs are rare.

	Yet another exception is where the low real-time latency of RCU's
	read-side primitives is critically important.

	One final exception is where RCU readers are used to prevent
	the ABA problem (\url{https://en.wikipedia.org/wiki/ABA_problem})
	for lockless updates.
	This does result in the mildly
	counter-intuitive situation where \co{rcu_read_lock()} and
	\co{rcu_read_unlock()} are used to protect updates, however, this
	approach can provide the same simplifications to certain types
	of lockless algorithms that garbage collectors do.

\item	Does the update code have proper mutual exclusion?

	RCU does allow \emph{readers} to run (almost) naked, but
	\emph{writers} must
	still use some sort of mutual exclusion, such as:

	\begin{enumerate}
	\item	locking,
	\item	atomic operations, or
	\item	restricting updates to a single task.
	\end{enumerate}

	If you choose (b), be prepared to describe how you have handled
	memory barriers on weakly ordered machines (pretty much all of
	them---even x86 allows later loads to be reordered to precede
	earlier stores), and be prepared to explain why this added
	complexity is worthwhile.
	If you choose (c), be prepared to
	explain how this single task does not become a major bottleneck
	on large systems (for example, if the task is updating information
	relating to itself that other tasks can read, there by definition
	can be no bottleneck).
	Note that the definition of ``large'' has
	changed significantly:
	Eight CPUs was ``large'' in the year 2000,
	but a hundred CPUs was unremarkable in 2017.

\item	Do the RCU read-side critical sections make proper use of
	\co{rcu_read_lock()} and friends?
	These primitives are needed
	to prevent grace periods from ending prematurely, which
	could result in data being unceremoniously freed out from
	under your read-side code, which can greatly increase the
	actuarial risk of your kernel.

	As a rough rule of thumb, any dereference of an RCU-protected
	pointer must be covered by \co{rcu_read_lock()}, \co{rcu_read_lock_bh()},
	\co{rcu_read_lock_sched()}, or by the appropriate update-side lock.
	Explicit disabling of preemption (\co{preempt_disable()}, for example)
	can serve as \co{rcu_read_lock_sched()}, but is less readable and
	prevents lockdep from detecting locking issues.
	Acquiring a
	spinlock also enters an RCU read-side critical section.

	Please note that you \emph{cannot} rely on code known to be built
	only in non-preemptible kernels.
	Such code can and will break,
	especially in kernels built with \co{CONFIG_PREEMPT_COUNT=y}.

	Letting RCU-protected pointers ``leak'' out of an RCU read-side
	critical section is every bit as bad as letting them leak out
	from under a lock.
	Unless, of course, you have arranged some
	other means of protection, such as a lock or a reference count
	\emph{before} letting them out of the RCU read-side critical section.

\item	Does the update code tolerate concurrent accesses?

	The whole point of RCU is to permit readers to run without
	any locks or atomic operations.
	This means that readers will
	be running while updates are in progress.
	There are a number
	of ways to handle this concurrency, depending on the situation:

	\begin{enumerate}
	\item	Use the RCU variants of the list and hlist update
		primitives to add, remove, and replace elements on
		an RCU-protected list.
		Alternatively, use the other
		RCU-protected data structures that have been added to
		the Linux kernel.

		This is almost always the best approach.

	\item	Proceed as in (a) above, but also maintain per-element
		locks (that are acquired by both readers and writers)
		that guard per-element state.
		Fields that the readers
		refrain from accessing can be guarded by some other lock
		acquired only by updaters, if desired.

		This also works quite well.

	\item	Make updates appear atomic to readers.
		For example,
		pointer updates to properly aligned fields will
		appear atomic, as will individual atomic primitives.
		Sequences of operations performed under a lock will \emph{not}
		appear to be atomic to RCU readers, nor will sequences
		of multiple atomic primitives.
		One alternative is to
		move multiple individual fields to a separate structure,
		thus solving the multiple-field problem by imposing an
		additional level of indirection.

		This can work, but is starting to get a bit tricky.

	\item	Carefully order the updates and the reads so that readers
		see valid data at all phases of the update.
		This is often
		more difficult than it sounds, especially given modern
		CPUs' tendency to reorder memory references.
		One must
		usually liberally sprinkle memory-ordering operations
		through the code, making it difficult to understand and
		to test.
		Where it works, it is better to use things
		like \co{smp_store_release()} and \co{smp_load_acquire()}, but in
		some cases the \co{smp_mb()} full memory barrier is required.

		As noted earlier, it is usually better to group the
		changing data into a separate structure, so that the
		change may be made to appear atomic by updating a pointer
		to reference a new structure containing updated values.
	\end{enumerate}

\item	Weakly ordered CPUs pose special challenges.
	Almost all CPUs
	are weakly ordered---even x86 CPUs allow later loads to be
	reordered to precede earlier stores.
	RCU code must take all of
	the following measures to prevent memory-corruption problems:

	\begin{enumerate}
	\item	Readers must maintain proper ordering of their memory
		accesses.
		The \co{rcu_dereference()} primitive ensures that
		the CPU picks up the pointer before it picks up the data
		that the pointer points to.
		This really is necessary
		on Alpha CPUs.

		The \co{rcu_dereference()} primitive is also an excellent
		documentation aid, letting the person reading the
		code know exactly which pointers are protected by RCU\@.
		Please note that compilers can also reorder code, and
		they are becoming increasingly aggressive about doing
		just that.
		The \co{rcu_dereference()} primitive therefore also
		prevents destructive compiler optimizations.
		However,
		with a bit of devious creativity, it is possible to
		mishandle the return value from \co{rcu_dereference()}.
		Please see \path{rcu_dereference.rst} for more information.

		The \co{rcu_dereference()} primitive is used by the
		various \qtco{_rcu()} list-traversal primitives, such
		as the \co{list_for_each_entry_rcu()}.
		Note that it is
		perfectly legal (if redundant) for update-side code to
		use \co{rcu_dereference()} and the \qtco{_rcu()} list-traversal
		primitives.
		This is particularly useful in code that
		is common to readers and updaters.
		However, lockdep
		will complain if you access \co{rcu_dereference()} outside
		of an RCU read-side critical section.
		See \path{lockdep.rst}
		to learn what to do about this.

		Of course, neither \co{rcu_dereference()} nor the \qtco{_rcu()}
		list-traversal primitives can substitute for a good
		concurrency design coordinating among multiple updaters.

	\item	If the list macros are being used, the \co{list_add_tail_rcu()}
		and \co{list_add_rcu()} primitives must be used in order
		to prevent weakly ordered machines from misordering
		structure initialization and pointer planting.
		Similarly, if the hlist macros are being used, the
		\co{hlist_add_head_rcu()} primitive is required.

	\item	If the list macros are being used, the \co{list_del_rcu()}
		primitive must be used to keep \co{list_del()}'s pointer
		poisoning from inflicting toxic effects on concurrent
		readers.
		Similarly, if the hlist macros are being used,
		the \co{hlist_del_rcu()} primitive is required.

		The \co{list_replace_rcu()} and \co{hlist_replace_rcu()} primitives
		may be used to replace an old structure with a new one
		in their respective types of RCU-protected lists.

	\item	Rules similar to (4b) and (4c) apply to the \qtco{hlist_nulls}
		type of RCU-protected linked lists.

	\item	Updates must ensure that initialization of a given
		structure happens before pointers to that structure are
		publicized.
		Use the \co{rcu_assign_pointer()} primitive
		when publicizing a pointer to a structure that can
		be traversed by an RCU read-side critical section.
	\end{enumerate}

\item	If any of \co{call_rcu()}, \co{call_srcu()}, \co{call_rcu_tasks()}, or
	\co{call_rcu_tasks_trace()} is used, the callback function may be
	invoked from softirq context, and in any case with bottom halves
	disabled.  In particular, this callback function cannot block.
	If you need the callback to block, run that code in a workqueue
	handler scheduled from the callback.
	The \co{queue_rcu_work()}
	function does this for you in the case of \co{call_rcu()}.

\item	Since \co{synchronize_rcu()} can block, it cannot be called
	from any sort of irq context.
	The same rule applies
	for \co{synchronize_srcu()}, \co{synchronize_rcu_expedited()},
	\co{synchronize_srcu_expedited()}, \co{synchronize_rcu_tasks()},
	\co{synchronize_rcu_tasks_rude()}, and \co{synchronize_rcu_tasks_trace()}.

	The expedited forms of these primitives have the same semantics
	as the non-expedited forms, but expediting is more CPU intensive.
	Use of the expedited primitives should be restricted to rare
	configuration-change operations that would not normally be
	undertaken while a real-time workload is running.
	Note that
	IPI-sensitive real-time workloads can use the \co{rcupdate.rcu_normal}
	kernel boot parameter to completely disable expedited grace
	periods, though this might have performance implications.

	In particular, if you find yourself invoking one of the expedited
	primitives repeatedly in a loop, please do everyone a favor:
	Restructure your code so that it batches the updates, allowing
	a single non-expedited primitive to cover the entire batch.
	This will very likely be faster than the loop containing the
	expedited primitive, and will be much much easier on the rest
	of the system, especially to real-time workloads running on the
	rest of the system.
	Alternatively, instead use asynchronous
	primitives such as \co{call_rcu()}.

\item	As of v4.20, a given kernel implements only one RCU flavor, which
	is RCU-sched for \co{PREEMPTION=n} and RCU-preempt for \co{PREEMPTION=y}.
	If the updater uses \co{call_rcu()} or \co{synchronize_rcu()}, then
	the corresponding readers may use:
	\begin{enumerate*}[(1)]
	\item \tco{rcu_read_lock()} and
	\tco{rcu_read_unlock()},
	\item any pair of primitives that disables
	and re-enables softirq, for example, \tco{rcu_read_lock_bh()} and
	\tco{rcu_read_unlock_bh()}, or
	\item any pair of primitives that disables
	and re-enables preemption, for example, \tco{rcu_read_lock_sched()} and
	\tco{rcu_read_unlock_sched()}.
	\end{enumerate*}
	If the updater uses \co{synchronize_srcu()}
	or \co{call_srcu()}, then the corresponding readers must use
	\co{srcu_read_lock()} and \co{srcu_read_unlock()}, and with the same
	\co{srcu_struct}.
	The rules for the expedited RCU grace-period-wait
	primitives are the same as for their non-expedited counterparts.

	Similarly, it is necessary to correctly use the RCU Tasks flavors:

	\begin{enumerate}
	\item	If the updater uses \co{synchronize_rcu_tasks()} or
		\co{call_rcu_tasks()}, then the readers must refrain from
		executing voluntary context switches, that is, from
		blocking.

	\item	If the updater uses \co{call_rcu_tasks_trace()}
		or \co{synchronize_rcu_tasks_trace()}, then the
		corresponding readers must use \co{rcu_read_lock_trace()}
		and \co{rcu_read_unlock_trace()}.

	\item	If an updater uses \co{synchronize_rcu_tasks_rude()},
		then the corresponding readers must use anything that
		disables preemption, for example, \co{preempt_disable()}
		and \co{preempt_enable()}.
	\end{enumerate}

	Mixing things up will result in confusion and broken kernels, and
	has even resulted in an exploitable security issue.
	Therefore,
	when using non-obvious pairs of primitives, commenting is
	of course a must.
	One example of non-obvious pairing is
	the XDP feature in networking, which calls BPF programs from
	network-driver NAPI (softirq) context.
	BPF relies heavily on RCU
	protection for its data structures, but because the BPF program
	invocation happens entirely within a single \co{local_bh_disable()}
	section in a NAPI poll cycle, this usage is safe.
	The reason
	that this usage is safe is that readers can use anything that
	disables BH when updaters use \co{call_rcu()} or \co{synchronize_rcu()}.

\item	Although \co{synchronize_rcu()} is slower than is \co{call_rcu()},
	it usually results in simpler code.
	So, unless update
	performance is critically important, the updaters cannot block,
	or the latency of \co{synchronize_rcu()} is visible from userspace,
	\co{synchronize_rcu()} should be used in preference to \co{call_rcu()}.
	Furthermore, \co{kfree_rcu()} and \co{kvfree_rcu()} usually result
	in even simpler code than does \co{synchronize_rcu()} without
	\co{synchronize_rcu()}'s multi-millisecond latency.
	So please take
	advantage of \co{kfree_rcu()}'s and \co{kvfree_rcu()}'s ``fire and forget''
	memory-freeing capabilities where it applies.

	An especially important property of the \co{synchronize_rcu()}
	primitive is that it automatically self-limits{:} if grace periods
	are delayed for whatever reason, then the \co{synchronize_rcu()}
	primitive will correspondingly delay updates.
	In contrast,
	code using \co{call_rcu()} should explicitly limit update rate in
	cases where grace periods are delayed, as failing to do so can
	result in excessive realtime latencies or even OOM conditions.

	Ways of gaining this self-limiting property when using \co{call_rcu()},
	\co{kfree_rcu()}, or \co{kvfree_rcu()} include:

	\begin{enumerate}
	\item	Keeping a count of the number of data-structure elements
		used by the RCU-protected data structure, including
		those waiting for a grace period to elapse.
		Enforce a
		limit on this number, stalling updates as needed to allow
		previously deferred frees to complete.
		Alternatively,
		limit only the number awaiting deferred free rather than
		the total number of elements.

		One way to stall the updates is to acquire the update-side
		mutex.
		(Don't try this with a spinlock---other CPUs
		spinning on the lock could prevent the grace period
		from ever ending.)
		Another way to stall the updates
		is for the updates to use a wrapper function around
		the memory allocator, so that this wrapper function
		simulates OOM when there is too much memory awaiting an
		RCU grace period.
		There are of course many other
		variations on this theme.

	\item	Limiting update rate.
		For example, if updates occur only
		once per hour, then no explicit rate limiting is
		required, unless your system is already badly broken.
		Older versions of the dcache subsystem take this approach,
		guarding updates with a global lock, limiting their rate.

	\item	Trusted update---if updates can only be done manually by
		superuser or some other trusted user, then it might not
		be necessary to automatically limit them.
		The theory
		here is that superuser already has lots of ways to crash
		the machine.

	\item	Periodically invoke \co{rcu_barrier()}, permitting a limited
		number of updates per grace period.
	\end{enumerate}

	The same cautions apply to \co{call_srcu()}, \co{call_rcu_tasks()}, and
	\co{call_rcu_tasks_trace()}.
	This is why there is an \co{srcu_barrier()},
	and \co{rcu_barrier_tasks()}, respectively.

	Note that although these primitives do take action to avoid
	memory exhaustion when any given CPU has too many callbacks,
	a determined user or administrator can still exhaust memory.
	This is especially the case if a system with a large number of
	CPUs has been configured to offload all of its RCU callbacks onto
	a single CPU, or if the system has relatively little free memory.

\item	All RCU list-traversal primitives, which include
	\co{rcu_dereference()}, \co{list_for_each_entry_rcu()}, and
	\co{list_for_each_safe_rcu()}, must be either within an RCU read-side
	critical section or must be protected by appropriate update-side
	locks.
	RCU read-side critical sections are delimited by
	\co{rcu_read_lock()} and \co{rcu_read_unlock()}, or by similar primitives
	such as \co{rcu_read_lock_bh()} and \co{rcu_read_unlock_bh()}, in which
	case the matching \co{rcu_dereference()} primitive must be used in
	order to keep lockdep happy, in this case, \co{rcu_dereference_bh()}.

	The reason that it is permissible to use RCU list-traversal
	primitives when the update-side lock is held is that doing so
	can be quite helpful in reducing code bloat when common code is
	shared between readers and updaters.
	Additional primitives
	are provided for this case, as discussed in \path{lockdep.rst}.

	One exception to this rule is when data is only ever added to
	the linked data structure, and is never removed during any
	time that readers might be accessing that structure.
	In such
	cases, \co{READ_ONCE()} may be used in place of \co{rcu_dereference()}
	and the read-side markers (\co{rcu_read_lock()} and \co{rcu_read_unlock()},
	for example) may be omitted.

\item	Conversely, if you are in an RCU read-side critical section,
	and you don't hold the appropriate update-side lock, you \emph{must}
	use the \qtco{_rcu()} variants of the list macros.
	Failing to do so
	will break Alpha, cause aggressive compilers to generate bad code,
	and confuse people trying to understand your code.

\item	Any lock acquired by an RCU callback must be acquired elsewhere
	with softirq disabled, e.g., via \co{spin_lock_bh()}.
	Failing to
	disable softirq on a given acquisition of that lock will result
	in deadlock as soon as the RCU softirq handler happens to run
	your RCU callback while interrupting that acquisition's critical
	section.

\item	RCU callbacks can be and are executed in parallel.
	In many cases,
	the callback code simply wrappers around \co{kfree()}, so that this
	is not an issue (or, more accurately, to the extent that it is
	an issue, the memory-allocator locking handles it).
	However,
	if the callbacks do manipulate a shared data structure, they
	must use whatever locking or other synchronization is required
	to safely access and/or modify that data structure.

	Do not assume that RCU callbacks will be executed on the same
	CPU that executed the corresponding \co{call_rcu()}, \co{call_srcu()},
	\co{call_rcu_tasks()}, or \co{call_rcu_tasks_trace()}.
	For example, if
	a given CPU goes offline while having an RCU callback pending,
	then that RCU callback will execute on some surviving CPU\@.
	(If this was not the case, a self-spawning RCU callback would
	prevent the victim CPU from ever going offline.)
	Furthermore,
	CPUs designated by \co{rcu_nocbs=} might well \emph{always} have their
	RCU callbacks executed on some other CPUs, in fact, for some
	real-time workloads, this is the whole point of using the
	\co{rcu_nocbs=} kernel boot parameter.

	In addition, do not assume that callbacks queued in a given order
	will be invoked in that order, even if they all are queued on the
	same CPU\@.
	Furthermore, do not assume that same-CPU callbacks will
	be invoked serially.
	For example, in recent kernels, CPUs can be
	switched between offloaded and de-offloaded callback invocation,
	and while a given CPU is undergoing such a switch, its callbacks
	might be concurrently invoked by that CPU's softirq handler and
	that CPU's rcuo kthread.
	At such times, that CPU's callbacks
	might be executed both concurrently and out of order.

\item	Unlike most flavors of RCU, it \emph{is} permissible to block in an
	SRCU read-side critical section (demarked by \co{srcu_read_lock()}
	and \co{srcu_read_unlock()}), hence the ``SRCU'': ``sleepable RCU''.
	Please note that if you don't need to sleep in read-side critical
	sections, you should be using RCU rather than SRCU, because RCU
	is almost always faster and easier to use than is SRCU\@.

	Also unlike other forms of RCU, explicit initialization and
	cleanup is required either at build time via \co{DEFINE_SRCU()}
	or \co{DEFINE_STATIC_SRCU()} or at runtime via \co{init_srcu_struct()}
	and \co{cleanup_srcu_struct()}.
	These last two are passed a
	\qtco{struct srcu_struct} that defines the scope of a given
	SRCU domain.
	Once initialized, the \co{srcu_struct} is passed
	to \co{srcu_read_lock()}, \co{srcu_read_unlock()}, \co{synchronize_srcu()},
	\co{synchronize_srcu_expedited()}, and \co{call_srcu()}.
	A given
	\co{synchronize_srcu()} waits only for SRCU read-side critical
	sections governed by \co{srcu_read_lock()} and \co{srcu_read_unlock()}
	calls that have been passed the same \co{srcu_struct}.
	This property
	is what makes sleeping read-side critical sections tolerable---%
	a given subsystem delays only its own updates, not those of other
	subsystems using SRCU\@.
	Therefore, SRCU is less prone to OOM the
	system than RCU would be if RCU's read-side critical sections
	were permitted to sleep.

	The ability to sleep in read-side critical sections does not
	come for free.
	First, corresponding \co{srcu_read_lock()} and
	\co{srcu_read_unlock()} calls must be passed the same \co{srcu_struct}.
	Second, grace-period-detection overhead is amortized only
	over those updates sharing a given \co{srcu_struct}, rather than
	being globally amortized as they are for other forms of RCU\@.
	Therefore, SRCU should be used in preference to \co{rw_semaphore}
	only in extremely read-intensive situations, or in situations
	requiring SRCU's read-side deadlock immunity or low read-side
	realtime latency.
	You should also consider \co{percpu_rw_semaphore}
	when you need lightweight readers.

	SRCU's expedited primitive (\co{synchronize_srcu_expedited()})
	never sends IPIs to other CPUs, so it is easier on
	real-time workloads than is \co{synchronize_rcu_expedited()}.

	It is also permissible to sleep in RCU Tasks Trace read-side
	critical section, which are delimited by \co{rcu_read_lock_trace()} and
	\co{rcu_read_unlock_trace()}.
	However, this is a specialized flavor
	of RCU, and you should not use it without first checking with
	its current users.
	In most cases, you should instead use SRCU\@.

	Note that \co{rcu_assign_pointer()} relates to SRCU just as it does to
	other forms of RCU, but instead of \co{rcu_dereference()} you should
	use \co{srcu_dereference()} in order to avoid lockdep splats.

\item	The whole point of \co{call_rcu()}, \co{synchronize_rcu()}, and friends
	is to wait until all pre-existing readers have finished before
	carrying out some otherwise-destructive operation.
	It is
	therefore critically important to \emph{first} remove any path
	that readers can follow that could be affected by the
	destructive operation, and \emph{only then} invoke \co{call_rcu()},
	\co{synchronize_rcu()}, or friends.

	Because these primitives only wait for pre-existing readers, it
	is the caller's responsibility to guarantee that any subsequent
	readers will execute safely.

\item	The various RCU read-side primitives do \emph{not} necessarily contain
	memory barriers.
	You should therefore plan for the CPU
	and the compiler to freely reorder code into and out of RCU
	read-side critical sections.
	It is the responsibility of the
	RCU update-side primitives to deal with this.

	For SRCU readers, you can use \co{smp_mb__after_srcu_read_unlock()}
	immediately after an \co{srcu_read_unlock()} to get a full barrier.

\item	Use \co{CONFIG_PROVE_LOCKING}, \co{CONFIG_DEBUG_OBJECTS_RCU_HEAD}, and the
	\co{__rcu} sparse checks to validate your RCU code.
	These can help
	find problems as follows:

        \begin{description}[style=nextline]
	\item[\tco{CONFIG_PROVE_LOCKING}:]
		check that accesses to RCU-protected data structures
		are carried out under the proper RCU read-side critical
		section, while holding the right combination of locks,
		or whatever other conditions are appropriate.

	\item[\tco{CONFIG_DEBUG_OBJECTS_RCU_HEAD}:]
		check that you don't pass the same object to \co{call_rcu()}
		(or friends) before an RCU grace period has elapsed
		since the last time that you passed that same object to
		\co{call_rcu()} (or friends).

	\item[\tco{CONFIG_RCU_STRICT_GRACE_PERIOD}:]
		combine with KASAN to check for pointers leaked out
		of RCU read-side critical sections.
		This Kconfig
		option is tough on both performance and scalability,
		and so is limited to four-CPU systems.

	\item[\tco{__rcu sparse checks}:]
		tag the pointer to the RCU-protected data structure
		with \co{__rcu}, and sparse will warn you if you access that
		pointer without the services of one of the variants
		of \co{rcu_dereference()}.
	\end{description}

	These debugging aids can help you find problems that are
	otherwise extremely difficult to spot.

\item	If you pass a callback function defined within a module
	to one of \co{call_rcu()}, \co{call_srcu()}, \co{call_rcu_tasks()}, or
	\co{call_rcu_tasks_trace()}, then it is necessary to wait for all
	pending callbacks to be invoked before unloading that module.
	Note that it is absolutely \emph{not} sufficient to wait for a grace
	period!
	For example, \co{synchronize_rcu()} implementation is \emph{not}
	guaranteed to wait for callbacks registered on other CPUs via
	\co{call_rcu()}.
	Or even on the current CPU if that CPU recently
	went offline and came back online.

	You instead need to use one of the barrier functions:

	\begin{itemize}
	\item	\co{call_rcu()} -> \co{rcu_barrier()}
	\item	\co{call_srcu()} -> \co{srcu_barrier()}
	\item	\co{call_rcu_tasks()} -> \co{rcu_barrier_tasks()}
	\item	\co{call_rcu_tasks_trace()} -> \co{rcu_barrier_tasks_trace()}
        \end{itemize}

	However, these barrier functions are absolutely \emph{not} guaranteed
	to wait for a grace period.
	For example, if there are no
	\co{call_rcu()} callbacks queued anywhere in the system, \co{rcu_barrier()}
	can and will return immediately.

	So if you need to wait for both a grace period and for all
	pre-existing callbacks, you will need to invoke both functions,
	with the pair depending on the flavor of RCU\@:

        \begin{itemize}
	\item	Either \co{synchronize_rcu()} or \co{synchronize_rcu_expedited()},
		together with \co{rcu_barrier()}
	\item	Either \co{synchronize_srcu()} or \co{synchronize_srcu_expedited()},
		together with and \co{srcu_barrier()}
	\item	\co{synchronize_rcu_tasks()} and \co{rcu_barrier_tasks()}
	\item	\co{synchronize_tasks_trace()} and \co{rcu_barrier_tasks_trace()}
	\end{itemize}

	If necessary, you can use something like workqueues to execute
	the requisite pair of functions concurrently.

	See \path{rcubarrier.rst} for more information.
\end{enumerate}
