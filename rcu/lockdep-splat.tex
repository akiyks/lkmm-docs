% SPDX-License-Identifier: GPL-2.0

\section{Lockdep-RCU Splat}
\label{sec:rcu:Lockdep-RCU Splat}

Lockdep-RCU was added to the Linux kernel in early 2010
(\url{http://lwn.net/Articles/371986/}).
This facility checks for some common
misuses of the RCU API, most notably using one of the \co{rcu_dereference()}
family to access an RCU-protected pointer without the proper protection.
When such misuse is detected, an lockdep-RCU splat is emitted.

The usual cause of a lockdep-RCU splat is someone accessing an
RCU-protected data structure without either (1)~being in the right kind of
RCU read-side critical section or (2)~holding the right update-side lock.
This problem can therefore be serious{:} it might result in random memory
overwriting or worse.
There can of course be false positives, this
being the real world and all that.

So let's look at an example RCU lockdep splat from 3.0-rc5, one that
has long since been fixed:

\begin{VerbatimU}[gobble=4]
    =============================
    WARNING: suspicious RCU usage
    -----------------------------
    block/cfq-iosched.c:2776 suspicious rcu_dereference_protected() usage!
\end{VerbatimU}

\noindent%
other info that might help us debug this:

\begin{VerbatimU}[samepage=false,gobble=4]
    rcu_scheduler_active = 1, debug_locks = 0
    3 locks held by scsi_scan_6/1552:
    #0:  (&shost->scan_mutex){+.+.}, at: [<ffffffff8145efca>]
    scsi_scan_host_selected+0x5a/0x150
    #1:  (&eq->sysfs_lock){+.+.}, at: [<ffffffff812a5032>]
    elevator_exit+0x22/0x60
    #2:  (&(&q->__queue_lock)->rlock){-.-.}, at: [<ffffffff812b6233>]
    cfq_exit_queue+0x43/0x190

    stack backtrace:
    Pid: 1552, comm: scsi_scan_6 Not tainted 3.0.0-rc5 #17
    Call Trace:
    [<ffffffff810abb9b>] lockdep_rcu_dereference+0xbb/0xc0
    [<ffffffff812b6139>] __cfq_exit_single_io_context+0xe9/0x120
    [<ffffffff812b626c>] cfq_exit_queue+0x7c/0x190
    [<ffffffff812a5046>] elevator_exit+0x36/0x60
    [<ffffffff812a802a>] blk_cleanup_queue+0x4a/0x60
    [<ffffffff8145cc09>] scsi_free_queue+0x9/0x10
    [<ffffffff81460944>] __scsi_remove_device+0x84/0xd0
    [<ffffffff8145dca3>] scsi_probe_and_add_lun+0x353/0xb10
    [<ffffffff817da069>] ? error_exit+0x29/0xb0
    [<ffffffff817d98ed>] ? _raw_spin_unlock_irqrestore+0x3d/0x80
    [<ffffffff8145e722>] __scsi_scan_target+0x112/0x680
    [<ffffffff812c690d>] ? trace_hardirqs_off_thunk+0x3a/0x3c
    [<ffffffff817da069>] ? error_exit+0x29/0xb0
    [<ffffffff812bcc60>] ? kobject_del+0x40/0x40
    [<ffffffff8145ed16>] scsi_scan_channel+0x86/0xb0
    [<ffffffff8145f0b0>] scsi_scan_host_selected+0x140/0x150
    [<ffffffff8145f149>] do_scsi_scan_host+0x89/0x90
    [<ffffffff8145f170>] do_scan_async+0x20/0x160
    [<ffffffff8145f150>] ? do_scsi_scan_host+0x90/0x90
    [<ffffffff810975b6>] kthread+0xa6/0xb0
    [<ffffffff817db154>] kernel_thread_helper+0x4/0x10
    [<ffffffff81066430>] ? finish_task_switch+0x80/0x110
    [<ffffffff817d9c04>] ? retint_restore_args+0xe/0xe
    [<ffffffff81097510>] ? __kthread_init_worker+0x70/0x70
    [<ffffffff817db150>] ? gs_change+0xb/0xb
\end{VerbatimU}

Line 2776 of \path{block/cfq-iosched.c} in v3.0-rc5 is as follows:

\begin{VerbatimU}
	if (rcu_dereference(ioc->ioc_data) == cic) {
\end{VerbatimU}

This form says that it must be in a plain vanilla RCU read-side critical
section, but the ``other info'' list above shows that this is not the
case.
Instead, we hold three locks, one of which might be RCU related.
And maybe that lock really does protect this reference.
If so, the fix
is to inform RCU, perhaps by changing \co{__cfq_exit_single_io_context()} to
take the struct \co{request_queue} \qco{q} from \co{cfq_exit_queue()} as an argument,
which would permit us to invoke \co{rcu_dereference_protected} as follows:

\begin{VerbatimU}
	if (rcu_dereference_protected(ioc->ioc_data,
	                              lockdep_is_held(&q->queue_lock)) == cic) {
\end{VerbatimU}

With this change, there would be no lockdep-RCU splat emitted if this
code was invoked either from within an RCU read-side critical section
or with the \co{->queue_lock} held.
In particular, this would have suppressed
the above lockdep-RCU splat because \co{->queue_lock} is held (see \co{#2} in the
list above).

On the other hand, perhaps we really do need an RCU read-side critical
section.
In this case, the critical section must span the use of the
return value from \co{rcu_dereference()}, or at least until there is some
reference count incremented or some such.
One way to handle this is to
add \co{rcu_read_lock()} and \co{rcu_read_unlock()} as follows:

\begin{VerbatimU}
	rcu_read_lock();
	if (rcu_dereference(ioc->ioc_data) == cic) {
		spin_lock(&ioc->lock);
		rcu_assign_pointer(ioc->ioc_data, NULL);
		spin_unlock(&ioc->lock);
	}
	rcu_read_unlock();
\end{VerbatimU}

With this change, the \co{rcu_dereference()} is always within an RCU
read-side critical section, which again would have suppressed the
above lockdep-RCU splat.

But in this particular case, we don't actually dereference the pointer
returned from \co{rcu_dereference()}.
Instead, that pointer is just compared
to the cic pointer, which means that the \co{rcu_dereference()} can be replaced
by \co{rcu_access_pointer()} as follows:

\begin{VerbatimU}
	if (rcu_access_pointer(ioc->ioc_data) == cic) {
\end{VerbatimU}

Because it is legal to invoke \co{rcu_access_pointer()} without protection,
this change would also suppress the above lockdep-RCU splat.
