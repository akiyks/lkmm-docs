% SPDX-License-Identifier: GPL-2.0

\section{RCU and lockdep checking}
\label{sec:rcu:RCU and lockdep checking}

All flavors of RCU have lockdep checking available, so that lockdep is
aware of when each task enters and leaves any flavor of RCU read-side
critical section.
Each flavor of RCU is tracked separately (but note
that this is not the case in 2.6.32 and earlier).
This allows lockdep's
tracking to include RCU state, which can sometimes help when debugging
deadlocks and the like.

In addition, RCU provides the following primitives that check lockdep's
state:

\begin{itemize}
\item	\co{rcu_read_lock_held()} for normal RCU.
\item	\co{rcu_read_lock_bh_held()} for RCU-bh.
\item	\co{rcu_read_lock_sched_held()} for RCU-sched.
\item	\co{rcu_read_lock_any_held()} for any of normal RCU, RCU-bh, and RCU-sched.
\item	\co{srcu_read_lock_held()} for SRCU.
\item	\co{rcu_read_lock_trace_held()} for RCU Tasks Trace.
\end{itemize}

These functions are conservative, and will therefore return 1 if they
aren't certain (for example, if \co{CONFIG_DEBUG_LOCK_ALLOC} is not set).
This prevents things like \co{WARN_ON(!rcu_read_lock_held())} from giving false
positives when lockdep is disabled.

In addition, a separate kernel config parameter \co{CONFIG_PROVE_RCU} enables
checking of \co{rcu_dereference()} primitives:

\begin{description}[style=nextline]
\item[\tco{rcu_dereference(p)}]
		Check for RCU read-side critical section.
\item[\tco{rcu_dereference_bh(p)}]
		Check for RCU-bh read-side critical section.
\item[\tco{rcu_dereference_sched(p)}]
		Check for RCU-sched read-side critical section.
\item[\tco{srcu_dereference(p, sp)}]
		Check for SRCU read-side critical section.
\item[\tco{rcu_dereference_check(p, c)}]
		Use explicit check expression \qtco{c} along with
		\co{rcu_read_lock_held()}.
		This is useful in code that is
		invoked by both RCU readers and updaters.
\item[\tco{rcu_dereference_bh_check(p, c)}]
		Use explicit check expression \qtco{c} along with
		\co{rcu_read_lock_bh_held()}.
		This is useful in code that
		is invoked by both RCU-bh readers and updaters.
\item[\tco{rcu_dereference_sched_check(p, c)}]
		Use explicit check expression \qtco{c} along with
		\co{rcu_read_lock_sched_held()}.
		This is useful in code that
		is invoked by both RCU-sched readers and updaters.
\item[\tco{srcu_dereference_check(p, c)}]
		Use explicit check expression \qtco{c} along with
		\co{srcu_read_lock_held()}.
		This is useful in code that
		is invoked by both SRCU readers and updaters.
\item[\tco{rcu_dereference_raw(p)}]
		Don't check.
		(Use sparingly, if at all.)
\item[\tco{rcu_dereference_raw_check(p)}]
		Don't do lockdep at all.
		(Use sparingly, if at all.)
\item[\tco{rcu_dereference_protected(p, c)}]
		Use explicit check expression \qtco{c}, and omit all barriers
		and compiler constraints.
		This is useful when the data
		structure cannot change, for example, in code that is
		invoked only by updaters.
\item[\tco{rcu_access_pointer(p)}]
		Return the value of the pointer and omit all barriers,
		but retain the compiler constraints that prevent duplicating
		or coalescing.
		This is useful when testing the
		value of the pointer itself, for example, against NULL.
\end{description}

The \co{rcu_dereference_check()} check expression can be any boolean
expression, but would normally include a lockdep expression.
For a
moderately ornate example, consider the following:

\begin{VerbatimU}
	file = rcu_dereference_check(fdt->fd[fd],
	                             lockdep_is_held(&files->file_lock) ||
	                             atomic_read(&files->count) == 1);
\end{VerbatimU}

This expression picks up the pointer \qtco{fdt->fd[fd]} in an RCU-safe manner,
and, if \co{CONFIG_PROVE_RCU} is configured, verifies that this expression
is used in:

\begin{enumerate}[(1)]
\item	An RCU read-side critical section (implicit), or
\item	with \co{files->file_lock} held, or
\item	on an unshared \co{files_struct}.
\end{enumerate}

In case (1), the pointer is picked up in an RCU-safe manner for vanilla
RCU read-side critical sections, in case (2), the \co{->file_lock} prevents
any change from taking place, and finally in case (3), the current task
is the only task accessing the \co{file_struct}, again preventing any change
from taking place.
If the above statement was invoked only from updater
code, it could instead be written as follows:

\begin{VerbatimU}
	file = rcu_dereference_protected(fdt->fd[fd],
	                                 lockdep_is_held(&files->file_lock) ||
	                                 atomic_read(&files->count) == 1);
\end{VerbatimU}

This would verify cases (2) and (3) above, and furthermore lockdep would
complain even if this was used in an RCU read-side critical section unless
one of these two cases held.
Because \co{rcu_dereference_protected()} omits
all barriers and compiler constraints, it generates better code than do
the other flavors of \co{rcu_dereference()}.
On the other hand, it is illegal
to use \co{rcu_dereference_protected()} if either the RCU-protected pointer
or the RCU-protected data that it points to can change concurrently.

Like \co{rcu_dereference()}, when lockdep is enabled, RCU list and hlist
traversal primitives check for being called from within an RCU read-side
critical section.
However, a lockdep expression can be passed to them
as a additional optional argument.
With this lockdep expression, these
traversal primitives will complain only if the lockdep expression is
false and they are called from outside any RCU read-side critical section.

For example, the workqueue \co{for_each_pwq()} macro is intended to be used
either within an RCU read-side critical section or with \co{wq->mutex} held.
It is thus implemented as follows:

\begin{VerbatimU}
	#define for_each_pwq(pwq, wq)
	        list_for_each_entry_rcu((pwq), &(wq)->pwqs, pwqs_node,
	                                lock_is_held(&(wq->mutex).dep_map))
\end{VerbatimU}
