% .. SPDX-License-Identifier: GPL-2.0

\section{Using RCU's CPU Stall Detector}
\label{sec:rcu:Using RCU's CPU Stall Detector}

This document first discusses what sorts of issues RCU's CPU stall
detector can locate, and then discusses kernel parameters and \co{Kconfig}
options that can be used to fine-tune the detector's operation.
Finally,
this document explains the stall detector's ``splat'' format.


\subsection{What Causes RCU CPU Stall Warnings?}

So your kernel printed an RCU CPU stall warning.
The next question is
``What caused it?''
The following problems can result in RCU CPU stall
warnings:

\begin{itemize}
\item	A CPU looping in an RCU read-side critical section.

\item	A CPU looping with interrupts disabled.

\item	A CPU looping with preemption disabled.

\item	A CPU looping with bottom halves disabled.

\item	For \co{!CONFIG_PREEMPTION} kernels, a CPU looping anywhere in the
	kernel without potentially invoking \co{schedule()}.
	If the looping
	in the kernel is really expected and desirable behavior, you
	might need to add some calls to \co{cond_resched()}.

\item	Booting Linux using a console connection that is too slow to
	keep up with the boot-time console-message rate.
	For example,
	a 115Kbaud serial console can be \emph{way} too slow to keep up
	with boot-time message rates, and will frequently result in
	RCU CPU stall warning messages.
	Especially if you have added
	debug \co{printk()}s.

\item	Anything that prevents RCU's grace-period kthreads from running.
	This can result in the \qco{All QSes seen} console-log message.
	This message will include information on when the kthread last
	ran and how often it should be expected to run.
        It can also
	result in the \qco{rcu_.*kthread starved for} console-log message,
	which will include additional debugging information.

\item	A CPU-bound real-time task in a \co{CONFIG_PREEMPTION} kernel, which might
	happen to preempt a low-priority task in the middle of an RCU
	read-side critical section.
	This is especially damaging if
	that low-priority task is not permitted to run on any other CPU,
	in which case the next RCU grace period can never complete, which
	will eventually cause the system to run out of memory and hang.
	While the system is in the process of running itself out of
	memory, you might see stall-warning messages.

\item	A CPU-bound real-time task in a \co{CONFIG_PREEMPT_RT} kernel that
	is running at a higher priority than the RCU softirq threads.
	This will prevent RCU callbacks from ever being invoked,
	and in a \co{CONFIG_PREEMPT_RCU} kernel will further prevent
	RCU grace periods from ever completing.
	Either way, the
	system will eventually run out of memory and hang.
	In the
	\co{CONFIG_PREEMPT_RCU} case, you might see stall-warning
	messages.

	You can use the \co{rcutree.kthread_prio} kernel boot parameter to
	increase the scheduling priority of RCU's kthreads, which can
	help avoid this problem.
	However, please note that doing this
	can increase your system's context-switch rate and thus degrade
	performance.

\item	A periodic interrupt whose handler takes longer than the time
	interval between successive pairs of interrupts.
	This can
	prevent RCU's kthreads and softirq handlers from running.
	Note that certain high-overhead debugging options, for example
	the \co{function_graph} tracer, can result in interrupt handler taking
	considerably longer than normal, which can in turn result in
	RCU CPU stall warnings.

\item	Testing a workload on a fast system, tuning the stall-warning
	timeout down to just barely avoid RCU CPU stall warnings, and then
	running the same workload with the same stall-warning timeout on a
	slow system.
	Note that thermal throttling and on-demand governors
	can cause a single system to be sometimes fast and sometimes slow!

\item	A hardware or software issue shuts off the scheduler-clock
	interrupt on a CPU that is not in dyntick-idle mode.
	This
	problem really has happened, and seems to be most likely to
	result in RCU CPU stall warnings for \co{CONFIG_NO_HZ_COMMON=n} kernels.

\item	A hardware or software issue that prevents time-based wakeups
	from occurring.
	These issues can range from misconfigured or
	buggy timer hardware through bugs in the interrupt or exception
	path (whether hardware, firmware, or software) through bugs
	in Linux's timer subsystem through bugs in the scheduler, and,
	yes, even including bugs in RCU itself.
	It can also result in
	the \qco{rcu_.*timer wakeup didn't happen for} console-log message,
	which will include additional debugging information.

\item	A low-level kernel issue that either fails to invoke one of the
	variants of \co{rcu_eqs_enter(true)}, \co{rcu_eqs_exit(true)}, \co{ct_idle_enter()},
	\co{ct_idle_exit()}, \co{ct_irq_enter()}, or \co{ct_irq_exit()} on the one
	hand, or that invokes one of them too many times on the other.
	Historically, the most frequent issue has been an omission
	of either \co{irq_enter()} or \co{irq_exit()}, which in turn invoke
	\co{ct_irq_enter()} or \co{ct_irq_exit()}, respectively
	Building your
	kernel with \co{CONFIG_RCU_EQS_DEBUG=y} can help track down these types
	of issues, which sometimes arise in architecture-specific code.

\item	A bug in the RCU implementation.

\item	A hardware failure.
	This is quite unlikely, but is not at all
	uncommon in large datacenter.
	In one memorable case some decades
	back, a CPU failed in a running system, becoming unresponsive,
	but not causing an immediate crash.
	This resulted in a series
	of RCU CPU stall warnings, eventually leading the realization
	that the CPU had failed.
\end{itemize}

The RCU, RCU-sched, RCU-tasks, and RCU-tasks-trace implementations have
CPU stall warning.
Note that SRCU does \emph{not} have CPU stall warnings.
Please note that RCU only detects CPU stalls when there is a grace period
in progress.
No grace period, no CPU stall warnings.

To diagnose the cause of the stall, inspect the stack traces.
The offending function will usually be near the top of the stack.
If you have a series of stall warnings from a single extended stall,
comparing the stack traces can often help determine where the stall
is occurring, which will usually be in the function nearest the top of
that portion of the stack which remains the same from trace to trace.
If you can reliably trigger the stall, \co{ftrace} can be quite helpful.

RCU bugs can often be debugged with the help of \co{CONFIG_RCU_TRACE}
and with RCU's event tracing.
For information on RCU's event tracing,
see \path{include/trace/events/rcu.h}.


\subsection{Fine-Tuning the RCU CPU Stall Detector}

The \co{rcuupdate.rcu_cpu_stall_suppress} module parameter disables RCU's
CPU stall detector, which detects conditions that unduly delay RCU grace
periods.
This module parameter enables CPU stall detection by default,
but may be overridden via boot-time parameter or at runtime via \co{sysfs}.
The stall detector's idea of what constitutes ``unduly delayed'' is
controlled by a set of kernel configuration variables and cpp macros:

\begin{description}[style=nextline]
\item[\tco{CONFIG_RCU_CPU_STALL_TIMEOUT}]
	This kernel configuration parameter defines the period of time
	that RCU will wait from the beginning of a grace period until it
	issues an RCU CPU stall warning.
	This time period is normally
	21 seconds.

	This configuration parameter may be changed at runtime via the
	\path{/sys/module/rcupdate/parameters/rcu_cpu_stall_timeout}, however
	this parameter is checked only at the beginning of a cycle.
	So if you are 10~seconds into a 40-second stall, setting this
	\co{sysfs} parameter to (say) five will shorten the timeout for the
	\emph{next} stall, or the following warning for the current stall
	(assuming the stall lasts long enough).
	It will not affect the
	timing of the next warning for the current stall.

	Stall-warning messages may be enabled and disabled completely via
	\path{/sys/module/rcupdate/parameters/rcu_cpu_stall_suppress}.

\item[\tco{CONFIG_RCU_EXP_CPU_STALL_TIMEOUT}]
	Same as the \co{CONFIG_RCU_CPU_STALL_TIMEOUT} parameter but only for
	the expedited grace period.
        This parameter defines the period
	of time that RCU will wait from the beginning of an expedited
	grace period until it issues an RCU CPU stall warning.
	This time
	period is normally 20~milliseconds on Android devices.
	A zero
	value causes the \co{CONFIG_RCU_CPU_STALL_TIMEOUT} value to be used,
	after conversion to milliseconds.

	This configuration parameter may be changed at runtime via the
	\path{/sys/module/rcupdate/parameters/rcu_exp_cpu_stall_timeout}, however
	this parameter is checked only at the beginning of a cycle.
	If you
	are in a current stall cycle, setting it to a new value will change
	the timeout for the \emph{next} stall.

	Stall-warning messages may be enabled and disabled completely via
	\path{/sys/module/rcupdate/parameters/rcu_cpu_stall_suppress}.

\item[\tco{RCU_STALL_DELAY_DELTA}]
	Although the \co{lockdep} facility is extremely useful, it does add
	some overhead.
	Therefore, under \co{CONFIG_PROVE_RCU}, the
	\co{RCU_STALL_DELAY_DELTA} macro allows five extra seconds before
	giving an RCU CPU stall warning message.
	(This is a cpp
	macro, not a kernel configuration parameter.)

\item[\tco{RCU_STALL_RAT_DELAY}]
	The CPU stall detector tries to make the offending CPU print its
	own warnings, as this often gives better-quality stack traces.
	However, if the offending CPU does not detect its own stall in
	the number of jiffies specified by \co{RCU_STALL_RAT_DELAY}, then
	some other CPU will complain.
	This delay is normally set to
	two jiffies.
	(This is a cpp macro, not a kernel configuration
	parameter.)

\item[\tco{rcupdate.rcu_task_stall_timeout}]
	This boot/\co{sysfs} parameter controls the RCU-tasks and
	RCU-tasks-trace stall warning intervals.
	A value of zero or less
	suppresses RCU-tasks stall warnings.
	A positive value sets the
	stall-warning interval in seconds.
	An RCU-tasks stall warning
	starts with the line:

\begin{VerbatimU}[gobble=2]
		INFO: rcu_tasks detected stalls on tasks:
\end{VerbatimU}

	And continues with the output of \co{sched_show_task()} for each
	task stalling the current RCU-tasks grace period.

	An RCU-tasks-trace stall warning starts (and continues) similarly:

\begin{VerbatimU}[gobble=2]
		INFO: rcu_tasks_trace detected stalls on tasks
\end{VerbatimU}
\end{description}

\subsection{Interpreting RCU's CPU Stall-Detector ``Splats''}

For non-RCU-tasks flavors of RCU, when a CPU detects that some other
CPU is stalling, it will print a message similar to the following:

\begin{VerbatimU}
	INFO: rcu_sched detected stalls on CPUs/tasks:
	2-...: (3 GPs behind) idle=06c/0/0 softirq=1453/1455 fqs=0
	16-...: (0 ticks this GP) idle=81c/0/0 softirq=764/764 fqs=0
	(detected by 32, t=2603 jiffies, g=7075, q=625)
\end{VerbatimU}

This message indicates that CPU~32 detected that CPUs~2 and~16 were both
causing stalls, and that the stall was affecting RCU-sched.
This message
will normally be followed by stack dumps for each CPU\@.
Please note that
\co{PREEMPT_RCU} builds can be stalled by tasks as well as by CPUs, and that
the tasks will be indicated by PID, for example, \qco{P3421}.
It is even
possible for an \co{rcu_state} stall to be caused by both CPUs \emph{and} tasks,
in which case the offending CPUs and tasks will all be called out in the list.
In some cases, CPUs will detect themselves stalling, which will result
in a self-detected stall.

CPU~2's \qco{(3 GPs behind)} indicates that this CPU has not interacted with
the RCU core for the past three grace periods.
In contrast, CPU~16's \qco{(0 ticks this GP)}
indicates that this CPU has not taken any scheduling-clock
interrupts during the current stalled grace period.

The \qco{idle=} portion of the message prints the dyntick-idle state.
The hex number before the first \qco{/} is the low-order 16~bits of the
dynticks counter, which will have an even-numbered value if the CPU
is in dyntick-idle mode and an odd-numbered value otherwise.
The hex
number between the two \qco{/}s is the value of the nesting, which will be
a small non-negative number if in the idle loop (as shown above) and a
very large positive number otherwise.
The number following the final
\qco{/} is the NMI nesting, which will be a small non-negative number.

The \qco{softirq=} portion of the message tracks the number of RCU softirq
handlers that the stalled CPU has executed.
The number before the \qco{/}
is the number that had executed since boot at the time that this CPU
last noted the beginning of a grace period, which might be the current
(stalled) grace period, or it might be some earlier grace period (for
example, if the CPU might have been in dyntick-idle mode for an extended
time period).
The number after the \qco{/} is the number that have executed
since boot until the current time.
If this latter number stays constant
across repeated stall-warning messages, it is possible that RCU's softirq
handlers are no longer able to execute on this CPU\@.
This can happen if
the stalled CPU is spinning with interrupts are disabled, or, in \co{-rt}
kernels, if a high-priority process is starving RCU's softirq handler.

The \qco{fqs=} shows the number of force-quiescent-state idle/offline
detection passes that the grace-period kthread has made across this
CPU since the last time that this CPU noted the beginning of a grace
period.

The \qco{detected by} line indicates which CPU detected the stall (in this
case, CPU~32), how many jiffies have elapsed since the start of the grace
period (in this case 2603), the grace-period sequence number (7075), and
an estimate of the total number of RCU callbacks queued across all CPUs
(625 in this case).

If the grace period ends just as the stall warning starts printing,
there will be a spurious stall-warning message, which will include
the following:

\begin{VerbatimU}
	INFO: Stall ended before state dump start
\end{VerbatimU}

This is rare, but does happen from time to time in real life.
It is also
possible for a zero-jiffy stall to be flagged in this case, depending
on how the stall warning and the grace-period initialization happen to
interact.
Please note that it is not possible to entirely eliminate this
sort of false positive without resorting to things like \co{stop_machine()},
which is overkill for this sort of problem.

If all CPUs and tasks have passed through quiescent states, but the
grace period has nevertheless failed to end, the stall-warning splat
will include something like the following:

\begin{VerbatimU}[breaklines=true]
	All QSes seen, last rcu_preempt kthread activity 23807 (4297905177-4297881370), jiffies_till_next_fqs=3, root ->qsmask 0x0
\end{VerbatimU}

The \qco{23807} indicates that it has been more than 23~thousand jiffies
since the grace-period kthread ran.
The \qco{jiffies_till_next_fqs}
indicates how frequently that kthread should run, giving the number
of jiffies between force-quiescent-state scans, in this case three,
which is way less than 23807.
Finally, the root \co{rcu_node} structure's
\co{->qsmask} field is printed, which will normally be zero.

If the relevant grace-period kthread has been unable to run prior to
the stall warning, as was the case in the \qco{All QSes seen} line above,
the following additional line is printed:

\begin{VerbatimU}[breaklines=true]
	rcu_sched kthread starved for 23807 jiffies! g7075 f0x0 RCU_GP_WAIT_FQS(3) ->state=0x1 ->cpu=5
	Unless rcu_sched kthread gets sufficient CPU time, OOM is now expected behavior.
\end{VerbatimU}


Starving the grace-period kthreads of CPU time can of course result
in RCU CPU stall warnings even when all CPUs and tasks have passed
through the required quiescent states.
The \qco{g} number shows the current
grace-period sequence number, the \qco{f} precedes the \co{->gp_flags} command
to the grace-period kthread, the \qco{RCU_GP_WAIT_FQS} indicates that the
kthread is waiting for a short timeout, the \qco{state} precedes value of the
\co{task_struct} \co{->state field}, and the \qco{cpu} indicates that the grace-period
kthread last ran on CPU~5.

If the relevant grace-period kthread does not wake from FQS wait in a
reasonable time, then the following additional line is printed:

\begin{VerbatimU}[breaklines=true]
	kthread timer wakeup didn't happen for 23804 jiffies! g7076 f0x0 RCU_GP_WAIT_FQS(5) ->state=0x402
\end{VerbatimU}

The \qco{23804} indicates that kthread's timer expired more than 23~thousand
jiffies ago.
The rest of the line has meaning similar to the kthread
starvation case.

Additionally, the following line is printed::

\begin{VerbatimU}
	Possible timer handling issue on cpu=4 timer-softirq=11142
\end{VerbatimU}

Here \qco{cpu} indicates that the grace-period kthread last ran on CPU~4,
where it queued the fqs timer.
The number following the \qco{timer-softirq}
is the current \qco{TIMER_SOFTIRQ} count on cpu~4.
If this value does not
change on successive RCU CPU stall warnings, there is further reason to
suspect a timer problem.

These messages are usually followed by stack dumps of the CPUs and tasks
involved in the stall.
These stack traces can help you locate the cause
of the stall, keeping in mind that the CPU detecting the stall will have
an interrupt frame that is mainly devoted to detecting the stall.


\subsection{Multiple Warnings From One Stall}

If a stall lasts long enough, multiple stall-warning messages will
be printed for it.
The second and subsequent messages are printed at
longer intervals, so that the time between (say) the first and second
message will be about three times the interval between the beginning
of the stall and the first message.
It can be helpful to compare the
stack dumps for the different messages for the same stalled grace period.


\subsection{Stall Warnings for Expedited Grace Periods}

If an expedited grace period detects a stall, it will place a message
like the following in dmesg:

\begin{VerbatimU}[breaklines=true]
	INFO: rcu_sched detected expedited stalls on CPUs/tasks: { 7-... } 21119 jiffies s: 73 root: 0x2/.
\end{VerbatimU}

This indicates that CPU~7 has failed to respond to a reschedule IPI\@.
The three periods (\qco{...}) following the CPU number indicate that the CPU
is online (otherwise the first period would instead have been \qco{O}),
that the CPU was online at the beginning of the expedited grace period
(otherwise the second period would have instead been \qco{o}), and that
the CPU has been online at least once since boot (otherwise, the third
period would instead have been \qco{N}).
The number before the "jiffies"
indicates that the expedited grace period has been going on for 21,119
jiffies.
The number following the \qco{s:} indicates that the expedited
grace-period sequence counter is~73.
The fact that this last value is
odd indicates that an expedited grace period is in flight.
The number
following \qco{root:} is a bitmask that indicates which children of the root
\co{rcu_node} structure correspond to CPUs and/or tasks that are blocking the
current expedited grace period.
If the tree had more than one level,
additional hex numbers would be printed for the states of the other
\co{rcu_node} structures in the tree.

As with normal grace periods, \co{PREEMPT_RCU} builds can be stalled by
tasks as well as by CPUs, and that the tasks will be indicated by PID,
for example, \qco{P3421}.

It is entirely possible to see stall warnings from normal and from
expedited grace periods at about the same time during the same run.

\subsection{\texttt{RCU\_CPU\_STALL\_CPUTIME}}

In kernels built with \co{CONFIG_RCU_CPU_STALL_CPUTIME=y} or booted with
\co{rcupdate.rcu_cpu_stall_cputime=1}, the following additional information
is supplied with each RCU CPU stall warning:

\begin{VerbatimU}
  rcu:          hardirqs   softirqs   csw/system
  rcu:  number:      624         45            0
  rcu: cputime:       69          1         2425   ==> 2500(ms)
\end{VerbatimU}

These statistics are collected during the sampling period.
The values
in row \qco{number:} are the number of hard interrupts, number of soft
interrupts, and number of context switches on the stalled CPU\@.
The
first three values in row \qco{cputime:}" indicate the CPU time in
milliseconds consumed by hard interrupts, soft interrupts, and tasks
on the stalled CPU\@.
The last number is the measurement interval, again
in milliseconds.
Because user-mode tasks normally do not cause RCU CPU
stalls, these tasks are typically kernel tasks, which is why only the
system CPU time are considered.

The sampling period is shown as follows:

\begin{VerbatimU}
  |<------------first timeout---------->|<-----second timeout----->|
  |<--half timeout-->|<--half timeout-->|                          |
  |                  |<--first period-->|                          |
  |                  |<-----------second sampling period---------->|
  |                  |                  |                          |
             snapshot time point    1st-stall                  2nd-stall
\end{VerbatimU}

The following describes four typical scenarios:

\begin{enumerate}
\item A CPU looping with interrupts disabled.

\begin{VerbatimU}
     rcu:          hardirqs   softirqs   csw/system
     rcu:  number:        0          0            0
     rcu: cputime:        0          0            0   ==> 2500(ms)
\end{VerbatimU}

   Because interrupts have been disabled throughout the measurement
   interval, there are no interrupts and no context switches.
   Furthermore, because CPU time consumption was measured using interrupt
   handlers, the system CPU consumption is misleadingly measured as zero.
   This scenario will normally also have \qco{(0 ticks this GP)} printed on
   this CPU's summary line.

\item A CPU looping with bottom halves disabled.

   This is similar to the previous example, but with non-zero number of
   and CPU time consumed by hard interrupts, along with non-zero CPU
   time consumed by in-kernel execution:

\begin{VerbatimU}
     rcu:          hardirqs   softirqs   csw/system
     rcu:  number:      624          0            0
     rcu: cputime:       49          0         2446   ==> 2500(ms)
\end{VerbatimU}

   The fact that there are zero softirqs gives a hint that these were
   disabled, perhaps via \co{local_bh_disable()}.
   It is of course possible
   that there were no softirqs, perhaps because all events that would
   result in softirq execution are confined to other CPUs.
   In this case,
   the diagnosis should continue as shown in the next example.

\item A CPU looping with preemption disabled.

   Here, only the number of context switches is zero:

\begin{VerbatimU}
     rcu:          hardirqs   softirqs   csw/system
     rcu:  number:      624         45            0
     rcu: cputime:       69          1         2425   ==> 2500(ms)
\end{VerbatimU}

   This situation hints that the stalled CPU was looping with preemption
   disabled.

\item No looping, but massive hard and soft interrupts.

\begin{VerbatimU}
     rcu:          hardirqs   softirqs   csw/system
     rcu:  number:       xx         xx            0
     rcu: cputime:       xx         xx            0   ==> 2500(ms)
\end{VerbatimU}

   Here, the number and CPU time of hard interrupts are all non-zero,
   but the number of context switches and the in-kernel CPU time consumed
   are zero.
   The number and cputime of soft interrupts will usually be
   non-zero, but could be zero, for example, if the CPU was spinning
   within a single hard interrupt handler.

   If this type of RCU CPU stall warning can be reproduced, you can
   narrow it down by looking at \path{/proc/interrupts} or by writing code to
   trace each interrupt, for example, by referring to \co{show_interrupts()}.
\end{enumerate}
