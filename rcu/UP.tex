\section{RCU on Uniprocessor Systems}
\label{sec:rcu:RCU on Uniprocessor Systems}

A common misconception is that, on UP systems, the \co{call_rcu()} primitive
may immediately invoke its function.
The basis of this misconception
is that since there is only one CPU, it should not be necessary to
wait for anything else to get done, since there are no other CPUs for
anything else to be happening on.
Although this approach will \emph{sort of}
work a surprising amount of the time, it is a very bad idea in general.
This document presents three examples that demonstrate exactly how bad
an idea this is.

\subsection{Example 1: softirq Suicide}

Suppose that an RCU-based algorithm scans a linked list containing
elements~A, B, and~C in process context, and can delete elements from
this same list in softirq context.
Suppose that the process-context scan
is referencing element~B when it is interrupted by softirq processing,
which deletes element~B, and then invokes \co{call_rcu()} to free element~B
after a grace period.

Now, if \co{call_rcu()} were to directly invoke its arguments, then upon return
from softirq, the list scan would find itself referencing a newly freed
element~B.
This situation can greatly decrease the life expectancy of
your kernel.

This same problem can occur if \co{call_rcu()} is invoked from a hardware
interrupt handler.

\subsection{Example 2: Function-Call Fatality}

Of course, one could avert the suicide described in the preceding example
by having \co{call_rcu()} directly invoke its arguments only if it was called
from process context.
However, this can fail in a similar manner.

Suppose that an RCU-based algorithm again scans a linked list containing
elements~A, B, and~C in process context, but that it invokes a function
on each element as it is scanned.
Suppose further that this function
deletes element~B from the list, then passes it to \co{call_rcu()} for deferred
freeing.
This may be a bit unconventional, but it is perfectly legal
RCU usage, since \co{call_rcu()} must wait for a grace period to elapse.
Therefore, in this case, allowing \co{call_rcu()} to immediately invoke
its arguments would cause it to fail to make the fundamental guarantee
underlying RCU, namely that \co{call_rcu()} defers invoking its arguments until
all RCU read-side critical sections currently executing have completed.

\QuickQuiz{
	Why is it \emph{not} legal to invoke \co{synchronize_rcu()} in this case?
}\QuickQuizAnswer{
	Because the calling function is scanning an RCU-protected linked
	list, and is therefore within an RCU read-side critical section.
	Therefore, the called function has been invoked within an RCU
	read-side critical section, and is not permitted to block.
}\QuickQuizEnd

\subsection{Example 3: Death by Deadlock}

Suppose that \co{call_rcu()} is invoked while holding a lock, and that the
callback function must acquire this same lock.
In this case, if
\co{call_rcu()} were to directly invoke the callback, the result would
be self-deadlock \emph{even if} this invocation occurred from a later
\co{call_rcu()} invocation a full grace period later.

In some cases, it would possible to restructure to code so that
the \co{call_rcu()} is delayed until after the lock is released.
However,
there are cases where this can be quite ugly:

\begin{enumerate}
\item	If a number of items need to be passed to \co{call_rcu()} within
	the same critical section, then the code would need to create
	a list of them, then traverse the list once the lock was
	released.

\item	In some cases, the lock will be held across some kernel API,
	so that delaying the \co{call_rcu()} until the lock is released
	requires that the data item be passed up via a common API\@.
	It is far better to guarantee that callbacks are invoked
	with no locks held than to have to modify such APIs to allow
	arbitrary data items to be passed back up through them.
\end{enumerate}

If \co{call_rcu()} directly invokes the callback, painful locking restrictions
or API changes would be required.

\QuickQuiz{
	What locking restriction must RCU callbacks respect?
}\QuickQuizAnswer{
	Any lock that is acquired within an RCU callback must be acquired
	elsewhere using an \co{_bh} variant of the spinlock primitive.
	For example, if \qco{mylock} is acquired by an RCU callback, then
	a process-context acquisition of this lock must use something
	like \co{spin_lock_bh()} to acquire the lock.
	Please note that
	it is also OK to use \co{_irq} variants of spinlocks, for example,
	\co{spin_lock_irqsave()}.

	If the process-context code were to simply use \co{spin_lock()},
	then, since RCU callbacks can be invoked from softirq context,
	the callback might be called from a softirq that interrupted
	the process-context critical section.
	This would result in
	self-deadlock.

	This restriction might seem gratuitous, since very few RCU
	callbacks acquire locks directly.
	However, a great many RCU
	callbacks do acquire locks \emph{indirectly}, for example, via
	the \co{kfree()} primitive.
}\QuickQuizEnd

It is important to note that userspace RCU implementations \emph{do}
permit \co{call_rcu()} to directly invoke callbacks, but only if a full
grace period has elapsed since those callbacks were queued.
This is
the case because some userspace environments are extremely constrained.
Nevertheless, people writing userspace RCU implementations are strongly
encouraged to avoid invoking callbacks from \co{call_rcu()}, thus obtaining
the deadlock-avoidance benefits called out above.

\subsection{Summary}

Permitting \co{call_rcu()} to immediately invoke its arguments breaks RCU,
even on a UP system.
So do not do it!
Even on a UP system, the RCU
infrastructure \emph{must} respect grace periods, and \emph{must} invoke callbacks
from a known environment in which no locks are held.

Note that it \emph{is} safe for \co{synchronize_rcu()} to return immediately on
UP systems, including \co{PREEMPT SMP} builds running on UP systems.

\QuickQuiz{
	Why can't \co{synchronize_rcu()} return immediately on UP systems running
	preemptible RCU\@?
}\QuickQuizAnswer{
	Because some other task might have been preempted in the middle
	of an RCU read-side critical section.
	If \co{synchronize_rcu()}
	simply immediately returned, it would prematurely signal the
	end of the grace period, which would come as a nasty shock to
	that other thread when it started running again.
}\QuickQuizEnd
