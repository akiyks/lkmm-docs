\section{RCU and Unloadable Modules}
\label{sec:rcu:RCU and Unloadable Modules}

[Originally published in LWN Jan. 14, 2007: \url{http://lwn.net/Articles/217484/}]

RCU updaters sometimes use \co{call_rcu()} to initiate an asynchronous wait for
a grace period to elapse.
This primitive takes a pointer to an \co{rcu_head}
struct placed within the RCU-protected data structure and another pointer
to a function that may be invoked later to free that structure.
Code to
delete an element~\co{p} from the linked list from IRQ context might then be
as follows:

\begin{VerbatimU}
	list_del_rcu(p);
	call_rcu(&p->rcu, p_callback);
\end{VerbatimU}

Since \co{call_rcu()} never blocks, this code can safely be used from within
IRQ context.
The function \co{p_callback()} might be defined as follows:

\begin{VerbatimU}
	static void p_callback(struct rcu_head *rp)
	{
		struct pstruct *p = container_of(rp, struct pstruct, rcu);

		kfree(p);
	}
\end{VerbatimU}


\subsection{Unloading Modules That Use \co{call_rcu()}}
\label{sec:rcu:Unloading Modules That Use call_rcu()}

But what if the \co{p_callback()} function is defined in an unloadable module?

If we unload the module while some RCU callbacks are pending,
the CPUs executing these callbacks are going to be severely
disappointed when they are later invoked, as fancifully depicted at
\url{http://lwn.net/images/ns/kernel/rcu-drop.jpg}.

We could try placing a \co{synchronize_rcu()} in the module-exit code path,
but this is not sufficient.
Although \co{synchronize_rcu()} does wait for a
grace period to elapse, it does not wait for the callbacks to complete.

One might be tempted to try several back-to-back \co{synchronize_rcu()}
calls, but this is still not guaranteed to work.
If there is a very
heavy RCU-callback load, then some of the callbacks might be deferred in
order to allow other processing to proceed.
For but one example, such
deferral is required in realtime kernels in order to avoid excessive
scheduling latencies.


\subsection{\co{rcu_barrier()}}
\label{sec:rcu:rcu_barrier()}

This situation can be handled by the \co{rcu_barrier()} primitive.
Rather
than waiting for a grace period to elapse, \co{rcu_barrier()} waits for all
outstanding RCU callbacks to complete.
Please note that \co{rcu_barrier()}
does \emph{not} imply \co{synchronize_rcu()}, in particular, if there are no RCU
callbacks queued anywhere, \co{rcu_barrier()} is within its rights to return
immediately, without waiting for anything, let alone a grace period.

Pseudo-code using \co{rcu_barrier()} is as follows:

\begin{enumerate}
\item Prevent any new RCU callbacks from being posted.
\item Execute \co{rcu_barrier()}.
\item Allow the module to be unloaded.
\end{enumerate}

There is also an \co{srcu_barrier()} function for SRCU, and you of course
must match the flavor of \co{srcu_barrier()} with that of \co{call_srcu()}.
If your module uses multiple \co{srcu_struct} structures, then it must also
use multiple invocations of \co{srcu_barrier()} when unloading that module.
For example, if it uses \co{call_rcu()}, \co{call_srcu()} on \co{srcu_struct_1}, and
\co{call_srcu()} on \co{srcu_struct_2}, then the following three lines of code
will be required when unloading:

\begin{VerbatimN}
  rcu_barrier();
  srcu_barrier(&srcu_struct_1);
  srcu_barrier(&srcu_struct_2);
\end{VerbatimN}

If latency is of the essence, workqueues could be used to run these
three functions concurrently.

An ancient version of the rcutorture module makes use of \co{rcu_barrier()}
in its exit function as follows:

\begin{fcvlabel}[ln:rcubarrier]
\begin{VerbatimN}[commandchars=\%\@\$]
	static void
	rcu_torture_cleanup(void)
	{
		int i;

		fullstop = 1;        %lnlbl@setglb$
		if (shuffler_task != NULL) {    %lnlbl@ifshuffl$
			VERBOSE_PRINTK_STRING("Stopping rcu_torture_shuffle task");
			kthread_stop(shuffler_task);
		}
		shuffler_task = NULL;

		if (writer_task != NULL) {
			VERBOSE_PRINTK_STRING("Stopping rcu_torture_writer task");
			kthread_stop(writer_task);
		}
		writer_task = NULL;

		if (reader_tasks != NULL) {
			for (i = 0; i < nrealreaders; i++) {
				if (reader_tasks[i] != NULL) {
					VERBOSE_PRINTK_STRING(
					                     "Stopping rcu_torture_reader task");
					kthread_stop(reader_tasks[i]);
				}
				reader_tasks[i] = NULL;
			}
			kfree(reader_tasks);
			reader_tasks = NULL;
		}
		rcu_torture_current = NULL;

		if (fakewriter_tasks != NULL) {
			for (i = 0; i < nfakewriters; i++) {
				if (fakewriter_tasks[i] != NULL) {
					VERBOSE_PRINTK_STRING(
					                      "Stopping rcu_torture_fakewriter task");
					kthread_stop(fakewriter_tasks[i]);
				}
				fakewriter_tasks[i] = NULL;
			}
			kfree(fakewriter_tasks);
			fakewriter_tasks = NULL;
		}

		if (stats_task != NULL) {
			VERBOSE_PRINTK_STRING("Stopping rcu_torture_stats task");
			kthread_stop(stats_task);
		}
		stats_task = NULL;   %lnlbl@statstasknull$

		/* Wait for all RCU callbacks to fire. */
		rcu_barrier();   %lnlbl@rcubarrier$

		rcu_torture_stats_print(); /* -After- the stats thread is stopped! */ %lnlbl@print:b$

		if (cur_ops->cleanup != NULL)
			cur_ops->cleanup();
		if (atomic_read(&n_rcu_torture_error))
			rcu_torture_print_module_parms("End of test: FAILURE");
		else
			rcu_torture_print_module_parms("End of test: SUCCESS");  %lnlbl@print:e$
	}
\end{VerbatimN}
\end{fcvlabel}

\begin{fcvref}[ln:rcubarrier]
\Clnref{setglb} sets a global variable that prevents any RCU callbacks from
re-posting themselves.
This will not be necessary in most cases, since
RCU callbacks rarely include calls to \co{call_rcu()}.
However, the rcutorture
module is an exception to this rule, and therefore needs to set this
global variable.

\Clnrefrange{ifshuffl}{statstasknull} %7,50
stop all the kernel tasks associated with the rcutorture
module.
Therefore, once execution reaches \clnref{rcubarrier}, % line 53
no more rcutorture
RCU callbacks will be posted. The \co{rcu_barrier()} call on \clnref{rcubarrier} % line 53
waits
for any pre-existing callbacks to complete.

Then \clnrefrange{print:b}{print:e} % lines 55-62
print status and do operation-specific cleanup, and
then return, permitting the module-unload operation to be completed.
\end{fcvref}

\QuickQuiz{
	Is there any other situation where \co{rcu_barrier()} might
	be required?
}\QuickQuizAnswer{
	Interestingly enough, \co{rcu_barrier()} was not originally
	implemented for module unloading.
	Nikita Danilov was using
	RCU in a filesystem, which resulted in a similar situation at
	filesystem-unmount time.
	Dipankar Sarma coded up \co{rcu_barrier()}
	in response, so that Nikita could invoke it during the
	filesystem-unmount process.

	Much later, yours truly hit the RCU module-unload problem when
	implementing rcutorture, and found that \co{rcu_barrier()} solves
	this problem as well.
}\QuickQuizEnd

Your module might have additional complications.
For example, if your
module invokes \co{call_rcu()} from timers, you will need to first refrain
from posting new timers, cancel (or wait for) all the already-posted
timers, and only then invoke \co{rcu_barrier()} to wait for any remaining
RCU callbacks to complete.

Of course, if your module uses \co{call_rcu()}, you will need to invoke
\co{rcu_barrier()} before unloading.
Similarly, if your module uses
\co{call_srcu()}, you will need to invoke \co{srcu_barrier()} before unloading,
and on the same \co{srcu_struct} structure.
If your module uses \co{call_rcu()}
\emph{and} \co{call_srcu()}, then (as noted above) you will need to invoke
\co{rcu_barrier()} \emph{and} \co{srcu_barrier()}.


\subsection{Implementing \co{rcu_barrier()}}
\label{sec:rcu:Implementing rcu_barrier()}

Dipankar Sarma's implementation of \co{rcu_barrier()} makes use of the fact
that RCU callbacks are never reordered once queued on one of the per-CPU
queues.
His implementation queues an RCU callback on each of the per-CPU
callback queues, and then waits until they have all started executing, at
which point, all earlier RCU callbacks are guaranteed to have completed.

The original code for \co{rcu_barrier()} was roughly as follows:

\begin{fcvlabel}[ln:original:rcubarrier]
\begin{VerbatimN}[commandchars=\%\@\$]
	void rcu_barrier(void)
	{
		BUG_ON(in_interrupt());  %lnlbl@bugon$
		/* Take cpucontrol mutex to protect against CPU hotplug */
		mutex_lock(&rcu_barrier_mutex);  %lnlbl@lock$
		init_completion(&rcu_barrier_completion);   %lnlbl@init$
		atomic_set(&rcu_barrier_cpu_count, 1);      %lnlbl@set$
		on_each_cpu(rcu_barrier_func, NULL, 0, 1);  %lnlbl@eachcpu$
		if (atomic_dec_and_test(&rcu_barrier_cpu_count))  %lnlbl@dec$
			complete(&rcu_barrier_completion);        %lnlbl@complete$
		wait_for_completion(&rcu_barrier_completion);   %lnlbl@wait$
		mutex_unlock(&rcu_barrier_mutex);   %lnlbl@unlock$
	}
\end{VerbatimN}
\end{fcvlabel}

\begin{fcvref}[ln:original:rcubarrier]
\Clnref{bugon} % Line 3
verifies that the caller is in process context, and
\clnref{lock,unlock} % lines 5 and 12
use \co{rcu_barrier_mutex} to ensure that only one \co{rcu_barrier()} is using the
global completion and counters at a time, which are initialized on
\clnref{init,set} % lines 6 and 7.
\Clnref{eachcpu} % Line 8
causes each CPU to invoke \co{rcu_barrier_func()}, which is
shown below.
Note that the final \qco{1} in \co{on_each_cpu()}'s argument list
ensures that all the calls to \co{rcu_barrier_func()} will have completed
before \co{on_each_cpu()} returns.
\Clnref{dec} % Line 9
removes the initial count from
\co{rcu_barrier_cpu_count}, and if this count is now zero,
\clnref{complete} % line 10
finalizes
the completion, which prevents
\clnref{wait} % line 11
from blocking.
Either way,
\clnref{wait} % line 11
then waits (if needed) for the completion.
\end{fcvref}

\QuickQuiz{
	\begin{fcvref}[ln:original:rcubarrier]
	Why doesn't \clnref{eachcpu} % line 8
	initialize \co{rcu_barrier_cpu_count} to zero,
	thereby avoiding the need for \clnref{dec,complete}? % lines 9 and 10
        \end{fcvref}
}\QuickQuizAnswer{
	\begin{fcvref}[ln:original:rcubarrier]
	Suppose that the \co{on_each_cpu()} function shown on \clnref{eachcpu} % line 8
	was
	delayed, so that CPU~0's \co{rcu_barrier_func()} executed and
	the corresponding grace period elapsed, all before CPU~1's
	\co{rcu_barrier_func()} started executing.
	This would result in
	\co{rcu_barrier_cpu_count} being decremented to zero, so that \clnref{}'s % line 11
	\co{wait_for_completion()} would return immediately, failing to
	wait for CPU~1's callbacks to be invoked.

	Note that this was not a problem when the \co{rcu_barrier()} code
	was first added back in 2005.
	This is because \co{on_each_cpu()}
	disables preemption, which acted as an RCU read-side critical
	section, thus preventing CPU~0's grace period from completing
	until \co{on_each_cpu()} had dealt with all of the CPUs.
	However,
	with the advent of preemptible RCU, \co{rcu_barrier()} no longer
	waited on nonpreemptible regions of code in preemptible kernels,
	that being the job of the new \co{rcu_barrier_sched()} function.

	However, with the RCU flavor consolidation around v4.20, this
	possibility was once again ruled out, because the consolidated
	RCU once again waits on nonpreemptible regions of code.

	Nevertheless, that extra count might still be a good idea.
	Relying on these sort of accidents of implementation can result
	in later surprise bugs when the implementation changes.
        \end{fcvref}
}\QuickQuizEnd


This code was rewritten in 2008 and several times thereafter, but this
still gives the general idea.

The \co{rcu_barrier_func()} runs on each CPU, where it invokes \co{call_rcu()}
to post an RCU callback, as follows:

\begin{fcvlabel}[ln:modern-rcu-barrier]
\begin{VerbatimN}[commandchars=\%\@\$]
	static void rcu_barrier_func(void *notused)
	{
		int cpu = smp_processor_id();   %lnlbl@cpu$
		struct rcu_data *rdp = &per_cpu(rcu_data, cpu);  %lnlbl@rdp$
		struct rcu_head *head;

		head = &rdp->barrier;                 %lnlbl@head$
		atomic_inc(&rcu_barrier_cpu_count);    %lnlbl@inc$
		call_rcu(head, rcu_barrier_callback);  %lnlbl@call$
	}
\end{VerbatimN}
\end{fcvlabel}

\begin{fcvref}[ln:modern-rcu-barrier]
\Clnref{cpu,rdp} % Lines 3 and 4
locate RCU's internal per-CPU \co{rcu_data} structure,
which contains the struct \co{rcu_head} that needed for the later call to
\co{call_rcu()}.
\Clnref{head} % Line 7
picks up a pointer to this struct \co{rcu_head}, and
\clnref{inc} % line 8
increments the global counter.
This counter will later be decremented
by the callback.
\Clnref{call} % Line 9
then registers the \co{rcu_barrier_callback()} on
the current CPU's queue.
\end{fcvref}

The \co{rcu_barrier_callback()} function simply atomically decrements the
\co{rcu_barrier_cpu_count} variable and finalizes the completion when it
reaches zero, as follows:

\begin{VerbatimN}
  1  static void rcu_barrier_callback(struct rcu_head *notused)
  2  {
  3    if (atomic_dec_and_test(&rcu_barrier_cpu_count))
  4      complete(&rcu_barrier_completion);
  5  }
\end{VerbatimN}

\QuickQuiz{
	What happens if CPU~0's \co{rcu_barrier_func()} executes
	immediately (thus incrementing \co{rcu_barrier_cpu_count} to the
	value one), but the other CPU's \co{rcu_barrier_func()} invocations
	are delayed for a full grace period?
	Couldn't this result in
	\co{rcu_barrier()} returning prematurely?
}\QuickQuizAnswer{
	This cannot happen.
	The reason is that \co{on_each_cpu()} has its last
	argument, the wait flag, set to \qco{1}.
	This flag is passed through
	to \co{smp_call_function()} and further to \co{smp_call_function_on_cpu()},
	causing this latter to spin until the cross-CPU invocation of
	\co{rcu_barrier_func()} has completed.
	This by itself would prevent
	a grace period from completing on non-\co{CONFIG_PREEMPTION} kernels,
	since each CPU must undergo a context switch (or other quiescent
	state) before the grace period can complete.
        However, this is
	of no use in \co{CONFIG_PREEMPTION} kernels.

	Therefore, \co{on_each_cpu()} disables preemption across its call
	to \co{smp_call_function()} and also across the local call to
	\co{rcu_barrier_func()}.
	Because recent RCU implementations treat
	preemption-disabled regions of code as RCU read-side critical
	sections, this prevents grace periods from completing.
	This
	means that all CPUs have executed \co{rcu_barrier_func()} before
	the first \co{rcu_barrier_callback()} can possibly execute, in turn
	preventing \co{rcu_barrier_cpu_count} from prematurely reaching zero.

	But if \co{on_each_cpu()} ever decides to forgo disabling preemption,
	as might well happen due to real-time latency considerations,
	initializing \co{rcu_barrier_cpu_count} to one will save the day.
}\QuickQuizEnd

The current \co{rcu_barrier()} implementation is more complex, due to the need
to avoid disturbing idle CPUs (especially on battery-powered systems)
and the need to minimally disturb non-idle CPUs in real-time systems.
In addition, a great many optimizations have been applied.
However,
the code above illustrates the concepts.


\subsection{\co{rcu_barrier()} Summary}
\label{sec:rcu:rcu_barrier() Summary}

The \co{rcu_barrier()} primitive is used relatively infrequently, since most
code using RCU is in the core kernel rather than in modules.
However, if
you are using RCU from an unloadable module, you need to use \co{rcu_barrier()}
so that your module may be safely unloaded.
