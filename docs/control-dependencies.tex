\section{Control dependencies}

A major difficulty with control dependencies is that current compilers
do not support them.
One purpose of this document is therefore to help you prevent your
compiler from breaking your code.
However, control dependencies also pose other challenges, which leads to the
second purpose of this document, namely to help you to avoid breaking
your own code, even in the absence of help from your compiler.

One such challenge is that control dependencies order only later stores.
Therefore, a load-load control dependency will not preserve ordering
unless a read memory barrier is provided.
Consider the following code:

\begin{VerbatimU}
	q = READ_ONCE(a);
	if (q)
		p = READ_ONCE(b);
\end{VerbatimU}

This is not guaranteed to provide any ordering because some types of CPUs
are permitted to predict the result of the load from~\qco{b}.
This prediction can cause other CPUs to see this load as having happened
before the load from~\qco{a}.
This means that an explicit read barrier is required, for example as follows:

\begin{VerbatimU}
	q = READ_ONCE(a);
	if (q) {
		smp_rmb();
		p = READ_ONCE(b);
	}
\end{VerbatimU}

However, stores are not speculated.
This means that ordering is (usually) guaranteed for load-store control
dependencies, as in the following example:

\begin{VerbatimU}
	q = READ_ONCE(a);
	if (q)
		WRITE_ONCE(b, 1);
\end{VerbatimU}

Control dependencies can pair with each other and with other types
of ordering.
But please note that neither the \co{READ_ONCE()} nor the \co{WRITE_ONCE()}
are optional.
Without the \co{READ_ONCE()}, the compiler might fuse the load from~\qco{a}
with other loads.
Without the \co{WRITE_ONCE()}, the compiler might fuse the store to~\qco{b}
with other stores.
Worse yet, the compiler might convert the store into a load and a check followed
by a store, and this compiler-generated load would not be ordered by
the control dependency.

Furthermore, if the compiler is able to prove that the value of
variable~\qco{a} is always non-zero, it would be well within its rights
to optimize the original example by eliminating the \qco{if} statement
as follows:

\begin{VerbatimU}
	q = a;
	b = 1;  /* BUG: Compiler and CPU can both reorder!!! */
\end{VerbatimU}

So don't leave out either the \co{READ_ONCE()} or the \co{WRITE_ONCE()}.
In particular, although \co{READ_ONCE()} does force the compiler to emit a
load, it does \emph{not} force the compiler to actually use the loaded value.

It is tempting to try use control dependencies to enforce ordering on
identical stores on both branches of the \qco{if} statement as follows:

\begin{VerbatimU}
	q = READ_ONCE(a);
	if (q) {
		barrier();
		WRITE_ONCE(b, 1);
		do_something();
	} else {
		barrier();
		WRITE_ONCE(b, 1);
		do_something_else();
	}
\end{VerbatimU}

Unfortunately, current compilers will transform this as follows at high
optimization levels:

\begin{VerbatimU}
	q = READ_ONCE(a);
	barrier();
	WRITE_ONCE(b, 1);  /* BUG: No ordering vs. load from a!!! */
	if (q) {
		/* WRITE_ONCE(b, 1); -- moved up, BUG!!! */
		do_something();
	} else {
		/* WRITE_ONCE(b, 1); -- moved up, BUG!!! */
		do_something_else();
	}
\end{VerbatimU}

Now there is no conditional between the load from~\qco{a} and the store
to~\qco{b}, which means that the CPU is within its rights to reorder them:
The conditional is absolutely required, and must be present in the final
assembly code, after all of the compiler and link-time optimizations
have been applied.
Therefore, if you need ordering in this example, you must use explicit
memory ordering, for example, \co{smp_store_release()}:

\begin{VerbatimU}
	q = READ_ONCE(a);
	if (q) {
		smp_store_release(&b, 1);
		do_something();
	} else {
		smp_store_release(&b, 1);
		do_something_else();
	}
\end{VerbatimU}

Without explicit memory ordering, control-dependency-based ordering is
guaranteed only when the stores differ, for example:

\begin{VerbatimU}
	q = READ_ONCE(a);
	if (q) {
		WRITE_ONCE(b, 1);
		do_something();
	} else {
		WRITE_ONCE(b, 2);
		do_something_else();
	}
\end{VerbatimU}

The initial \co{READ_ONCE()} is still required to prevent the compiler from
knowing too much about the value of~\qco{a}.

But please note that you need to be careful what you do with the local
variable~\qco{q}, otherwise the compiler might be able to guess the value
and again remove the conditional branch that is absolutely required to
preserve ordering.
For example:

\begin{VerbatimU}
	q = READ_ONCE(a);
	if (q % MAX) {
		WRITE_ONCE(b, 1);
		do_something();
	} else {
		WRITE_ONCE(b, 2);
		do_something_else();
	}
\end{VerbatimU}

If \co{MAX} is compile-time defined to be~1, then the compiler knows that
\co{(q \% MAX)} must be equal to zero, regardless of the value of~\qco{q}.
The compiler is therefore within its rights to transform the above code
into the following:

\begin{VerbatimU}
	q = READ_ONCE(a);
	WRITE_ONCE(b, 2);
	do_something_else();
\end{VerbatimU}

Given this transformation, the CPU is not required to respect the ordering
between the load from variable~\qco{a} and the store to variable~\qco{b}.
It is tempting to add a \co{barrier()}, but this does not help.
The conditional is gone, and the barrier won't bring it back.
Therefore, if you need to relying on control dependencies to produce
this ordering, you should make sure that \co{MAX} is greater than one,
perhaps as follows:

\begin{VerbatimU}
	q = READ_ONCE(a);
	BUILD_BUG_ON(MAX <= 1); /* Order load from a with store to b. */
	if (q % MAX) {
		WRITE_ONCE(b, 1);
		do_something();
	} else {
		WRITE_ONCE(b, 2);
		do_something_else();
	}
\end{VerbatimU}

Please note once again that each leg of the \qco{if} statement absolutely
must store different values to~\qco{b}.  As in previous examples, if the two
values were identical, the compiler could pull this store outside of the
\qco{if} statement, destroying the control dependency's ordering properties.

You must also be careful avoid relying too much on boolean short-circuit
evaluation.
Consider this example:

\begin{VerbatimU}
	q = READ_ONCE(a);
	if (q || 1 > 0)
		WRITE_ONCE(b, 1);
\end{VerbatimU}

Because the first condition cannot fault and the second condition is
always true, the compiler can transform this example as follows, again
destroying the control dependency's ordering:

\begin{VerbatimU}
	q = READ_ONCE(a);
	WRITE_ONCE(b, 1);
\end{VerbatimU}

This is yet another example showing the importance of preventing the
compiler from out-guessing your code.
Again, although \co{READ_ONCE()} really does force the compiler to emit
code for a given load, the compiler is within its rights to discard the
loaded value.

In addition, control dependencies apply only to the then-clause and
else-clause of the \qco{if} statement in question.
In particular, they do not necessarily order the code following the
entire \qco{if} statement:

\begin{VerbatimU}
	q = READ_ONCE(a);
	if (q) {
		WRITE_ONCE(b, 1);
	} else {
		WRITE_ONCE(b, 2);
	}
	WRITE_ONCE(c, 1);  /* BUG: No ordering against the read from "a". */
\end{VerbatimU}

It is tempting to argue that there in fact is ordering because the
compiler cannot reorder volatile accesses and also cannot reorder
the writes to~\qco{b} with the condition.
Unfortunately for this line of reasoning, the compiler might compile
the two writes to~\qco{b} as conditional-move instructions, as in this
fanciful pseudo-assembly language:

\begin{VerbatimU}
	ld r1,a
	cmp r1,$0
	cmov,ne r4,$1
	cmov,eq r4,$2
	st r4,b
	st $1,c
\end{VerbatimU}

The control dependencies would then extend only to the pair of \co{cmov}
instructions and the store depending on them.
This means that a weakly ordered CPU would have no dependency of any sort
between the load from~\qco{a} and the store to~\qco{c}.
In short, control dependencies provide ordering only to the stores in
the then-clause and else-clause of the \qco{if} statement in question
(including functions invoked by those two clauses), and not to code
following that \qco{if} statement.


In summary:

\begin{itemize}
\item Control dependencies can order prior loads against later stores.
      However, they do \emph{not} guarantee any other sort of ordering:
      Not prior loads against later loads, nor prior stores against
      later anything.
      If you need these other forms of ordering, use \co{smp_load_acquire()},
      \co{smp_store_release()}, or, in the case of prior stores and later
      loads, \co{smp_mb()}.

\item If both legs of the \qco{if} statement contain identical stores to
      the same variable, then you must explicitly order those stores,
      either by preceding both of them with \co{smp_mb()} or by using
      \co{smp_store_release()}.
      Please note that it is \emph{not} sufficient to use \co{barrier()}
      at beginning and end of each leg of the \qco{if} statement
      because, as shown by the example above, optimizing compilers can
      destroy the control dependency while respecting the letter of the
      \co{barrier()} law.

\item Control dependencies require at least one run-time conditional
      between the prior load and the subsequent store, and this
      conditional must involve the prior load.
      If the compiler is able to optimize the conditional away, it will
      have also optimized away the ordering.
      Careful use of \co{READ_ONCE()} and \co{WRITE_ONCE()} can help to
      preserve the needed conditional.

\item Control dependencies require that the compiler avoid reordering the
      dependency into nonexistence.
      Careful use of \co{READ_ONCE()} or \co{atomic\{,64\}_read()} can
      help to preserve your control dependency.

\item Control dependencies apply only to the then-clause and else-clause
      of the \qco{if} statement containing the control dependency, including
      any functions that these two clauses call.
      Control dependencies do \emph{not} apply to code beyond the end
      of that \qco{if} statement.

\item Control dependencies pair normally with other types of barriers.

\item Control dependencies do \emph{not} provide multicopy atomicity.
      If you need all the CPUs to agree on the ordering of a given store
      against all other accesses, use \co{smp_mb()}.

\item Compilers do not understand control dependencies.
      It is therefore your job to ensure that they do not break your code.

\end{itemize}
