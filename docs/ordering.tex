\section{Overview of memory-ordering operations}

This section gives an overview of the categories of memory-ordering
operations provided by the Linux-kernel memory model (LKMM).


\subsection{Categories of ordering}

This section lists LKMM's three top-level categories of memory-ordering
operations in decreasing order of strength:

\begin{enumerate}
 \item	Barriers (also known as ``fences'').
	A barrier orders some or all of the CPU's prior operations
	against some or all of its subsequent operations.

 \item	Ordered memory accesses.
	These operations order themselves against some or all of
	the CPU's prior accesses or some or all	of the CPU's subsequent
	accesses, depending on the subcategory of the operation.

 \item	Unordered accesses, as the name indicates, have no ordering
	properties except to the extent that they interact with an
	operation in the previous categories.
	This being the real world, some of these ``unordered''
	operations provide limited ordering in some special situations.
\end{enumerate}

Each of the above categories is described in more detail by one of the
following sections.


\subsection{Barriers}

Each of the following categories of barriers is described in its own
subsection below:

\begin{enumerate}
 \item	Full memory barriers.

 \item	Read-modify-write (RMW) ordering augmentation barriers.

 \item	Write memory barrier.

 \item	Read memory barrier.

 \item	Compiler barrier.
\end{enumerate}

Note well that many of these primitives generate absolutely no code
in kernels built with \co{CONFIG_SMP=n}.
Therefore, if you are writing a device driver, which must correctly
order accesses to a physical device even in kernels built with
\co{CONFIG_SMP=n}, please use the ordering primitives provided for
that purpose.
For example, instead of \co{smp_mb()}, use \co{mb()}.
See the ``Linux Kernel Device Drivers'' book or the
\url{https://lwn.net/Articles/698014/} article for more information.


\subsection{Full memory barriers}

The Linux-kernel primitives that provide full ordering include:

\begin{itemize}
 \item	The \co{smp_mb()} full memory barrier.

 \item	Value-returning RMW atomic operations whose names do not end in
	\co{_acquire}, \co{_release}, or \co{_relaxed}.

 \item	RCU's grace-period primitives.
\end{itemize}

First, the \co{smp_mb()} full memory barrier orders all of the CPU's prior
accesses against all subsequent accesses from the viewpoint of all CPUs.
In other words, all CPUs will agree that any earlier action taken
by that CPU happened before any later action taken by that same CPU\@.
For example, consider the following:

\begin{VerbatimU}
	WRITE_ONCE(x, 1);
	smp_mb(); // Order store to x before load from y.
	r1 = READ_ONCE(y);
\end{VerbatimU}

All CPUs will agree that the store to~\qco{x} happened before the load
from~\qco{y}, as indicated by the comment.
And yes, please comment your memory-ordering primitives.
It is surprisingly hard to remember their purpose after even a few months.

Second, some RMW atomic operations provide full ordering.
These operations include value-returning RMW atomic operations
(that is, those with non-void return types) whose names do not end in
\co{_acquire}, \co{_release}, or \co{_relaxed}.
Examples include \co{atomic_add_return()}, \co{atomic_dec_and_test()},
\co{cmpxchg()}, and \co{xchg()}.
Note that conditional RMW atomic operations such as \co{cmpxchg()} are
only guaranteed to provide ordering when they succeed.
When RMW atomic operations provide full ordering, they partition the
CPU's accesses into three groups:

\begin{enumerate}
 \item	All code that executed prior to the RMW atomic operation.

 \item	The RMW atomic operation itself.

 \item	All code that executed after the RMW atomic operation.
\end{enumerate}

All CPUs will agree that any operation in a given partition happened
before any operation in a higher-numbered partition.

In contrast, non-value-returning RMW atomic operations (that is, those
with void return types) do not guarantee any ordering whatsoever.
Nor do value-returning RMW atomic operations whose names end in \co{_relaxed}.
Examples of the former include \co{atomic_inc()} and \co{atomic_dec()},
while examples of the latter include \co{atomic_cmpxchg_relaxed()} and
\co{atomic_xchg_relaxed()}.
Similarly, value-returning non-RMW atomic operations such as
\co{atomic_read()} do not guarantee full ordering, and are covered in
the later section on unordered operations.

Value-returning RMW atomic operations whose names end in \co{_acquire} or
\co{_release} provide limited ordering, and will be described later in this
document.

Finally, RCU's grace-period primitives provide full ordering.
These primitives include \co{synchronize_rcu()}, \co{synchronize_rcu_expedited()},
\co{synchronize_srcu()} and so on.
However, these primitives have orders of magnitude greater overhead
than \co{smp_mb()}, \co{atomic_xchg()}, and so on.
Furthermore, RCU's grace-period primitives can only be invoked in
sleepable contexts.
Therefore, RCU's grace-period primitives are typically instead used to
provide ordering against RCU read-side critical sections, as documented
in their comment headers.
But of course if you need a \co{synchronize_rcu()} to interact with
readers, it costs you nothing to also rely on its additional
full-memory-barrier semantics.
Just please carefully comment this, otherwise your future self will hate you.


\subsection{RMW ordering augmentation barriers}

As noted in the previous section, non-value-returning RMW operations
such as \co{atomic_inc()} and \co{atomic_dec()} guarantee no ordering whatsoever.
Nevertheless, a number of popular CPU families, including x86, provide
full ordering for these primitives.
One way to obtain full ordering on all architectures is to add a call
to \co{smp_mb()}:

\begin{VerbatimU}
	WRITE_ONCE(x, 1);
	atomic_inc(&my_counter);
	smp_mb(); // Inefficient on x86!!!
	r1 = READ_ONCE(y);
\end{VerbatimU}

This works, but the added \co{smp_mb()} adds needless overhead for
x86, on which \co{atomic_inc()} provides full ordering all by itself.
The \co{smp_mb__after_atomic()} primitive can be used instead:

\begin{VerbatimU}
	WRITE_ONCE(x, 1);
	atomic_inc(&my_counter);
	smp_mb__after_atomic(); // Order store to x before load from y.
	r1 = READ_ONCE(y);
\end{VerbatimU}

The \co{smp_mb__after_atomic()} primitive emits code only on CPUs whose
\co{atomic_inc()} implementations do not guarantee full ordering, thus
incurring no unnecessary overhead on x86.
There are a number of variations on the \co{smp_mb__*()} theme:

\begin{itemize}
 \item	\co{smp_mb__before_atomic()}, which provides full ordering prior
	to an unordered RMW atomic operation.

 \item	\co{smp_mb__after_atomic()}, which, as shown above, provides full
	ordering subsequent to an unordered RMW atomic operation.

 \item	\co{smp_mb__after_spinlock()}, which provides full ordering subsequent
	to a successful spinlock acquisition.
	Note that \co{spin_lock()} is always successful but \co{spin_trylock()}
	might not be.

 \item	\co{smp_mb__after_srcu_read_unlock()}, which provides full ordering
	subsequent to an \co{srcu_read_unlock()}.
\end{itemize}

It is bad practice to place code between the \co{smp__*()} primitive and the
operation whose ordering that it is augmenting.
The reason is that the ordering of this intervening code will differ from
one CPU architecture to another.


\subsection{Write memory barrier}

The Linux kernel's write memory barrier is \co{smp_wmb()}.
If a CPU executes the following code:

\begin{VerbatimU}
	WRITE_ONCE(x, 1);
	smp_wmb();
	WRITE_ONCE(y, 1);
\end{VerbatimU}

Then any given CPU will see the write to~\qco{x} has having happened before
the write to~\qco{y}.  However, you are usually better off using a release
store, as described in the ``Release Operations'' section below.

Note that \co{smp_wmb()} might fail to provide ordering for unmarked C-language
stores because profile-driven optimization could determine that the
value being overwritten is almost always equal to the new value.
Such a compiler might then reasonably decide to transform \qtco{x = 1} and
\qtco{y = 1} as follows:

\begin{VerbatimU}
	if (x != 1)
		x = 1;
	smp_wmb(); // BUG: does not order the reads!!!
	if (y != 1)
		y = 1;
\end{VerbatimU}

Therefore, if you need to use \co{smp_wmb()} with unmarked C-language writes,
you will need to make sure that none of the compilers used to build
the Linux kernel carry out this sort of transformation, both now and in
the future.


\subsection{Read memory barrier}

The Linux kernel's read memory barrier is \co{smp_rmb()}.
If a CPU executes the following code:

\begin{VerbatimU}
	r0 = READ_ONCE(y);
	smp_rmb();
	r1 = READ_ONCE(x);
\end{VerbatimU}

Then any given CPU will see the read from~\qco{y} as having preceded the read
from~\qco{x}.
However, you are usually better off using an acquire load, as described
in the ``Acquire Operations'' section below.

\subsection{Compiler barrier}

The Linux kernel's compiler barrier is \co{barrier()}.
This primitive prohibits compiler code-motion optimizations that might
move memory references across the point in the code containing the
\co{barrier()}, but does not constrain hardware memory ordering.
For example, this can be used to prevent to compiler from moving code
across an infinite loop:

\begin{VerbatimU}
	WRITE_ONCE(x, 1);
	while (dontstop)
		barrier();
	r1 = READ_ONCE(y);
\end{VerbatimU}

Without the \co{barrier()}, the compiler would be within its rights to move the
\co{WRITE_ONCE()} to follow the loop.
This code motion could be problematic in the case where an interrupt
handler terminates the loop.
Another way to handle this is to use \co{READ_ONCE()} for the load of
\qco{dontstop}.

Note that the barriers discussed previously use \co{barrier()} or its low-level
equivalent in their implementations.


\subsection{Ordered memory accesses}

The Linux kernel provides a wide variety of ordered memory accesses:

\begin{enumerate}
 \item	Release operations.

 \item	Acquire operations.

 \item	RCU read-side ordering.

 \item	Control dependencies.
\end{enumerate}

Each of the above categories has its own section below.


\subsection{Release operations}

Release operations include \co{smp_store_release()}, \co{atomic_set_release()},
\co{rcu_assign_pointer()}, and value-returning RMW operations whose names
end in \co{_release}.
These operations order their own store against all of the CPU's prior
memory accesses.
Release operations often provide improved readability and performance
compared to explicit barriers.
For example, use of \co{smp_store_release()} saves a line compared to the
\co{smp_wmb()} example above:

\begin{VerbatimU}
	WRITE_ONCE(x, 1);
	smp_store_release(&y, 1);
\end{VerbatimU}

More important, \co{smp_store_release()} makes it easier to connect up the
different pieces of the concurrent algorithm.
The variable stored to by the \co{smp_store_release()}, in this case~\qco{y},
will normally be used in an acquire operation in other parts of the
concurrent algorithm.

To see the performance advantages, suppose that the above example read
from~\qco{x} instead of writing to it.
Then an \co{smp_wmb()} could not guarantee ordering, and an \co{smp_mb()}
would be needed instead:

\begin{VerbatimU}
	r1 = READ_ONCE(x);
	smp_mb();
	WRITE_ONCE(y, 1);
\end{VerbatimU}

But \co{smp_mb()} often incurs much higher overhead than does
\co{smp_store_release()}, which still provides the needed ordering of~\qco{x}
against=\qco{y}.
On x86, the version using \co{smp_store_release()} might compile
to a simple load instruction followed by a simple store instruction.
In contrast, the \co{smp_mb()} compiles to an expensive instruction that
provides the needed ordering.

There is a wide variety of release operations:

\begin{itemize}
 \item	Store operations, including not only the aforementioned
	\co{smp_store_release()}, but also \co{atomic_set_release()}, and
	\co{atomic_long_set_release()}.

 \item	RCU's \co{rcu_assign_pointer()} operation.  This is the same as
	\co{smp_store_release()} except that:
	(1)~It takes the pointer to be assigned to instead of a pointer
	to that pointer, (2)~It is intended to be used in conjunction with
	\co{rcu_dereference()} and similar rather than \co{smp_load_acquire()},
	and (3)~It checks for an RCU-protected pointer in ``sparse'' runs.

 \item	Value-returning RMW operations whose names end in \co{_release},
	such as \co{atomic_fetch_add_release()} and \co{cmpxchg_release()}.
	Note that release ordering is guaranteed only against the
	memory-store portion of the RMW operation, and not against the
	memory-load portion.
	Note also that conditional operations such as \co{cmpxchg_release()}
	are only guaranteed to provide ordering when they succeed.
\end{itemize}

As mentioned earlier, release operations are often paired with acquire
operations, which are the subject of the next section.


\subsubsection{Acquire operations}

Acquire operations include \co{smp_load_acquire()}, \co{atomic_read_acquire()},
and value-returning RMW operations whose names end in \co{_acquire}.
These operations order their own load against all of the CPU's subsequent
memory accesses.
Acquire operations often provide improved performance and readability
compared to explicit barriers.
For example, use of \co{smp_load_acquire()} saves a line compared to the
\co{smp_rmb()} example above:

\begin{VerbatimU}
	r0 = smp_load_acquire(&y);
	r1 = READ_ONCE(x);
\end{VerbatimU}

As with \co{smp_store_release()}, this also makes it easier to connect
the different pieces of the concurrent algorithm by looking for the
\co{smp_store_release()} that stores to~\qco{y}.
In addition, \co{smp_load_acquire()} improves upon \co{smp_rmb()} by
ordering against subsequent stores as well as against subsequent loads.

There are a couple of categories of acquire operations:

\begin{itemize}
 \item	Load operations, including not only the aforementioned
	\co{smp_load_acquire()}, but also \co{atomic_read_acquire()}, and
	\co{atomic64_read_acquire()}.

 \item	Value-returning RMW operations whose names end in \co{_acquire},
	such as \co{atomic_xchg_acquire()} and \co{atomic_cmpxchg_acquire()}.
	Note that acquire ordering is guaranteed only against the
	memory-load portion of the RMW operation, and not against the
	memory-store portion.
	Note also that conditional operations such as
	\co{atomic_cmpxchg_acquire()} are only guaranteed to provide
	ordering when they succeed.
\end{itemize}

Symmetry being what it is, acquire operations are often paired with the
release operations covered earlier.
For example, consider the following example, where \co{task0()} and
\co{task1()} execute concurrently:

\begin{VerbatimU}
	void task0(void)
	{
		WRITE_ONCE(x, 1);
		smp_store_release(&y, 1);
	}

	void task1(void)
	{
		r0 = smp_load_acquire(&y);
		r1 = READ_ONCE(x);
	}
\end{VerbatimU}

If~\qco{x} and~\qco{y} are both initially zero, then either \co{r0}'s final
value will be zero or \co{r1}'s final value will be one, thus providing
the required ordering.


\subsubsection{RCU read-side ordering}

This category includes read-side markers such as \co{rcu_read_lock()}
and \co{rcu_read_unlock()} as well as pointer-traversal primitives such as
\co{rcu_dereference()} and \co{srcu_dereference()}.

Compared to locking primitives and RMW atomic operations, markers
for RCU read-side critical sections incur very low overhead because
they interact only with the corresponding grace-period primitives.
For example, the \co{rcu_read_lock()} and \co{rcu_read_unlock()} markers interact
with \co{synchronize_rcu()}, \co{synchronize_rcu_expedited()}, and \co{call_rcu()}.
The way this works is that if a given call to \co{synchronize_rcu()} cannot
prove that it started before a given call to \co{rcu_read_lock()}, then
that \co{synchronize_rcu()} must block until the matching \co{rcu_read_unlock()}
is reached.
For more information, please see the \co{synchronize_rcu()}
docbook header comment and the material in \path{Documentation/RCU}.

RCU's pointer-traversal primitives, including \co{rcu_dereference()} and
\co{srcu_dereference()}, order their load (which must be a pointer) against any
of the CPU's subsequent memory accesses whose address has been calculated
from the value loaded.
There is said to be an \emph{address dependency} from the value returned by
the \co{rcu_dereference()} or \co{srcu_dereference()} to that subsequent
memory access.

A call to \co{rcu_dereference()} for a given RCU-protected pointer is
usually paired with a call to a call to \co{rcu_assign_pointer()} for that
same pointer in much the same way that a call to \co{smp_load_acquire()} is
paired with a call to \co{smp_store_release()}.
Calls to \co{rcu_dereference()} and \co{rcu_assign_pointer} are often
buried in other APIs, for example, the RCU list API members defined in
\path{include/linux/rculist.h}.
For more information, please see the docbook headers in that file, the most
recent LWN article on the RCU API (\url{https://lwn.net/Articles/777036/}),
and of course the material in \path{Documentation/RCU}.

If the pointer value is manipulated between the \co{rcu_dereference()}
that returned it and a later \co{dereference()}, please read
\cref{sec:rcu:Proper care and feeding of return values from rcu_dereference()}.
It can also be quite helpful to review uses in the Linux kernel.


\subsubsection{Control dependencies}

A control dependency extends from a marked load (\co{READ_ONCE()} or stronger)
through an \qco{if} condition to a marked store (\co{WRITE_ONCE()} or stronger)
that is executed only by one of the legs of that \qco{if} statement.
Control dependencies are so named because they are mediated by
control-flow instructions such as comparisons and conditional branches.

In short, you can use a control dependency to enforce ordering between
an \co{READ_ONCE()} and a \co{WRITE_ONCE()} when there is an \qco{if} condition
between them.
The canonical example is as follows:

\begin{VerbatimU}
	q = READ_ONCE(a);
	if (q)
		WRITE_ONCE(b, 1);
\end{VerbatimU}

In this case, all CPUs would see the read from~\qco{a} as happening before
the write to~\qco{b}.

However, control dependencies are easily destroyed by compiler
optimizations, so any use of control dependencies must take into account
all of the compilers used to build the Linux kernel.
Please see the \path{control-dependencies.txt} file for more information.


\subsection{Unordered accesses}

Each of these two categories of unordered accesses has a section below:

\begin{enumerate}
 \item	Unordered marked operations.

 \item	Unmarked C-language accesses.
\end{enumerate}


\subsubsection{Unordered marked operations}

Unordered operations to different variables are just that, unordered.
However, if a group of CPUs apply these operations to a single variable,
all the CPUs will agree on the operation order.
Of course, the ordering of unordered marked accesses can also be
constrained using the mechanisms described earlier in this document.

These operations come in three categories:

\begin{itemize}
 \item	Marked writes, such as \co{WRITE_ONCE()} and \co{atomic_set()}.
	These primitives required the compiler to emit the corresponding store
	instructions in the expected execution order, thus suppressing
	a number of destructive optimizations.
	However, they provide no hardware ordering guarantees, and in
	fact many CPUs will happily reorder marked writes with each other
	or with other unordered operations, unless these operations are
	to the same variable.

 \item	Marked reads, such as \co{READ_ONCE()} and \co{atomic_read()}.
	These primitives required the compiler to emit the corresponding load
	instructions in the expected execution order, thus suppressing
	a number of destructive optimizations.
	However, they provide no hardware ordering guarantees, and in
	fact many CPUs will happily reorder marked reads with each other
	or with other unordered operations, unless these operations are
	to the same variable.

 \item	Unordered RMW atomic operations.
	These are non-value-returning RMW atomic operations whose names
	do not end in \co{_acquire} or \co{_release}, and also
	value-returning RMW operations whose names end in \co{_relaxed}.
	Examples include \co{atomic_add()}, \co{atomic_or()},
	and \co{atomic64_fetch_xor_relaxed()}.
	These operations do carry out the specified RMW operation atomically,
	for example, five concurrent \co{atomic_inc()} operations applied
	to a given variable will reliably increase the value of that
	variable by five.
	However, many CPUs will happily reorder these operations with
	each other or with other unordered operations.

	This category of operations can be efficiently ordered using
	\co{smp_mb__before_atomic()} and \co{smp_mb__after_atomic()}, as was
	discussed in the ``RMW Ordering Augmentation Barriers'' section.
\end{itemize}

In short, these operations can be freely reordered unless they are all
operating on a single variable or unless they are constrained by one of
the operations called out earlier in this document.


\subsubsection{Unmarked C-language accesses}

Unmarked C-language accesses are normal variable accesses to normal
variables, that is, to variables that are not \qco{volatile} and are not
C11 atomic variables.
These operations provide no ordering guarantees, and further do not
guarantee ``atomic'' access.
For example, the compiler might (and sometimes does) split a plain
C-language store into multiple smaller stores.
A load from that same variable running on some other CPU while such
a store is executing might see a value that is a mashup of the old
value and the new value.

Unmarked C-language accesses are unordered, and are also subject to
any number of compiler optimizations, many of which can break your
concurrent code.
It is possible to used unmarked C-language accesses for shared variables
that are subject to concurrent access, but great care is required on
an ongoing basis.
The compiler-constraining \co{barrier()} primitive can be helpful, as
can the various ordering primitives discussed in this document.
It nevertheless bears repeating that use of unmarked C-language accesses
requires careful attention to not just your code, but to all the
compilers that might be used to build it.
Such compilers might replace a series of loads with a single load,
and might replace a series of stores with a single store.
Some compilers will even split a single store into multiple smaller stores.

But there are some ways of using unmarked C-language accesses for shared
variables without such worries:

\begin{itemize}
 \item	Guard all accesses to a given variable by a particular lock,
	so that there are never concurrent conflicting accesses to
	that variable.
	(There are ``conflicting accesses'' when
	(1)~at least one of the concurrent accesses to a variable is an
	unmarked C-language access and (2)~when at least one of those
	accesses is a write, whether marked or not.)

 \item	As above, but using other synchronization primitives such
	as reader-writer locks or sequence locks.

 \item	Use locking or other means to ensure that all concurrent accesses
	to a given variable are reads.

 \item	Restrict use of a given variable to statistics or heuristics
	where the occasional bogus value can be tolerated.

 \item	Declare the accessed variables as C11 atomics.
	\url{https://lwn.net/Articles/691128/}

 \item	Declare the accessed variables as \qco{volatile}.
\end{itemize}

If you need to live more dangerously, please do take the time to
understand the compilers.
One place to start is these two LWN articles:

\begin{description}[style=nextline]
\item[Who's afraid of a big bad optimizing compiler?]
	\url{https://lwn.net/Articles/793253}
\item[Calibrating your fear of big bad optimizing compilers]
	\url{https://lwn.net/Articles/799218}
\end{description}  

Used properly, unmarked C-language accesses can reduce overhead on
fastpaths.
However, the price is great care and continual attention to your compiler
as new versions come out and as new optimizations are enabled.
