\section{Linux-kernel memory model litmus tests}

This section describes the LKMM litmus-test format by example, describes
some tricks and traps, and finally outlines LKMM's limitations.
Earlier versions of this material appeared in a number of LWN articles,
including:

\begin{description}[style=nextline]
 \item[A formal kernel memory-ordering model (part 2)]
	\url{https://lwn.net/Articles/720550/}
 \item[Axiomatic validation of memory barriers and atomic instructions]
	\url{https://lwn.net/Articles/608550/}
 \item[Validating Memory Barriers and Atomic Instructions]
	\url{https://lwn.net/Articles/470681/}
\end{description}

This document presents information in decreasing order of applicability,
so that, where possible, the information that has proven more commonly
useful is shown near the beginning.

For information on installing LKMM, including the underlying \qco{herd7}
tool, please see \path{tools/memory-model/README}.


\subsection{Copy-pasta}

As with other software, it is often better (if less macho) to adapt an
existing litmus test than it is to create one from scratch.
A number of litmus tests may be found in the kernel source tree:

\begin{itemize}
  \item \path{tools/memory-model/litmus-tests/}
  \item \path{Documentation/litmus-tests/}
\end{itemize}

Several thousand more example litmus tests are available on github
and kernel.org:

\begin{itemize}
  \item \url{https://github.com/paulmckrcu/litmus}
  \item \url{https://git.kernel.org/pub/scm/linux/kernel/git/paulmck/perfbook.git/tree/CodeSamples/formal/herd}
  \item \url{https://git.kernel.org/pub/scm/linux/kernel/git/paulmck/perfbook.git/tree/CodeSamples/formal/litmus}
\end{itemize}

The \co{-l} and \co{-L} arguments to \qco{git grep} can be quite helpful
in identifying existing litmus tests that are similar to the one you need.
But even if you start with an existing litmus test, it is still helpful
to have a good understanding of the litmus-test format.


\subsection{Examples and format}

This section describes the overall format of litmus tests, starting
with a small example of the message-passing pattern and moving on to
more complex examples that illustrate explicit initialization and LKMM's
minimalistic set of flow-control statements.


\subsubsection{Message-passing example}

This section gives an overview of the format of a litmus test using an
example based on the common message-passing use case.
This use case appears often in the Linux kernel.
For example, a flag (modeled by~\qco{y} below) indicates that a buffer
(modeled by~\qco{x} below) is now completely filled in and ready for use.
It would be very bad if the consumer saw the flag set, but, due to memory
misordering, saw old values in the buffer.

This example asks whether \co{smp_store_release()} and \co{smp_load_acquire()}
suffices to avoid this bad outcome:

\begin{fcvlabel}[ln:docs:litmus:MP]
\begin{VerbatimN}[commandchars=\%\@\$]
	C MP+pooncerelease+poacquireonce  %lnlbl@C$

	{}    %lnlbl@init$

	P0(int *x, int *y)   %lnlbl@P0:b$
	{
		WRITE_ONCE(*x, 1);   %lnlbl@P0:x$
		smp_store_release(y, 1);   %lnlbl@P0:y$
	}   %lnlbl@P0:e$

	P1(int *x, int *y)   %lnlbl@P1:b$
	{
		int r0;   %lnlbl@P1:r0$
		int r1;   %lnlbl@P1:r1$

		r0 = smp_load_acquire(y);   %lnlbl@P1:y$
		r1 = READ_ONCE(*x);    %lnlbl@P1:x$
	}   %lnlbl@P1:e$

	exists (1:r0=1 /\ 1:r1=0)  %lnlbl@exists$
\end{VerbatimN}
\end{fcvlabel}

\begin{fcvref}[ln:docs:litmus:MP]
\Clnref{C} starts with~\qco{C}, which identifies this file as being in the
LKMM C-language format (which, as we will see, is a small fragment
of the full C language).
The remainder of \clnref{C} is the name of the test, which by convention
is the filename with the \qco{.litmus} suffix stripped.
In this case, the actual test may be found in
\path{tools/memory-model/litmus-tests/MP+pooncerelease+poacquireonce.litmus}
in the Linux-kernel source tree.

Mechanically generated litmus tests will often have an optional
double-quoted comment string on the second line.
Such strings are ignored when running the test.
Yes, you can add your own comments to litmus tests, but this is a bit
involved due to the use of multiple parsers.
For now, you can use C-language comments in the C code, and these comments
may be in either the \qco{/* */} or the \qco{//} style.
A later section will cover the full litmus-test commenting story.

\Clnref{init} is the initialization section.
Because the default initialization to zero suffices for this test,
the~\qco{\{\}} syntax is used, which mean the initialization section is empty.
Litmus tests requiring non-default initialization must have non-empty
initialization sections, as in the example that will be presented later
in this document.

\Clnref{P0:b,P0:e} show the first process and \clnref{P1:b,P1:e} the
second process.
Each process corresponds to a Linux-kernel task (or kthread, workqueue, thread,
and so on; LKMM discussions often use these terms interchangeably).
The name of the first process is~\qco{P0} and that of the second~\qco{P1}.
You can name your processes anything you like as long as the names consist
of a single~\qco{P} followed by a number, and as long as the numbers are
consecutive starting with zero.
This can actually be quite helpful, for example, a \co{.litmus} file
matching \qco{^P1(} but not matching \qco{^P2(} must contain a two-process
litmus test.

The argument list for each function are pointers to the global variables
used by that function.
Unlike normal C-language function parameters, the names are significant.
The fact that both \co{P0()} and \co{P1()} have a formal parameter
named~\qco{x} means that these two processes are working with the
same global variable, also named~\qco{x}.
So the \qco{int *x, int *y} on \co{P0()} and \co{P1()} mean that both
processes are working with two shared global variables, \qco{x} and~\qco{y}.
Global variables are always passed to processes by reference, hence
\qco{P0(int *x, int *y)}, but \emph{never} \qco{P0(int x, int y)}.

\co{P0()} has no local variables, but \co{P1()} has two of them
named~\qco{r0} and~\qco{r1}.
These names may be freely chosen, but for historical reasons stemming from
other litmus-test formats, it is conventional to use names consisting
of~\qco{r} followed by a number as shown here.
A common bug in litmus tests is forgetting to add a global variable to
a process's parameter list.
This will sometimes result in an error message, but can also cause the
intended global to instead be silently treated as an undeclared local
variable.

Each process's code is similar to Linux-kernel C, as can be seen on
lines \lnref{P0:x}--\lnref{P0:y} and \lnref{P1:r0}--\lnref{P1:x}.
This code may use many of the Linux kernel's atomic operations, some
of its exclusive-lock functions, and some of its RCU and SRCU functions.
An approximate list of the currently supported functions may be found in
the \path{linux-kernel.def} file.

The \co{P0()} process does \qtco{WRITE_ONCE(*x, 1)} on \clnref{P0:x}.
Because \qco{x}~is a pointer in \co{P0()}'s parameter list, this does an
unordered store to global variable~\qco{x}.
\Clnref{P0:y} does \qtco{smp_store_release(y, 1)}, and because \qco{y}~is
also in \co{P0()}'s parameter list, this does a release store to global
variable~\qco{y}.

The \co{P1()} process declares two local variables on \clnref{P1:r0,P1:r1}.
\Clnref{P1:y} does \qtco{r0 = smp_load_acquire(y)} which does an acquire load
from global variable~\qco{y} into local variable~\qco{r0}.
\Clnref{P1:x} does a \qtco{r1 = READ_ONCE(*x)}, which does an unordered
load from~\qco{*x} into local variable~\qco{r1}.
Both \qco{x}~and~\qco{y} are in \co{P1()}'s parameter list, so both
reference the same global variables that are used by \co{P0()}.

\Clnref{exists} is the \qco{exists} assertion expression to evaluate
the final state.
This final state is evaluated after the dust has settled; both processes
have completed and all of their memory references and memory barriers
have propagated to all parts of the system.
The references to the local variables~\qco{r0} and~\qco{r1} in \clnref{exists}
must be prefixed with \qco{1:} to specify which process they are local to.

Note that the assertion expression is written in the litmus-test
language rather than in~C.
For example, single~\qco{=} is an equality operator rather than an assignment.
The ``\co{/\\}'' character combination means ``and''.
Similarly, ``\co{\\/}'' stands for ``or''.
Both of these are ASCII-art representations of the corresponding
mathematical symbols.
Finally, \qco{\~} stands for ``logical not'', which is~\qco{\!} in~C,
and not to be confused with the C-language \qco{\~}~operator which instead
stands for ``bitwise not''.
Parentheses may be used to override precedence.

The \qco{exists} assertion on \clnref{exists} is satisfied if the consumer
sees the flag (\qco{y}) set but the buffer (\qco{x}) as not yet filled in,
that is, if \co{P1()} loaded a value from~\qco{x} that was equal to~1
but loaded a value from~\qco{y} that was still equal to zero.
\end{fcvref}

This example can be checked by running the following command, which
absolutely must be run from the \path{tools/memory-model} directory and from
this directory only:

\begin{VerbatimU}[breaklines=true]
herd7 -conf linux-kernel.cfg litmus-tests/MP+pooncerelease+poacquireonce.litmus
\end{VerbatimU}

The output is the result of something similar to a full state-space
search, and is as follows:

\begin{fcvlabel}[ln:docs:litmus:MPout]
\begin{VerbatimN}[commandchars=\%\@\$]
	Test MP+pooncerelease+poacquireonce Allowed  %lnlbl@name$
	States 3          %lnlbl@stat:b$
	1:r0=0; 1:r1=0;
	1:r0=0; 1:r1=1;
	1:r0=1; 1:r1=1;   %lnlbl@stat:e$
	No                %lnlbl@no$
	Witnesses         %lnlbl@witness$
	Positive: 0 Negative: 3    %lnlbl@posneg$
	Condition exists (1:r0=1 /\ 1:r1=0)   %lnlbl@exists$
	Observation MP+pooncerelease+poacquireonce Never 0 3   %lnlbl@obs$
	Time MP+pooncerelease+poacquireonce 0.00   %lnlbl@time$
	Hash=579aaa14d8c35a39429b02e698241d09      %lnlbl@hash$
\end{VerbatimN}
\end{fcvlabel}

\begin{fcvref}[ln:docs:litmus:MPout]
The most pertinent line is \clnref{obs}, which contains \qco{Never 0 3}, which
indicates that the bad result flagged by the \qco{exists} clause never
happens.
This line might instead say \qco{Sometimes} to indicate that the
bad result happened in some but not all executions, or it might say
\qco{Always} to indicate that the bad result happened in all executions.
(The \co{herd7} tool doesn't judge, so it is only an LKMM convention that the
\qco{exists} clause indicates a bad result.
To see this, invert the \qco{exists} clause's condition and run the test.)
The numbers (\qco{0 3}) at the end of this line indicate the number of end
states satisfying the \qco{exists} clause (0) and the number not not
satisfying that clause (3).

Another important part of this output is shown in \clnrefrange{stat:b}{stat:e},
repeated here:
\end{fcvref}

\begin{fcvlabel}[ln:docs:litmus:MPoutpart]
\begin{VerbatimN}[commandchars=\%\@\$,firstnumber=2]
	States 3         %lnlbl@stat:b$
	1:r0=0; 1:r1=0;  %lnlbl@stat:00$
	1:r0=0; 1:r1=1;  %lnlbl@stat:01$
	1:r0=1; 1:r1=1;  %lnlbl@stat:11$
\end{VerbatimN}
\end{fcvlabel}

\begin{fcvref}[ln:docs:litmus:MPoutpart]
\Clnref{stat:b} gives the total number of end states, and each of
\clnrefrange{stat:00}{stat:11} list one of these states, with the first
(\qco{1:r0=0; 1:r1=0;}) indicating that both of \co{P1()}'s loads returned
the value~\qco{0}.
\end{fcvref}
\begin{fcvref}[ln:docs:litmus:MPout]
As expected, given the \qco{Never} on \clnref{obs}, the state flagged by
the \qco{exists} clause is not listed.
This full list of states can be helpful when debugging a new litmus test.

The rest of the output is not normally needed, either due to irrelevance
or due to being redundant with the lines discussed above.
However, the following paragraph lists them for the benefit of readers
possessed of an insatiable curiosity.
Other readers should feel free to skip ahead.

\Clnref{name} echos the test name, along with the \qco{Test} and \qco{Allowed}.
\Clnref{no}'s \qco{No} says that the \qco{exists} clause was not satisfied by
any execution, and as such it has the same meaning as \clnref{obs}'s \qco{Never}.
\Clnref{witness} is a lead-in to \clnref{posneg}'s \qco{Positive: 0 Negative: 3},
which lists the number of end states satisfying and not satisfying the
\qco{exists} clause, just like the two numbers at the end of \clnref{obs}.
\Clnref{exists} repeats the \qco{exists} clause so that you don't have to
look it up in the litmus-test file.
The number at the end of \clnref{time} (which begins with \qco{Time}) gives the
time in seconds required to analyze the litmus test.
Small tests such as this one complete in a few milliseconds, so \qco{0.00}
is quite common.
\Clnref{hash} gives a hash of the contents for the litmus-test file, and is used
by tooling that manages litmus tests and their output.
This tooling is used by people modifying LKMM itself, and among other things
lets such people know which of the several thousand relevant litmus tests were
affected by a given change to LKMM\@.
\end{fcvref}


\subsubsection{Initialization}

The previous example relied on the default zero initialization for~\qco{x}
and~\qco{y}, but a similar litmus test could instead initialize them
to some other value:

\begin{fcvlabel}[ln:docs:litmus:MPinit]
\begin{VerbatimN}[commandchars=\%\@\$]
	C MP+pooncerelease+poacquireonce

	{               %lnlbl@init:b$
		x=42;
		y=42;
	}               %lnlbl@init:e$

	P0(int *x, int *y)
	{
		WRITE_ONCE(*x, 1);
		smp_store_release(y, 1);
	}

	P1(int *x, int *y)
	{
		int r0;
		int r1;

		r0 = smp_load_acquire(y);
		r1 = READ_ONCE(*x);
	}

	exists (1:r0=1 /\ 1:r1=42)   %lnlbl@exists$
\end{VerbatimN}
\end{fcvlabel}

\begin{fcvref}[ln:docs:litmus:MPinit]
\Clnrefrange{init:b}{init:e} now initialize both~\qco{x} and~\qco{y} to
the value~42.
This also means that the \qco{exists} clause on \clnref{exists} must change
\qco{1:r1=0} to \qco{1:r1=42}.
\end{fcvref}

Running the test gives the same overall result as before, but with the
value~42 appearing in place of the value zero:

\begin{VerbatimN}
	Test MP+pooncerelease+poacquireonce Allowed
	States 3
	1:r0=1; 1:r1=1;
	1:r0=42; 1:r1=1;
	1:r0=42; 1:r1=42;
	No
	Witnesses
	Positive: 0 Negative: 3
	Condition exists (1:r0=1 /\ 1:r1=42)
	Observation MP+pooncerelease+poacquireonce Never 0 3
	Time MP+pooncerelease+poacquireonce 0.02
	Hash=ab9a9b7940a75a792266be279a980156
\end{VerbatimN}

It is tempting to avoid the open-coded repetitions of the value~\qco{42}
by defining another global variable \qco{initval=42} and replacing all
occurrences of \qco{42} with \qco{initval}.
This will not, repeat \emph{not}, initialize~\qco{x} and~\qco{y} to 42,
but instead to the address of \qco{initval} (try it!).
See the section below on linked lists to learn more about why this approach
to initialization can be useful.


\subsubsection{Control structures}

LKMM supports the C-language \qco{if} statement, which allows modeling of
conditional branches.
In LKMM, conditional branches can affect ordering, but only if you are
\emph{very} careful (compilers are surprisingly able to optimize away
conditional branches).
The following example shows the ``load buffering'' (LB) use case that is
used in the Linux kernel to synchronize between ring-buffer producers and
consumers.
In the example below, \co{P0()} is one side checking to see if an operation
may proceed and \co{P1()} is the other side completing its update.

\begin{fcvlabel}[ln:docs:litmus:LB]
\begin{VerbatimN}[commandchars=\%\@\$]
	C LB+fencembonceonce+ctrlonceonce

	{}

	P0(int *x, int *y)
	{
		int r0;

		r0 = READ_ONCE(*x);         %lnlbl@read$
		if (r0)                     %lnlbl@if$
			WRITE_ONCE(*y, 1);  %lnlbl@write$
	}

	P1(int *x, int *y)
	{
		int r0;

		r0 = READ_ONCE(*y);
		smp_mb();
		WRITE_ONCE(*x, 1);
	}

	exists (0:r0=1 /\ 1:r0=1)      %lnlbl@exists$
\end{VerbatimN}
\end{fcvlabel}

\begin{fcvref}[ln:docs:litmus:LB]
\co{P1()}'s \qco{if} statement on \clnref{if} works as expected, so that
\clnref{write} is executed only if \clnref{read} loads a non-zero value
from~\qco{x}.
Because \co{P1()}'s write of~\qco{1} to~\qco{x} happens only after
\co{P1()}'s read from~\qco{y}, one would hope that the \qco{exists} clause
cannot be satisfied.
\end{fcvref}
LKMM agrees:

\begin{VerbatimN}
	Test LB+fencembonceonce+ctrlonceonce Allowed
	States 2
	0:r0=0; 1:r0=0;
	0:r0=1; 1:r0=0;
	No
	Witnesses
	Positive: 0 Negative: 2
	Condition exists (0:r0=1 /\ 1:r0=1)
	Observation LB+fencembonceonce+ctrlonceonce Never 0 2
	Time LB+fencembonceonce+ctrlonceonce 0.00
	Hash=e5260556f6de495fd39b556d1b831c3b
\end{VerbatimN}

However, there is no \qco{while} statement due to the fact that full
state-space search has some difficulty with iteration.
However, there are tricks that may be used to handle some special cases,
which are discussed below.
In addition, loop-unrolling tricks may be applied, albeit sparingly.


\subsection{Tricks and traps}

This section covers extracting debug output from \co{herd7}, emulating
spin loops, handling trivial linked lists, adding comments to litmus tests,
emulating \co{call_rcu()}, and finally tricks to improve \co{herd7} performance
in order to better handle large litmus tests.


\subsubsection{Debug output}

By default, the \co{herd7} state output includes all variables mentioned
in the \qco{exists} clause.
But sometimes debugging efforts are greatly aided by the values of other
variables.
Consider this litmus test
(\path{tools/memory-order/litmus-tests/SB+rfionceonce-poonceonces.litmus}
but
slightly modified), which probes an obscure corner of hardware memory ordering:

\begin{VerbatimN}[commandchars=\%\@\$]
	C SB+rfionceonce-poonceonces

	{}

	P0(int *x, int *y)
	{
		int r1;
		int r2;

		WRITE_ONCE(*x, 1);
		r1 = READ_ONCE(*x);
		r2 = READ_ONCE(*y);
	}

	P1(int *x, int *y)
	{
		int r3;
		int r4;

		WRITE_ONCE(*y, 1);
		r3 = READ_ONCE(*y);
		r4 = READ_ONCE(*x);
	}

	exists (0:r2=0 /\ 1:r4=0)
\end{VerbatimN}

The \co{herd7} output is as follows:

\begin{VerbatimN}
	Test SB+rfionceonce-poonceonces Allowed
	States 4
	0:r2=0; 1:r4=0;
	0:r2=0; 1:r4=1;
	0:r2=1; 1:r4=0;
	0:r2=1; 1:r4=1;
	Ok
	Witnesses
	Positive: 1 Negative: 3
	Condition exists (0:r2=0 /\ 1:r4=0)
	Observation SB+rfionceonce-poonceonces Sometimes 1 3
	Time SB+rfionceonce-poonceonces 0.01
	Hash=c7f30fe0faebb7d565405d55b7318ada
\end{VerbatimN}

(This output indicates that CPUs are permitted to ``snoop their own
store buffers'', which all of Linux's CPU families other than s390 will
happily do.
Such snooping results in disagreement among CPUs on the order of stores
from different CPUs, which is rarely an issue.)

But the \co{herd7} output shows only the two variables mentioned in the
\qco{exists} clause.
Someone modifying this test might wish to know the values of~\qco{x},
\qco{y}, \qco{0:r1}, and \qco{0:r3} as well.
\begin{fcvref}[ln:docs:litmus:SBloc]
The \qco{locations} statement on \clnref{loc} shows how to cause \co{herd7} to
display additional variables:
\end{fcvref}

\begin{fcvlabel}[ln:docs:litmus:SBloc]
\begin{VerbatimN}[commandchars=\%\@\$]
	C SB+rfionceonce-poonceonces

	{}

	P0(int *x, int *y)
	{
		int r1;
		int r2;

		WRITE_ONCE(*x, 1);
		r1 = READ_ONCE(*x);
		r2 = READ_ONCE(*y);
	}

	P1(int *x, int *y)
	{
		int r3;
		int r4;

		WRITE_ONCE(*y, 1);
		r3 = READ_ONCE(*y);
		r4 = READ_ONCE(*x);
	}

	locations [0:r1; 1:r3; x; y]    %lnlbl@loc$
	exists (0:r2=0 /\ 1:r4=0)
\end{VerbatimN}
\end{fcvlabel}

The \co{herd7} output then displays the values of all the variables:

\begin{VerbatimN}
	Test SB+rfionceonce-poonceonces Allowed
	States 4
	0:r1=1; 0:r2=0; 1:r3=1; 1:r4=0; x=1; y=1;
	0:r1=1; 0:r2=0; 1:r3=1; 1:r4=1; x=1; y=1;
	0:r1=1; 0:r2=1; 1:r3=1; 1:r4=0; x=1; y=1;
	0:r1=1; 0:r2=1; 1:r3=1; 1:r4=1; x=1; y=1;
	Ok
	Witnesses
	Positive: 1 Negative: 3
	Condition exists (0:r2=0 /\ 1:r4=0)
	Observation SB+rfionceonce-poonceonces Sometimes 1 3
	Time SB+rfionceonce-poonceonces 0.01
	Hash=40de8418c4b395388f6501cafd1ed38d
\end{VerbatimN}

What if you would like to know the value of a particular global variable
at some particular point in a given process's execution?
One approach is to use a \co{READ_ONCE()} to load that global variable
into a new local variable, then add that local variable to the \qco{locations}
clause.
But be careful:
In some litmus tests, adding a \co{READ_ONCE()} will change the outcome!
For one example, please see the \path{C-READ_ONCE.litmus} and
\path{C-READ_ONCE-omitted.litmus} tests located here:

\begin{itemize}
  \item	\url{https://github.com/paulmckrcu/litmus/blob/master/manual/kernel/}
\end{itemize}


\subsubsection{Spin loops}

The analysis carried out by \co{herd7} explores full state space, which is
at best of exponential time complexity.
Adding processes and increasing the amount of code in a give process can
greatly increase execution time.
Potentially infinite loops, such as those used to wait for locks to become
available, are clearly problematic.

Fortunately, it is possible to avoid state-space explosion by specially
modeling such loops.
For example, the following litmus tests emulates locking using
\co{xchg_acquire()}, but instead of enclosing \co{xchg_acquire()}
in a spin loop, it instead excludes executions that fail to acquire the
lock using a \co{herd7} \qco{filter} clause.
Note that for exclusive locking, you are better off using the \co{spin_lock()}
and \co{spin_unlock()} that LKMM directly models, if for no other reason
that these are much faster.
However, the techniques illustrated in this section can be used for other
purposes, such as emulating reader-writer locking, which LKMM does not yet
model.

\begin{fcvlabel}[ln:docs:litmus:SBemuspin]
\begin{VerbatimN}[commandchars=\%\@\$]
	C C-SB+l-o-o-u+l-o-o-u-X

	{
	}

	P0(int *sl, int *x0, int *x1)
	{
		int r2;
		int r1;

		r2 = xchg_acquire(sl, 1);   %lnlbl@xchgacq$
		WRITE_ONCE(*x0, 1);         %lnlbl@write$
		r1 = READ_ONCE(*x1);        %lnlbl@read$
		smp_store_release(sl, 0);   %lnlbl@release$
	}

	P1(int *sl, int *x0, int *x1)
	{
		int r2;
		int r1;

		r2 = xchg_acquire(sl, 1);
		WRITE_ONCE(*x1, 1);
		r1 = READ_ONCE(*x0);
		smp_store_release(sl, 0);
}

	filter (0:r2=0 /\ 1:r2=0)        %lnlbl@filter$
	exists (0:r1=0 /\ 1:r1=0)
\end{VerbatimN}
\end{fcvlabel}

This litmus test may be found here:

\begin{itemize}
\item \url{https://git.kernel.org/pub/scm/linux/kernel/git/paulmck/perfbook.git/tree/CodeSamples/formal/herd/C-SB+l-o-o-u+l-o-o-u-X.litmus}
\end{itemize}

This test uses two global variables, \qco{x1} and~\qco{x2}, and also
emulates a single global spinlock named~\qco{sl}.
This spinlock is held by whichever process changes the value of~qco{sl}
from~\qco{0} to~\qco{1}, and is released when that process sets~\qco{sl}
back to~\qco{0}.
\begin{fcvref}[ln:docs:litmus:SBemuspin]
\co{P0()}'s lock acquisition is emulated on \clnref{xchgacq} using
\co{xchg_acquire()}, which unconditionally stores the value~\qco{1} to~\qco{sl}
and stores either~\qco{0} or~\qco{1} to~\qco{r2}, depending on whether
the lock acquisition was successful or unsuccessful (due to \qco{sl} already
having the value~\qco{1}), respectively.
\co{P1()} operates in a similar manner.

Rather unconventionally, execution appears to proceed to the critical
section on \clnref{write,read} in either case.
\Clnref{release} then uses an \co{smp_store_release()} to store zero
to~\qco{sl}, thus emulating lock release.

The case where \co{xchg_acquire()} fails to acquire the lock is handled by
the \qco{filter} clause on \clnref{filter}, which tells \co{herd7} to keep
only those executions in which both \qco{0:r2} and \qco{1:r2} are zero,
that is to pay attention only to those executions in which both locks
are actually acquired.
Thus, the bogus executions that would execute the critical sections are
discarded and any effects that they might have had are ignored.
Note well that the \qco{filter} clause keeps those executions for which
its expression is satisfied, that is, for which the expression evaluates
to true.
In other words, the \qco{filter} clause says what to keep, not what to discard.
\end{fcvref}

The result of running this test is as follows:

\begin{fcvlabel}[ln:docs:litmus:SBemuspinout]
\begin{VerbatimN}[commandchars=\%\@\$]
	Test C-SB+l-o-o-u+l-o-o-u-X Allowed
	States 2
	0:r1=0; 1:r1=1;
	0:r1=1; 1:r1=0;
	No
	Witnesses
	Positive: 0 Negative: 2
	Condition exists (0:r1=0 /\ 1:r1=0)
	Observation C-SB+l-o-o-u+l-o-o-u-X Never 0 2   %lnlbl@obs$
	Time C-SB+l-o-o-u+l-o-o-u-X 0.03
\end{VerbatimN}
\end{fcvlabel}

\begin{fcvref}[ln:docs:litmus:SBemuspinout]
The \qco{Never} on \clnref{obs} indicates that this use of \co{xchg_acquire()}
and \co{smp_store_release()} really does correctly emulate locking.
\end{fcvref}

Why doesn't the litmus test take the simpler approach of using a spin loop
to handle failed spinlock acquisitions, like the kernel does?
The key insight behind this litmus test is that spin loops have no effect
on the possible \qco{exists}-clause outcomes of program execution in the absence
of deadlock.
In other words, given a high-quality lock-acquisition primitive in a
deadlock-free program running on high-quality hardware, each lock
acquisition will eventually succeed.
Because \co{herd7} already explores the full state space, the length of
time required to actually acquire the lock does not matter.
After all, \co{herd7} already models all possible durations of the
\co{xchg_acquire()} statements.

Why not just add the \qco{filter} clause to the \qco{exists} clause, thus
avoiding the \qco{filter} clause entirely?
This does work, but is slower.
The reason that the \qco{filter} clause is faster is that (in the common case)
\co{herd7} knows to abandon an execution as soon as the \qco{filter} expression
fails to be satisfied.
In contrast, the \qco{exists} clause is evaluated only at the end of time,
thus requiring \co{herd7} to waste time on bogus executions in which both
critical sections proceed concurrently.
In addition, some LKMM users like the separation of concerns provided by
using the both the \qco{filter} and \qco{exists} clauses.

Readers lacking a pathological interest in odd corner cases should feel
free to skip the remainder of this section.

But what if the litmus test were to temporarily set \qco{0:r2} to a non-zero
value?
Wouldn't that cause \co{herd7} to abandon the execution prematurely
due to an early mismatch of the \qco{filter} clause?

Why not just try it?
\begin{fcvref}[ln:docs:litmus:SBemumod]
\Clnref{x2} of the following modified litmus test
introduces a new global variable~\qco{x2} that is initialized to~\qco{1}.
\Clnref{P1:r2} of \co{P1()} reads that variable into \qco{1:r2} to force
an early mismatch with the \qco{filter} clause.
\Clnref{P1:if} does a known-true "if" condition to avoid and static analysis
that \co{herd7} might do.
Finally the \qco{exists} clause on \clnref{exists} is updated to a condition
that is alway satisfied at the end of the test.

\begin{fcvlabel}[ln:docs:litmus:SBemumod]
\begin{VerbatimN}[commandchars=\%\@\$]
	C C-SB+l-o-o-u+l-o-o-u-X

	{
		x2=1;    %lnlbl@x2$
	}

	P0(int *sl, int *x0, int *x1)
	{
		int r2;
		int r1;

		r2 = xchg_acquire(sl, 1);
		WRITE_ONCE(*x0, 1);
		r1 = READ_ONCE(*x1);
		smp_store_release(sl, 0);
	}

	P1(int *sl, int *x0, int *x1, int *x2)
	{
		int r2;
		int r1;

		r2 = READ_ONCE(*x2);    %lnlbl@P1:r2$
		if (r2)                 %lnlbl@P1:if$
			r2 = xchg_acquire(sl, 1);
		WRITE_ONCE(*x1, 1);
		r1 = READ_ONCE(*x0);
		smp_store_release(sl, 0);
	}

	filter (0:r2=0 /\ 1:r2=0)
	exists (x1=1)                  %lnlbl@exists$
\end{VerbatimN}
\end{fcvlabel}

If the \qco{filter} clause were to check each variable at each point in the
execution, running this litmus test would display no executions because
all executions would be filtered out at \clnref{P1:r2}.
\end{fcvref}
However, the output is instead as follows:

\begin{fcvlabel}[ln:docs:litmus:SBNemumodout]
\begin{VerbatimN}[commandchars=\%\@\$]
	Test C-SB+l-o-o-u+l-o-o-u-X Allowed
	States 1
	x1=1;                 %lnlbl@stat$
	Ok
	Witnesses
	Positive: 2 Negative: 0
	Condition exists (x1=1)
	Observation C-SB+l-o-o-u+l-o-o-u-X Always 2 0
	Time C-SB+l-o-o-u+l-o-o-u-X 0.04
	Hash=080bc508da7f291e122c6de76c0088e3
\end{VerbatimN}
\end{fcvlabel}

\begin{fcvref}[ln:docs:litmus:SBNemumodout]
\Clnref{stat} shows that there is one execution that did not get filtered out,
so the \qco{filter} clause is evaluated only on the last assignment to
the variables that it checks.
In this case, the \qco{filter} clause is a disjunction, so it might be
evaluated twice, once at the final (and only) assignment to \qco{0:r2}
and once at the final assignment to \qco{1:r2}.
\end{fcvref}


\subsubsection{Linked lists}

LKMM can handle linked lists, but only linked lists in which each node
contains nothing except a pointer to the next node in the list.
This is of course quite restrictive, but there is nevertheless quite a
bit that can be done within these confines, as can be seen in the litmus
test at \path{tools/memory-model/litmus-tests/MP+onceassign+derefonce.litmus}:

\begin{fcvlabel}[ln:docs:litmus:MPderef]
\begin{VerbatimN}[commandchars=\%\@\$]
	C MP+onceassign+derefonce

	{
	y=z;     %lnlbl@inity$
	z=0;     %lnlbl@initz$
	}

	P0(int *x, int **y)
	{
		WRITE_ONCE(*x, 1);           %lnlbl@P0:x$
		rcu_assign_pointer(*y, x);   %lnlbl@P0:y$
	}

	P1(int *x, int **y)
	{
		int *r0;
		int r1;

		rcu_read_lock();             %lnlbl@P1:lock$
		r0 = rcu_dereference(*y);    %lnlbl@P1:y$
		r1 = READ_ONCE(*r0);         %lnlbl@P1:r1$
		rcu_read_unlock();           %lnlbl@P1:unlock$
	}

	exists (1:r0=x /\ 1:r1=0)
\end{VerbatimN}
\end{fcvlabel}

\begin{fcvref}[ln:docs:litmus:MPderef]
\Clnref{inity}'s \qco{y=z} may seem odd, given that \qco{z}~has not yet
been initialized.
But \qco{y=z} does not set the value of~\qco{y} to that of~\qco{z}, but instead
sets the value of~\qco{y} to the \emph{address} of~\qco{z}.
\Clnref{inity,initz} therefore create a simple linked list, with \qco{y}
pointing to~\qco{z} and~\qco{z} having a NULL pointer.
A much longer linked list could be created if desired, and circular singly
linked lists can also be created and manipulated.

The \qco{exists} clause works the same way, with the \qco{1:r0=x} comparing
\co{P1()}'s \qco{r0} not to the value of~\qco{x}, but again to its address.
This term of the \qco{exists} clause therefore tests whether
\clnref{P1:y}'s load from~\qco{y} saw the value stored by \clnref{P0:y},
which is in fact what is required in this case.

\co{P0()}'s \clnref{P0:x} initializes "x" to the value~1 then \clnref{P0:y}
links to~\qco{x} from~\qco{y}, replacing~\qco{z}.

\co{P1()}'s \clnref{P1:y} loads a pointer from~\qco{y}, and \clnref{P1:r1}
dereferences that pointer.
The RCU read-side critical section spanning \clnrefrange{P1:lock}{P1:unlock}
is just for show in this example.
Note that the address used for \clnref{P1:r1}'s load depends on
(in this case, ``is exactly the same as'') the value loaded by \clnref{P1:y}.
This is an example of what is called an ``address dependency''.
This particular address dependency extends from the load on \clnref{P1:y} to the
load on \clnref{P1:r1}.
Address dependencies provide a weak form of ordering.
\end{fcvref}

Running this test results in the following:

\begin{VerbatimN}[commandchars=\%\@\$]
	Test MP+onceassign+derefonce Allowed
	States 2
	1:r0=x; 1:r1=1;
	1:r0=z; 1:r1=0;
	No
	Witnesses
	Positive: 0 Negative: 2
	Condition exists (1:r0=x /\ 1:r1=0)
	Observation MP+onceassign+derefonce Never 0 2
	Time MP+onceassign+derefonce 0.00
	Hash=49ef7a741563570102448a256a0c8568
\end{VerbatimN}

The only possible outcomes feature \co{P1()} loading a pointer to~\qco{z}
(which contains zero) on the one hand and \co{P1()} loading a pointer to~\qco{x}
(which contains the value one) on the other.
This should be reassuring because it says that RCU readers cannot see the
old preinitialization values when accessing a newly inserted list node.
This undesirable scenario is flagged by the \qco{exists} clause, and
would occur if \co{P1()} loaded a pointer to~\qco{x}, but obtained
the pre-initialization value of zero after dereferencing that pointer.


\subsubsection{Comments}

Different portions of a litmus test are processed by different parsers,
which has the charming effect of requiring different comment syntax in
different portions of the litmus test.
The C-syntax portions use C-language comments (either \qco{/* */} or \qco{//}),
while the other portions use Ocaml comments \qco{(* *)}.

The following litmus test illustrates the comment style corresponding
to each syntactic unit of the test:

\begin{VerbatimN}[commandchars=\%\@\$]
	C MP+onceassign+derefonce (* A *)

	(* B *)

	{
	y=z; (* C *)
	z=0;
	} // D

	// E

	P0(int *x, int **y) // F
	{
		WRITE_ONCE(*x, 1);  // G
		rcu_assign_pointer(*y, x);
	}

	// H

	P1(int *x, int **y)
	{
		int *r0;
		int r1;

		rcu_read_lock();
		r0 = rcu_dereference(*y);
		r1 = READ_ONCE(*r0);
		rcu_read_unlock();
	}

	// I

	exists (* J *) (1:r0=x /\ (* K *) 1:r1=0) (* L *)
\end{VerbatimN}

In short, use C-language comments in the C code and Ocaml comments in
the rest of the litmus test.

On the other hand, if you prefer C-style comments everywhere, the
C preprocessor is your friend.


\subsubsection{Asynchronous RCU Grace Periods}

The following litmus test is derived from the example show in
\path{Documentation/litmus-tests/rcu/RCU+sync+free.litmus}, but
converted to emulate \co{call_rcu()}:

\begin{fcvlabel}[ln:docs:litmus:RCU]
\begin{VerbatimN}[commandchars=\%\@\$]
	C RCU+sync+free

	{
	int x = 1;      %lnlbl@initx$
	int *y = &x;
	int z = 1;      %lnlbl@initz$
	}

	P0(int *x, int *z, int **y)    %lnlbl@P0:b$
	{
		int *r0;
		int r1;

		rcu_read_lock();
		r0 = rcu_dereference(*y);
		r1 = READ_ONCE(*r0);
		rcu_read_unlock();
	}                              %lnlbl@P0:e$

	P1(int *z, int **y, int *c)    %lnlbl@P1:b$
	{
		rcu_assign_pointer(*y, z);
		smp_store_release(*c, 1); // Emulate call_rcu().
	}                              %lnlbl@P1:e$

	P2(int *x, int *z, int **y, int *c)   %lnlbl@P2:b$
	{
		int r0;

		r0 = smp_load_acquire(*c); // Note call_rcu() request.  %lnlbl@P2:c$
		synchronize_rcu(); // Wait one grace period.  %lnlbl@P2:sync$
		WRITE_ONCE(*x, 0); // Emulate the RCU callback.  %lnlbl@P2:x$
	}                                     %lnlbl@P2:e$

	filter (2:r0=1) (* Reject too-early starts. *)  %lnlbl@filter$
	exists (0:r0=x /\ 0:r1=0)
\end{VerbatimN}
\end{fcvlabel}

\begin{fcvref}[ln:docs:litmus:RCU]
\Clnrefrange{initx}{initz} initialize a linked list headed by~\qco{y}
that initially contains~\qco{x}.
In addition, \qco{z} is pre-initialized to prepare for \co{P1()}, which
will replace \qco{x} with~\qco{z} in this list.

\co{P0()} on \clnrefrange{P0:b}{P0:e} enters an RCU read-side critical
section, loads the list header~\qco{y} and dereferences it, leaving the
node in \qco{0:r0} and the node's value in \qco{0:r1}.

\co{P1()} on \clnrefrange{P1:b}{P1:e} updates the list header to instead
reference~\qco{z}, then emulates \co{call_rcu()} by doing a release store
into~\qco{c}.

\co{P2()} on \clnrefrange{P2:b}{P2:e} emulates the behind-the-scenes effect
of doing a \co{call_rcu()}.
\Clnref{P2:c} first does an acquire load from~\qco{c}, then \clnref{P2:sync}
waits for an RCU grace period to elapse, and finally \clnref{P2:x} emulates
the RCU callback, which in turn emulates a call to \co{kfree()}.

Of course, it is possible for \co{P2()} to start too soon, so that the
value of \qco{2:r0} is zero rather than the required value of~\qco{1}.
The \qco{filter} clause on \clnref{filter} handles this possibility, rejecting
all executions in which \qco{2:r0} is not equal to the value~\qco{1}.
\end{fcvref}


\subsubsection{Performance}

LKMM's exploration of the full state-space can be extremely helpful,
but it does not come for free.
The price is exponential computational complexity in terms of the number
of processes, the average number of statements in each process, and the
total number of stores in the litmus test.

So it is best to start small and then work up.
Where possible, break your code down into small pieces each representing
a core concurrency requirement.

That said, \co{herd7} is quite fast.
On an unprepossessing x86 laptop, it was able to analyze the following
10-process RCU litmus test in about six seconds.

\begin{itemize}
\item \url{https://github.com/paulmckrcu/litmus/blob/master/auto/C-RW-R+RW-R+RW-G+RW-G+RW-G+RW-G+RW-R+RW-R+RW-R+RW-R.litmus}
\end{itemize}

One way to make \co{herd7} run faster is to use the \qco{-speedcheck true}
option.
This option prevents \co{herd7} from generating all possible end states,
instead causing it to focus solely on whether or not the \qco{exists}
clause can be satisfied.
With this option, \co{herd7} evaluates the above litmus test in
about 300 milliseconds, for more than an order of magnitude improvement
in performance.

Larger 16-process litmus tests that would normally consume 15 minutes
of time complete in about 40 seconds with this option.
To be fair, you do get an extra 65,535 states when you leave off the
\qco{-speedcheck true} option.

\begin{itemize}
\item \url{https://github.com/paulmckrcu/litmus/blob/master/auto/C-RW-R+RW-R+RW-G+RW-G+RW-G+RW-G+RW-R+RW-R+RW-R+RW-R+RW-G+RW-G+RW-G+RW-G+RW-R+RW-R.litmus}
\end{itemize}
    
Nevertheless, litmus-test analysis really is of exponential complexity,
whether with or without \qco{-speedcheck true}.
Increasing by just three processes to a 19-process litmus test requires
2~hours and 40~minutes without, and about 8~minutes with \qco{-speedcheck true}.
Each of these results represent roughly an order of magnitude slowdown
compared to the 16-process litmus test.
Again, to be fair, the multi-hour run explores no fewer than 524,287
additional states compared to the shorter one.

\begin{itemize}
\item \url{https://github.com/paulmckrcu/litmus/blob/master/auto/C-RW-R+RW-R+RW-G+RW-G+RW-G+RW-G+RW-R+RW-R+RW-R+RW-R+RW-R+RW-R+RW-G+RW-G+RW-G+RW-G+RW-R+RW-R+RW-R.litmus}
\end{itemize}

If you don't like command-line arguments, you can obtain a similar speedup
by adding a \qco{filter} clause with exactly the same expression as your
"exists" clause.

However, please note that seeing the full set of states can be extremely
helpful when developing and debugging litmus tests.


\subsection{Limitations}

Limitations of the Linux-kernel memory model (LKMM) include:

\begin{enumerate}
  \item	Compiler optimizations are not accurately modeled.
	Of course, the use of \co{READ_ONCE()} and \co{WRITE_ONCE()} limits
	the compiler's ability to optimize, but under some circumstances it
	is possible for the compiler to undermine the memory model.
	For more information, see \path{Documentation/explanation.txt}
	(in particular, the ``THE PROGRAM ORDER RELATION: po AND po-loc''
	and ``A WARNING'' sections).

	Note that this limitation in turn limits LKMM's ability to
	accurately model address, control, and data dependencies.
	For example, if the compiler can deduce the value of some variable
	carrying a dependency, then the compiler can break that dependency
	by substituting a constant of that value.

	Conversely, LKMM will sometimes overestimate the amount of
	reordering compilers and CPUs can carry out, leading it to miss
	some pretty obvious cases of ordering.  A simple example is:

\begin{VerbatimU}
		r1 = READ_ONCE(x);
		if (r1 == 0)
			smp_mb();
		WRITE_ONCE(y, 1);
\end{VerbatimU}

	The \co{WRITE_ONCE()} does not depend on the \co{READ_ONCE()}, and as a
	result, LKMM does not claim ordering.
	However, even though no dependency is present, the \co{WRITE_ONCE()}
	will not be executed before the \co{READ_ONCE()}.
	There are two reasons for this:

	\begin{quote}
                The presence of the \co{smp_mb()} in one of the branches
                prevents the compiler from moving the \co{WRITE_ONCE()}
                up before the \qco{if} statement, since the compiler has
                to assume that \co{r1} will sometimes be \co{0} (but see the
                comment below);

                CPUs do not execute stores before po-earlier conditional
                branches, even in cases where the store occurs after the
                two arms of the branch have recombined.
	\end{quote}

	It is clear that it is not dangerous in the slightest for LKMM to
	make weaker guarantees than architectures.
	In fact, it is desirable, as it gives compilers room for making
	optimizations.
	For instance, suppose that a 0~value in \co{r1} would trigger undefined
	behavior elsewhere.
	Then a clever compiler might deduce that \co{r1} can never be~0 in
	the \co{if} condition.
	As a result, said clever compiler might deem it safe to optimize
	away the \co{smp_mb()}, eliminating the branch and any ordering
	an architecture would guarantee otherwise.

  \item	Multiple access sizes for a single variable are not supported,
	and neither are misaligned or partially overlapping accesses.

  \item	Exceptions and interrupts are not modeled.
	In some cases, this limitation can be overcome by modeling the
	interrupt or exception with an additional process.

  \item	I/O such as MMIO or DMA is not supported.

  \item	Self-modifying code (such as that found in the kernel's
	alternatives mechanism, function tracer, Berkeley Packet Filter
	JIT compiler, and module loader) is not supported.

  \item	Complete modeling of all variants of atomic read-modify-write
	operations, locking primitives, and RCU is not provided.
	For example, \co{call_rcu()} and \co{rcu_barrier()} are not supported.
	However, a substantial amount of support is provided for these
	operations, as shown in the linux-kernel.def file.

	Here are specific limitations:

	\begin{enumerate}
	\item	When \co{rcu_assign_pointer()} is passed \co{NULL}, the Linux
		kernel provides no ordering, but LKMM models this
		case as a store release.

	\item	The \qco{unless} RMW operations are not currently modeled:
		\co{atomic_long_add_unless()}, \co{atomic_inc_unless_negative()},
		and \co{atomic_dec_unless_positive()}.
		These can be emulated in litmus tests, for example, by
		using \co{atomic_cmpxchg()}.

		One exception of this limitation is \co{atomic_add_unless()},
		which is provided directly by \co{herd7} (so no corresponding
		definition in \path{linux-kernel.def}).
		\co{atomic_add_unless()} is modeled by \co{herd7} therefore
		it can be used in litmus tests.

	\item	The \co{call_rcu()} function is not modeled.
		As was shown above, it can be emulated in litmus tests by
		adding another process that invokes \co{synchronize_rcu()}
		and the body of the callback function, with (for example)
		a release-acquire from the site of the emulated \co{call_rcu()}
		to the beginning of the additional process.

	\item	The \co{rcu_barrier()} function is not modeled.
		It can be emulated in litmus tests emulating \co{call_rcu()} via
		(for example) a release-acquire from the end of each
		additional \co{call_rcu()} process to the site of the
		emulated \co{rcu_barrier()}.

	\item	Reader-writer locking is not modeled.
		It can be emulated in litmus tests using atomic
		read-modify-write operations.
	\end{enumerate}
\end{enumerate}

The fragment of the C language supported by these litmus tests is quite
limited and in some ways non-standard:

\begin{enumerate}
\item	There is no automatic C-preprocessor pass.
	ou can of course run it manually, if you choose.

\item	There is no way to create functions other than the \co{Pn()} functions
	that model the concurrent processes.

\item	The \co{Pn()} functions' formal parameters must be pointers to the
	global shared variables.
	Nothing can be passed by value into these functions.

\item	The only functions that can be invoked are those built directly
	into \co{herd7} or that are defined in the \path{linux-kernel.def} file.

\item	The \qco{switch}, \qco{do}, \qco{for}, \qco{while}, and \qco{goto}
	C statements are not supported.
	The \qco{switch} statement can be emulated by the \qco{if} statement.
	The \qco{do}, \qco{for}, and \qco{while} statements can
	often be emulated by manually unrolling the loop, or perhaps by
	enlisting the aid of the C preprocessor to minimize the resulting
	code duplication.
	Some uses of \qco{goto} can be emulated by \qco{if},
	and some others by unrolling.

\item	Although you can use a wide variety of types in litmus-test
	variable declarations, and especially in global-variable
	declarations, the \co{herd7} tool understands only int and
	pointer types.
	There is no support for floating-point types, enumerations,
	characters, strings, arrays, or structures.

\item	Parsing of variable declarations is very loose, with almost no
	type checking.

\item	Initializers differ from their C-language counterparts.
	For example, when an initializer contains the name of a shared
	variable, that name denotes a pointer to that variable, not
	the current value of that variable.
	For example, \qco{int x = y} is interpreted the way \qco{int x = &y}
	would be in C.

\item	Dynamic memory allocation is not supported, although this can
	be worked around in some cases by supplying multiple statically
	allocated variables.
\end{enumerate}

Some of these limitations may be overcome in the future, but others are
more likely to be addressed by incorporating the Linux-kernel memory model
into other tools.

Finally, please note that LKMM is subject to change as hardware, use cases,
and compilers evolve.
