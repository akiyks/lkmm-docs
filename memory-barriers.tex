\chapter{Linux Kermel Memory Barriers}

\begin{Note}
Pretty-printed version of \path{Documentation/memory-barriers.txt}
\end{Note}

\begin{Note}
Authors:

\begin{itemize}[nosep]
  \item David Howells <dhowells@redhat.com>
  \item Paul E. McKenney <paulmck@kernel.org>
  \item Will Deacon <will.deacon@arm.com>
  \item Peter Zijlstra <peterz@infradead.org>
\end{itemize}
\end{Note}

\section{Disclaimer}

This document is not a specification; it is intentionally (for the sake of
brevity) and unintentionally (due to being human) incomplete.
This document is meant as a guide to using the various memory barriers
provided by Linux, but in case of any doubt (and there are many) please ask.
Some doubts may be resolved by referring to the formal memory consistency
model and related documentation at \path{tools/memory-model/}.
Nevertheless, even this memory model should be viewed as the collective
opinion of its maintainers rather than as an infallible oracle.

To repeat, this document is not a specification of what Linux expects from
hardware.

The purpose of this document is twofold:

\begin{enumerate}
  \item to specify the minimum functionality that one can rely on for any
        particular barrier, and
  \item to provide a guide as to how to use the barriers that are available.
\end{enumerate}

Note that an architecture can provide more than the minimum requirement
for any particular barrier, but if the architecture provides less than
that, that architecture is incorrect.

Note also that it is possible that a barrier may be a no-op for an
architecture because the way that arch works renders an explicit barrier
unnecessary in that case.


\section{Abstract memory access model}

Consider the following abstract model of the system:

\begin{VerbatimU}
		            :                :
		            :                :
		            :                :
		+-------+   :   +--------+   :   +-------+
		|       |   :   |        |   :   |       |
		|       |   :   |        |   :   |       |
		| CPU 1 |<----->| Memory |<----->| CPU 2 |
		|       |   :   |        |   :   |       |
		|       |   :   |        |   :   |       |
		+-------+   :   +--------+   :   +-------+
		    ^       :       ^        :       ^
		    |       :       |        :       |
		    |       :       |        :       |
		    |       :       v        :       |
		    |       :   +--------+   :       |
		    |       :   |        |   :       |
		    |       :   |        |   :       |
		    +---------->| Device |<----------+
		            :   |        |   :
		            :   |        |   :
		            :   +--------+   :
		            :                :
\end{VerbatimU}

Each CPU executes a program that generates memory access operations.
In the abstract CPU, memory operation ordering is very relaxed, and a CPU may
actually perform the memory operations in any order it likes, provided
program causality appears to be maintained.
Similarly, the compiler may also arrange the instructions it emits in any
order it likes, provided it doesn't affect the apparent operation of the
program.

So in the above diagram, the effects of the memory operations performed by a
CPU are perceived by the rest of the system as the operations cross the
interface between the CPU and rest of the system (the dotted lines).


For example, consider the following sequence of events:

\begin{VerbatimU}
	CPU 1           CPU 2
	===============	===============
	{ A == 1; B == 2 }
	A = 3;          x = B;
	B = 4;          y = A;
\end{VerbatimU}

The set of accesses as seen by the memory system in the middle can be arranged
in 24 different combinations:

\begin{VerbatimU}
	STORE A=3, STORE B=4,   y=LOAD A->3, x=LOAD B->4
	STORE A=3, STORE B=4,   x=LOAD B->4, y=LOAD A->3
	STORE A=3, y=LOAD A->3, STORE B=4,   x=LOAD B->4
	STORE A=3, y=LOAD A->3, x=LOAD B->2, STORE B=4
	STORE A=3, x=LOAD B->2, STORE B=4,   y=LOAD A->3
	STORE A=3, x=LOAD B->2, y=LOAD A->3, STORE B=4
	STORE B=4, STORE A=3,   y=LOAD A->3, x=LOAD B->4
	STORE B=4, ...
	...
\end{VerbatimU}

\noindent%
and can thus result in four different combinations of values:

\begin{VerbatimU}
	x == 2, y == 1
	x == 2, y == 3
	x == 4, y == 1
	x == 4, y == 3
\end{VerbatimU}

Furthermore, the stores committed by a CPU to the memory system may not be
perceived by the loads made by another CPU in the same order as the stores were
committed.


As a further example, consider this sequence of events:

\begin{VerbatimU}
	CPU 1		CPU 2
	===============	===============
	{ A == 1, B == 2, C == 3, P == &A, Q == &C }
	B = 4;          Q = P;
	P = &B;         D = *Q;
\end{VerbatimU}

There is an obvious address dependency here, as the value loaded into~\co{D} depends
on the address retrieved from~\co{P} by CPU~2.
At the end of the sequence, any of the following results are possible:

\begin{VerbatimU}
	(Q == &A) and (D == 1)
	(Q == &B) and (D == 2)
	(Q == &B) and (D == 4)
\end{VerbatimU}

Note that CPU~2 will never try and load~\co{C} into~\co{D} because the CPU will load~\co{P}
into~\co{Q} before issuing the load of~\co{*Q}.


\subsection{Device operations}

Some devices present their control interfaces as collections of memory
locations, but the order in which the control registers are accessed is very
important.
For instance, imagine an ethernet card with a set of internal registers
that are accessed through an address port register~\co{A} and a data port
register~\co{D}.
To read internal register~5, the following code might then be used:

\begin{VerbatimU}
	*A = 5;
	x = *D;
\end{VerbatimU}

\noindent%
but this might show up as either of the following two sequences:

\begin{VerbatimU}
	STORE *A = 5, x = LOAD *D
	x = LOAD *D, STORE *A = 5
\end{VerbatimU}

\noindent%
the second of which will almost certainly result in a malfunction, since it set
the address \emph{after} attempting to read the register.


\subsection{Guarantees}

There are some minimal guarantees that may be expected of a CPU:

\begin{itemize}
 \item On any given CPU, dependent memory accesses will be issued in order, with
       respect to itself.
       This means that for:
       \begin{VerbatimU}
	Q = READ_ONCE(P); D = READ_ONCE(*Q);
       \end{VerbatimU}

       the CPU will issue the following memory operations:

       \begin{VerbatimU}
	Q = LOAD P, D = LOAD *Q
       \end{VerbatimU}

       and always in that order.
       However, on DEC Alpha, \co{READ_ONCE()} also
       emits a memory-barrier instruction, so that a DEC Alpha CPU will
       instead issue the following memory operations (\co{<MB>} stands for
       a memory-barrier instruction):

       \begin{VerbatimU}
	Q = LOAD P, <MB>, D = LOAD *Q, <MB>
       \end{VerbatimU}

       Whether on DEC Alpha or not, the \co{READ_ONCE()} also prevents compiler
       mischief.

 \item Overlapping loads and stores within a particular CPU will appear to be
       ordered within that CPU.  This means that for:

       \begin{VerbatimU}
	a = READ_ONCE(*X); WRITE_ONCE(*X, b);
       \end{VerbatimU}

       the CPU will only issue the following sequence of memory operations:

       \begin{VerbatimU}
	a = LOAD *X, STORE *X = b
       \end{VerbatimU}

       And for:

       \begin{VerbatimU}
	WRITE_ONCE(*X, c); d = READ_ONCE(*X);
       \end{VerbatimU}

       the CPU will only issue:

       \begin{VerbatimU}
	STORE *X = c, d = LOAD *X
       \end{VerbatimU}

       (Loads and stores overlap if they are targeted at overlapping pieces of
       memory).
\end{itemize}

And there are a number of things that \emph{must} or \emph{must not} be assumed:

\begin{itemize}
 \item It \emph{must not} be assumed that the compiler will do what you want
       with memory references that are not protected by \co{READ_ONCE()} and
       \co{WRITE_ONCE()}.
       Without them, the compiler is within its rights to do all sorts of
       ``creative'' transformations, which are covered in
       \cref{sec:Compiler barrier}.

 \item It \emph{must not} be assumed that independent loads and stores will be issued
       in the order given.  This means that for:

       \begin{VerbatimU}
	X = *A; Y = *B; *D = Z;
       \end{VerbatimU}

       we may get any of the following sequences:

       \begin{VerbatimU}
	X = LOAD *A,  Y = LOAD *B,  STORE *D = Z
	X = LOAD *A,  STORE *D = Z, Y = LOAD *B
	Y = LOAD *B,  X = LOAD *A,  STORE *D = Z
	Y = LOAD *B,  STORE *D = Z, X = LOAD *A
	STORE *D = Z, X = LOAD *A,  Y = LOAD *B
	STORE *D = Z, Y = LOAD *B,  X = LOAD *A
       \end{VerbatimU}

 \item It \emph{must} be assumed that overlapping memory accesses may be
       merged or discarded.
       This means that for:

       \begin{VerbatimU}
	X = *A; Y = *(A + 4);
       \end{VerbatimU}

       we may get any one of the following sequences:

       \begin{VerbatimU}
	X = LOAD *A; Y = LOAD *(A + 4);
	Y = LOAD *(A + 4); X = LOAD *A;
	{X, Y} = LOAD {*A, *(A + 4) };
       \end{VerbatimU}

       And for:

       \begin{VerbatimU}
	*A = X; *(A + 4) = Y;
       \end{VerbatimU}

       we may get any of:

       \begin{VerbatimU}
	STORE *A = X; STORE *(A + 4) = Y;
	STORE *(A + 4) = Y; STORE *A = X;
	STORE {*A, *(A + 4) } = {X, Y};
       \end{VerbatimU}
\end{itemize}

And there are anti-guarantees:

\begin{itemize}
 \item These guarantees do not apply to bitfields, because compilers often
       generate code to modify these using non-atomic read-modify-write
       sequences.
       Do not attempt to use bitfields to synchronize parallel algorithms.

 \item Even in cases where bitfields are protected by locks, all fields
       in a given bitfield must be protected by one lock.
       If two fields in a given bitfield are protected by different locks,
       the compiler's non-atomic read-modify-write sequences can cause
       an update to one field to corrupt the value of an adjacent field.

 \item These guarantees apply only to properly aligned and sized scalar
       variables.
       ``Properly sized'' currently means variables that are the same size
       as \qco{char}, \qco{short}, \qco{int} and \qco{long}.
       ``Properly aligned'' means the natural alignment, thus no
       constraints for \qco{char}, two-byte alignment for \qco{short},
       four-byte alignment for \qco{int}, and either four-byte or eight-byte
       alignment for \qco{long}, on 32-bit and 64-bit systems, respectively.
       Note that these guarantees were introduced into the C11 standard,
       so beware when using older pre-C11 compilers (for example, gcc 4.6).
       The portion of the standard containing this guarantee is
       Section 3.14, which defines ``memory location'' as follows:

       \begin{description}[style=nextline]
	\item [location]
		either an object of scalar type, or a maximal sequence
		of adjacent bit-fields all having nonzero width

		NOTE 1: Two threads of execution can update and access
		separate memory locations without interfering with
		each other.

		NOTE 2: A bit-field and an adjacent non-bit-field member
		are in separate memory locations. The same applies
		to two bit-fields, if one is declared inside a nested
		structure declaration and the other is not, or if the two
		are separated by a zero-length bit-field declaration,
		or if they are separated by a non-bit-field member
		declaration. It is not safe to concurrently update two
		bit-fields in the same structure if all members declared
		between them are also bit-fields, no matter what the
		sizes of those intervening bit-fields happen to be.
       \end{description}
\end{itemize}

\section{What are memory barriers?}

As can be seen above, independent memory operations are effectively performed
in random order, but this can be a problem for CPU-CPU interaction and for
I/O\@.
What is required is some way of intervening to instruct the compiler and the
CPU to restrict the order.

Memory barriers are such interventions.
They impose a perceived partial ordering over the memory operations on
either side of the barrier.

Such enforcement is important because the CPUs and other devices in a system
can use a variety of tricks to improve performance, including reordering,
deferral and combination of memory operations; speculative loads; speculative
branch prediction and various types of caching.
Memory barriers are used to override or suppress these tricks, allowing
the code to sanely control the interaction of multiple CPUs and/or devices.

\subsection{Varieties of memory barrier}

Memory barriers come in four basic varieties:

\begin{description}[style=nextline]
 \item[Write (or store) memory barriers:]
     A write memory barrier gives a guarantee that all the STORE operations
     specified before the barrier will appear to happen before all the STORE
     operations specified after the barrier with respect to the other
     components of the system.

     A write barrier is a partial ordering on stores only; it is not required
     to have any effect on loads.

     A CPU can be viewed as committing a sequence of store operations to the
     memory system as time progresses.
     All stores \emph{before} a write barrier will occur \emph{before} all
     the stores after the write barrier.

     \begin{Note}
     Note that write barriers should normally be paired with read or
     address-dependency barriers; see \cref{sec:SMP barrier pairing}.
     \end{Note}


 \item[Address-dependency barriers (HISTORICAL):]
     An address-dependency barrier is a weaker form of read barrier.

     \begin{Note}
     This item is marked as HISTORICAL\@:
     It covers the long-obsolete \co{smp_read_barrier_depends()} macro,
     the semantics of which are now implicit in all marked accesses.
     For more up-to-date information, including how compiler transformations
     can sometimes break address dependencies, see
     \cref{sec:rcu:Proper care and feeding of return values from rcu_dereference()}.
     \end{Note}

     In the case where two loads are performed such that the second depends
     on the result of the first (eg., the first load retrieves the address
     to which the second load will be directed), an address-dependency
     barrier would be required to make sure that the target of the second
     load is updated after the address obtained by the first load is
     accessed.

     An address-dependency barrier is a partial ordering on interdependent
     loads only; it is not required to have any effect on stores, independent
     loads or overlapping loads.

     As mentioned above in ``Write memory barriers'',
     the other CPUs in the system can be viewed as
     committing sequences of stores to the memory system that the CPU being
     considered can then perceive.
     An address-dependency barrier issued by the CPU under consideration
     guarantees that for any load preceding it, if that load touches one of
     a sequence of stores from another CPU, then by the time the barrier
     completes, the effects of all the stores prior to that touched by the
     load will be perceptible to any loads issued after the
     address-dependency barrier.

     See \cref{sec:Examples of memory barrier sequences} subsection for diagrams
     showing the ordering constraints.

     \begin{Note}
     Note that the first load really has to have an \emph{address}
     dependency and not a control dependency.
     If the address for the second load is dependent on the first load,
     but the dependency is through a conditional rather than actually
     loading the address itself, then it's a \emph{control} dependency and
     a full read barrier or better is required.
     See \cref{sec:Control dependencies} for more information.
     \end{Note}

     \begin{Note}
     Note that address-dependency barriers should normally be paired with
     write barriers; see \cref{sec:SMP barrier pairing}.
     \end{Note}

     \begin{Note}
     Kernel release v5.9 removed kernel APIs for explicit
     address-dependency barriers.
     Nowadays, APIs for marking loads from shared variables such as
     \co{READ_ONCE()} and \co{rcu_dereference()} provide implicit
     address-dependency barriers.
     \end{Note}

 \item[Read (or load) memory barriers:]
     A read barrier is an address-dependency barrier plus a guarantee that all
     the LOAD operations specified before the barrier will appear to happen
     before all the LOAD operations specified after the barrier with respect to
     the other components of the system.

     A read barrier is a partial ordering on loads only; it is not required to
     have any effect on stores.

     Read memory barriers imply address-dependency barriers, and so can
     substitute for them.

     \begin{Note}
     Note that read barriers should normally be paired with write barriers;
     see \cref{sec:SMP barrier pairing}.
     \end{Note}

 \item[General memory barriers:]
     A general memory barrier gives a guarantee that all the LOAD and STORE
     operations specified before the barrier will appear to happen before all
     the LOAD and STORE operations specified after the barrier with respect to
     the other components of the system.

     A general memory barrier is a partial ordering over both loads and stores.

     General memory barriers imply both read and write memory barriers, and so
     can substitute for either.
\end{description}

And a couple of implicit varieties:

\begin{description}[style=nextline]
 \item[ACQUIRE operations:]
     This acts as a one-way permeable barrier.
     It guarantees that all memory operations after the ACQUIRE operation
     will appear to happen after the ACQUIRE operation with respect to the
     other components of the system.
     ACQUIRE operations include LOCK operations and both \co{smp_load_acquire()}
     and \co{smp_cond_load_acquire()} operations.

     Memory operations that occur before an ACQUIRE operation may appear to
     happen after it completes.

     An ACQUIRE operation should almost always be paired with a RELEASE
     operation.


 \item[RELEASE operations:]
     This also acts as a one-way permeable barrier.
     It guarantees that all memory operations before the RELEASE operation
     will appear to happen before the RELEASE operation with respect to
     the other components of the system.
     RELEASE operations include UNLOCK operations and \co{smp_store_release()}
     operations.

     Memory operations that occur after a RELEASE operation may appear to
     happen before it completes.

     The use of ACQUIRE and RELEASE operations generally precludes the need
     for other sorts of memory barrier.
     In addition, a RELEASE+ACQUIRE pair is \emph{not} guaranteed to act
     as a full memory barrier.
     However, after an ACQUIRE on a given variable, all memory accesses
     preceding any prior RELEASE on that same variable are guaranteed to
     be visible.
     In other words, within a given variable's critical section, all
     accesses of all previous critical sections for that variable are
     guaranteed to have completed.

     This means that ACQUIRE acts as a minimal ``acquire'' operation and
     RELEASE acts as a minimal ``release'' operation.
\end{description}

A subset of the atomic operations described in \path{atomic_t.txt} have
ACQUIRE and RELEASE variants in addition to fully-ordered and relaxed
(no barrier semantics) definitions.
For compound atomics performing both a load and a store, ACQUIRE semantics
apply only to the load and RELEASE semantics apply only to the store portion
of the operation.

Memory barriers are only required where there's a possibility of interaction
between two CPUs or between a CPU and a device.
If it can be guaranteed that there won't be any such interaction in any
particular piece of code, then memory barriers are unnecessary in that
piece of code.


Note that these are the \emph{minimum} guarantees.
Different architectures may give more substantial guarantees, but they
may \emph{not} be relied upon outside of arch specific code.


\subsection{What may not be asssumed about memory barriers?}

There are certain things that the Linux kernel memory barriers do not guarantee:

\begin{itemize}
 \item
     There is no guarantee that any of the memory accesses specified before a
     memory barrier will be \emph{complete} by the completion of a memory
     barrier instruction; the barrier can be considered to draw a line in
     that CPU's access queue that accesses of the appropriate type may not
     cross.

 \item
     There is no guarantee that issuing a memory barrier on one CPU will have
     any direct effect on another CPU or any other hardware in the system.
     The indirect effect will be the order in which the second CPU sees the
     effects of the first CPU's accesses occur, but see the next point:

 \item
     There is no guarantee that a CPU will see the correct order of effects
     from a second CPU's accesses, even \emph{if} the second CPU uses a memory
     barrier, unless the first CPU \emph{also} uses a matching memory barrier
     (see \cref{sec:SMP barrier pairing}).

 \item
     There is no guarantee that some intervening piece of off-the-CPU
     hardware\footnote{
	For information on bus mastering DMA and coherency please read:
	\path{Documentation/driver-api/pci/pci.rst},
	\path{Documentation/core-api/dma-api-howto.rst}, and
	\path{Documentation/core-api/dma-api.rst}.}
     will not reorder the memory accesses.
     CPU cache coherency mechanisms should propagate the indirect effects
     of a memory barrier between CPUs, but might not do so in order.
\end{itemize}

\subsection{Address-dependency barriers (HISTORICAL)}

\begin{Note}
This section is marked as HISTORICAL\@:
It covers the long-obsolete \co{smp_read_barrier_depends()} macro, the
semantics of which are now implicit in all marked accesses.
For more up-to-date information, including how compiler transformations
can sometimes break address dependencies, see
\cref{sec:rcu:Proper care and feeding of return values from rcu_dereference()}.
\end{Note}

As of v4.15 of the Linux kernel, an \co{smp_mb()} was added to
\co{READ_ONCE()} for DEC Alpha, which means that about the only people who
need to pay attention to this section are those working on DEC Alpha
architecture-specific code and those working on \co{READ_ONCE()} itself.
For those who need it, and for those who are interested in the history,
here is the story of address-dependency barriers.

\begin{Note}
While address dependencies are observed in both load-to-load and
load-to-store relations, address-dependency barriers are not necessary
for load-to-store situations.
\end{Note}

The requirement of address-dependency barriers is a little subtle, and
it's not always obvious that they're needed.
To illustrate, consider the following sequence of events:

\begin{VerbatimU}
	CPU 1                 CPU 2
	===============	      ===============
	{ A == 1, B == 2, C == 3, P == &A, Q == &C }
	B = 4;
	<write barrier>
	WRITE_ONCE(P, &B);
	                      Q = READ_ONCE_OLD(P);
	                      D = *Q;
\end{VerbatimU}

\begin{Note}
\co{READ_ONCE_OLD()} corresponds to \co{READ_ONCE()} of
pre-4.15 kernel, which doesn't imply an address-dependency barrier.
\end{Note}

There's a clear address dependency here, and it would seem that by the end of
the sequence, \co{Q} must be either~\co{&A} or~\co{&B}, and that:

\begin{VerbatimU}
	(Q == &A) implies (D == 1)
	(Q == &B) implies (D == 4)
\end{VerbatimU}

But!
CPU~2's perception of~\co{P} may be updated \emph{before} its perception of~\co{B},
thus leading to the following situation:

\begin{VerbatimU}
	(Q == &B) and (D == 2) ????
\end{VerbatimU}

While this may seem like a failure of coherency or causality maintenance, it
isn't, and this behaviour can be observed on certain real CPUs (such as the DEC
Alpha).

To deal with this, \co{READ_ONCE()} provides an implicit address-dependency
barrier since kernel release v4.15
(\qco{addr-dep} stands for ``address-dependency''):

\begin{VerbatimU}
	CPU 1                 CPU 2
	===============	      ===============
	{ A == 1, B == 2, C == 3, P == &A, Q == &C }
	B = 4;
	<write barrier>
	WRITE_ONCE(P, &B);
	                      Q = READ_ONCE(P);
	                      <implicit addr-dep barrier>
	                      D = *Q;
\end{VerbatimU}

This enforces the occurrence of one of the two implications, and prevents the
third possibility from arising.


\begin{Note}
Note that this extremely counterintuitive situation arises most easily on
machines with split caches, so that, for example, one cache bank processes
even-numbered cache lines and the other bank processes odd-numbered cache
lines.
The pointer~\co{P} might be stored in an odd-numbered cache line, and the
variable~\co{B} might be stored in an even-numbered cache line.
Then, if the even-numbered bank of the reading CPU's cache is extremely
busy while the odd-numbered bank is idle, one can see the new value of
the pointer~\co{P} (\co{&B}), but the old value of the variable~\co{B} (2).
\end{Note}


An address-dependency barrier is not required to order dependent writes
because the CPUs that the Linux kernel supports don't do writes until they
are certain (1) that the write will actually happen, (2) of the location of
the write, and (3) of the value to be written.
But please carefully read \cref{sec:Control dependencies} and
\cref{sec:rcu:Proper care and feeding of return values from rcu_dereference()}:
The compiler can and does break dependencies in a great many highly
creative ways.

\begin{VerbatimU}
	CPU 1                 CPU 2
	===============	      ===============
	{ A == 1, B == 2, C = 3, P == &A, Q == &C }
	B = 4;
	<write barrier>
	WRITE_ONCE(P, &B);
	                      Q = READ_ONCE_OLD(P);
	                      WRITE_ONCE(*Q, 5);
\end{VerbatimU}

Therefore, no address-dependency barrier is required to order the read
into~\co{Q} with the store into~\co{*Q}.
In other words, this outcome is prohibited,
even without an implicit address-dependency barrier of modern \co{READ_ONCE()}:

\begin{VerbatimU}
	(Q == &B) && (B == 4)
\end{VerbatimU}

Please note that this pattern should be rare.
After all, the whole point of dependency ordering is to \emph{prevent}
writes to the data structure, along with the expensive cache misses
associated with those writes.
This pattern can be used to record rare error conditions and the like,
and the CPUs' naturally occurring ordering prevents such records from
being lost.


Note well that the ordering provided by an address dependency is local to
the CPU containing it.
See \cref{sec:Multicopy atomicity} for more information.


The address-dependency barrier is very important to the RCU system,
for example.
See \co{rcu_assign_pointer()} and \co{rcu_dereference()} in
\path{include/linux/rcupdate.h}.
This permits the current target of an RCU'd pointer to be replaced with
a new modified target, without the replacement target appearing to be
incompletely initialised.


\subsection{Control dependencies}
\label{sec:Control dependencies}

Control dependencies can be a bit tricky because current compilers do
not understand them.
The purpose of this section is to help you prevent the compiler's ignorance
from breaking your code.

A load-load control dependency requires a full read memory barrier, not
simply an (implicit) address-dependency barrier to make it work correctly.
Consider the following bit of code:

\begin{VerbatimU}
	q = READ_ONCE(a);
	<implicit address-dependency barrier>
	if (q) {
		/* BUG: No address dependency!!! */
		p = READ_ONCE(b);
	}
\end{VerbatimU}

This will not have the desired effect because there is no actual address
dependency, but rather a control dependency that the CPU may short-circuit
by attempting to predict the outcome in advance, so that other CPUs see
the load from~\qco{b} as having happened before the load from~\qco{a}.
In such a case what's actually required is:

\begin{VerbatimU}
	q = READ_ONCE(a);
	if (q) {
		<read barrier>
		p = READ_ONCE(b);
	}
\end{VerbatimU}

However, stores are not speculated.
This means that ordering \emph{is} provided for load-store control
dependencies, as in the following example:

\begin{VerbatimU}
	q = READ_ONCE(a);
	if (q) {
		WRITE_ONCE(b, 1);
	}
\end{VerbatimU}

Control dependencies pair normally with other types of barriers.
That said, please note that neither \co{READ_ONCE()} nor \co{WRITE_ONCE()}
are optional!
Without the \co{READ_ONCE()}, the compiler might combine the
load from~\qco{a} with other loads from~\qco{a}.  Without the \co{WRITE_ONCE()},
the compiler might combine the store to~\qco{b} with other stores to~\qco{b}.
Either can result in highly counterintuitive effects on ordering.

Worse yet, if the compiler is able to prove (say) that the value of
variable~\qco{a} is always non-zero, it would be well within its rights
to optimize the original example by eliminating the \qco{if} statement
as follows:

\begin{VerbatimU}
	q = a;
	b = 1;  /* BUG: Compiler and CPU can both reorder!!! */
\end{VerbatimU}

So don't leave out the \co{READ_ONCE()}.

It is tempting to try to enforce ordering on identical stores on both
branches of the \qco{if} statement as follows:

\begin{VerbatimU}
	q = READ_ONCE(a);
	if (q) {
		barrier();
		WRITE_ONCE(b, 1);
		do_something();
	} else {
		barrier();
		WRITE_ONCE(b, 1);
		do_something_else();
	}
\end{VerbatimU}

Unfortunately, current compilers will transform this as follows at high
optimization levels:

\begin{VerbatimU}
	q = READ_ONCE(a);
	barrier();
	WRITE_ONCE(b, 1);  /* BUG: No ordering vs. load from a!!! */
	if (q) {
		/* WRITE_ONCE(b, 1); -- moved up, BUG!!! */
		do_something();
	} else {
		/* WRITE_ONCE(b, 1); -- moved up, BUG!!! */
		do_something_else();
	}
\end{VerbatimU}

Now there is no conditional between the load from~\qco{a} and the store
to~\qco{b}, which means that the CPU is within its rights to reorder them:
The conditional is absolutely required, and must be present in the
assembly code even after all compiler optimizations have been applied.
Therefore, if you need ordering in this example, you need explicit
memory barriers, for example, \co{smp_store_release()}:

\begin{VerbatimU}
	q = READ_ONCE(a);
	if (q) {
		smp_store_release(&b, 1);
		do_something();
	} else {
		smp_store_release(&b, 1);
		do_something_else();
	}
\end{VerbatimU}

In contrast, without explicit memory barriers, two-legged-if control
ordering is guaranteed only when the stores differ, for example:

\begin{VerbatimU}
	q = READ_ONCE(a);
	if (q) {
		WRITE_ONCE(b, 1);
		do_something();
	} else {
		WRITE_ONCE(b, 2);
		do_something_else();
	}
\end{VerbatimU}

The initial \co{READ_ONCE()} is still required to prevent the compiler from
proving the value of~\qco{a}.

In addition, you need to be careful what you do with the local variable~\qco{q},
otherwise the compiler might be able to guess the value and again remove
the needed conditional.  For example:

\begin{VerbatimU}
	q = READ_ONCE(a);
	if (q % MAX) {
		WRITE_ONCE(b, 1);
		do_something();
	} else {
		WRITE_ONCE(b, 2);
		do_something_else();
	}
\end{VerbatimU}

If \co{MAX} is defined to be~1, then the compiler knows that \co{(q \% MAX)} is
equal to zero, in which case the compiler is within its rights to
transform the above code into the following:

\begin{VerbatimU}
	q = READ_ONCE(a);
	WRITE_ONCE(b, 2);
	do_something_else();
\end{VerbatimU}

Given this transformation, the CPU is not required to respect the ordering
between the load from variable~\qco{a} and the store to variable~\qco{b}.
It is tempting to add a \co{barrier()}, but this does not help.
The conditional is gone, and the barrier won't bring it back.
Therefore, if you are relying on this ordering, you should make sure that
MAX is greater than one, perhaps as follows:

\begin{VerbatimU}
	q = READ_ONCE(a);
	BUILD_BUG_ON(MAX <= 1); /* Order load from a with store to b. */
	if (q % MAX) {
		WRITE_ONCE(b, 1);
		do_something();
	} else {
		WRITE_ONCE(b, 2);
		do_something_else();
	}
\end{VerbatimU}

Please note once again that the stores to~\qco{b} differ.
If they were identical, as noted earlier, the compiler could pull this
store outside of the \qco{if} statement.

You must also be careful not to rely too much on boolean short-circuit
evaluation.
Consider this example:

\begin{VerbatimU}
	q = READ_ONCE(a);
	if (q || 1 > 0)
		WRITE_ONCE(b, 1);
\end{VerbatimU}

Because the first condition cannot fault and the second condition is
always true, the compiler can transform this example as following,
defeating control dependency:

\begin{VerbatimU}
	q = READ_ONCE(a);
	WRITE_ONCE(b, 1);
\end{VerbatimU}

This example underscores the need to ensure that the compiler cannot
out-guess your code.  More generally, although \co{READ_ONCE()} does force
the compiler to actually emit code for a given load, it does not force
the compiler to use the results.

In addition, control dependencies apply only to the then-clause and
else-clause of the if-statement in question.
In particular, it does not necessarily apply to code following the
if-statement:

\begin{VerbatimU}
	q = READ_ONCE(a);
	if (q) {
		WRITE_ONCE(b, 1);
	} else {
		WRITE_ONCE(b, 2);
	}
	WRITE_ONCE(c, 1);  /* BUG: No ordering against the read from 'a'. */
\end{VerbatimU}

It is tempting to argue that there in fact is ordering because the
compiler cannot reorder volatile accesses and also cannot reorder
the writes to~\qco{b} with the condition.
Unfortunately for this line of reasoning, the compiler might compile
the two writes to~\qco{b} as conditional-move instructions, as in this
fanciful pseudo-assembly language:

\begin{VerbatimU}
	ld r1,a
	cmp r1,$0
	cmov,ne r4,$1
	cmov,eq r4,$2
	st r4,b
	st $1,c
\end{VerbatimU}

A weakly ordered CPU would have no dependency of any sort between the load
from~\qco{a} and the store to~\qco{c}.
The control dependencies would extend only to the pair of cmov
instructions and the store depending on them.
In short, control dependencies apply only to the stores in the then-clause
and else-clause of the if-statement in question (including functions
invoked by those two clauses), not to code following that if-statement.


Note well that the ordering provided by a control dependency is local
to the CPU containing it.
See \cref{sec:Multicopy atomicity} for more information.


In summary:

\begin{itemize}
 \item
      Control dependencies can order prior loads against later stores.
      However, they do \emph{not} guarantee any other sort of ordering:
      Not prior loads against later loads, nor prior stores against
      later anything.
      If you need these other forms of ordering, use \co{smp_rmb()},
      \co{smp_wmb()}, or, in the case of prior stores and later loads,
      \co{smp_mb()}.

 \item
      If both legs of the \qco{if} statement begin with identical stores to
      the same variable, then those stores must be ordered, either by
      preceding both of them with \co{smp_mb()} or by using
      \co{smp_store_release()} to carry out the stores.
      Please note that it is \emph{not} sufficient to use \co{barrier()}
      at beginning of each leg of the \qco{if} statement because, as shown by
      the example above, optimizing compilers can destroy the control
      dependency while respecting the letter of the \co{barrier()} law.

 \item
      Control dependencies require at least one run-time conditional
      between the prior load and the subsequent store, and this
      conditional must involve the prior load.
      If the compiler is able to optimize the conditional away, it will
      have also optimized away the ordering.
      Careful use of \co{READ_ONCE()} and \co{WRITE_ONCE()} can help to
      preserve the needed conditional.

 \item
      Control dependencies require that the compiler avoid reordering the
      dependency into nonexistence.
      Careful use of \co{READ_ONCE()} or \co{atomic\{,64\}_read()} can help
      to preserve your control dependency.
      Please see \cref{sec:Compiler barrier} for more information.

 \item
      Control dependencies apply only to the then-clause and else-clause
      of the if-statement containing the control dependency, including
      any functions that these two clauses call.
      Control dependencies do \emph{not} apply to code following the
      if-statement containing the control dependency.

 \item
      Control dependencies pair normally with other types of barriers.

 \item
      Control dependencies do \emph{not} provide multicopy atomicity.
      If you need all the CPUs to see a given store at the same time,
      use \co{smp_mb()}.

 \item
      Compilers do not understand control dependencies.
      It is therefore your job to ensure that they do not break your code.
\end{itemize}

\subsection{SMP barrier pairing}
\label{sec:SMP barrier pairing}

When dealing with CPU-CPU interactions, certain types of memory barrier should
always be paired.
A lack of appropriate pairing is almost certainly an error.

General barriers pair with each other, though they also pair with most
other types of barriers, albeit without multicopy atomicity.
An acquire barrier pairs with a release barrier, but both may also pair
with other barriers, including of course general barriers.
A write barrier pairs with an address-dependency barrier,
a control dependency, an acquire barrier, a release barrier,
a read barrier, or a general barrier.
Similarly a read barrier, control dependency, or
an address-dependency barrier pairs with a write barrier,
an acquire barrier, a release barrier, or a general barrier:

\begin{VerbatimU}
	CPU 1                 CPU 2
	===============       ===============
	WRITE_ONCE(a, 1);
	<write barrier>
	WRITE_ONCE(b, 2);     x = READ_ONCE(b);
	                      <read barrier>
	                      y = READ_ONCE(a);
\end{VerbatimU}

Or:

\begin{VerbatimU}
	CPU 1                 CPU 2
	===============	      ===============================
	a = 1;
	<write barrier>
	WRITE_ONCE(b, &a);    x = READ_ONCE(b);
	                      <implicit AD barrier>
	                      y = *x;
\end{VerbatimU}

Or even:

\begin{VerbatimU}
	CPU 1                 CPU 2
	===============	      ===============================
	r1 = READ_ONCE(y);
	<general barrier>
	WRITE_ONCE(x, 1);     if (r2 = READ_ONCE(x)) {
	                      <implicit control dependency>
	                          WRITE_ONCE(y, 1);
	                      }

	assert(r1 == 0 || r2 == 0);
\end{VerbatimU}

Basically, the read barrier always has to be there, even though it can be of
the ``weaker'' type.

\begin{Note}
Note that the stores before the write barrier would normally be expected to
match the loads after the read barrier or the address-dependency barrier, and
vice versa:

\begin{VerbatimU}
	CPU 1                               CPU 2
	===================                 ===================
	WRITE_ONCE(a, 1);    }----   --->{  v = READ_ONCE(c);
	WRITE_ONCE(b, 2);    }    \ /    {  w = READ_ONCE(d);
	<write barrier>            \        <read barrier>
	WRITE_ONCE(c, 3);    }    / \    {  x = READ_ONCE(a);
	WRITE_ONCE(d, 4);    }----   --->{  y = READ_ONCE(b);
\end{VerbatimU}
\end{Note}

\subsection{Examples of memory barrier sequences}
\label{sec:Examples of memory barrier sequences}

Firstly, write barriers act as partial orderings on store operations.
Consider the following sequence of events:

\begin{VerbatimU}
	CPU 1
	=======================
	STORE A = 1
	STORE B = 2
	STORE C = 3
	<write barrier>
	STORE D = 4
	STORE E = 5
\end{VerbatimU}

This sequence of events is committed to the memory coherence system in an order
that the rest of the system might perceive as the unordered set of
\co{\{ STORE A, STORE B, STORE C \}} all occurring before the unordered
set of \co{\{ STORE D, STORE E \}}:

\begin{VerbatimU}
	+-------+       :      :
	|       |       +------+
	|       |------>| C=3  |     }     /\
	|       |  :    +------+     }-----  \  -----> Events perceptible to
	|       |  :    | A=1  |     }        \/       the rest of the system
	|       |  :    +------+     }
	| CPU 1 |  :    | B=2  |     }
	|       |       +------+     }
	|       |   wwwwwwwwwwwwwwww }   <--- At this point the write barrier
	|       |       +------+     }        requires all stores prior to the
	|       |  :    | E=5  |     }        barrier to be committed before
	|       |  :    +------+     }        further stores may take place
	|       |------>| D=4  |     }
	|       |       +------+
	+-------+       :      :
	                   |
	                   | Sequence in which stores are committed to the
	                   | memory system by CPU 1
	                   V
\end{VerbatimU}

Secondly, address-dependency barriers act as partial orderings on address-
dependent loads.
Consider the following sequence of events:

\begin{VerbatimU}
	CPU 1                   CPU 2
	======================= =======================
		{ B = 7; X = 9; Y = 8; C = &Y }
	STORE A = 1
	STORE B = 2
	<write barrier>
	STORE C = &B            LOAD X
	STORE D = 4             LOAD C (gets &B)
	                        LOAD *C (reads B)
\end{VerbatimU}

Without intervention, CPU~2 may perceive the events on CPU~1 in some
effectively random order, despite the write barrier issued by CPU~1:

\begin{VerbatimU}
	+-------+       :      :                :       :
	|       |       +------+                +-------+  | Sequence of update
	|       |------>| B=2  |-----       --->| Y->8  |  | of perception on
	|       |  :    +------+     \          +-------+  | CPU 2
	| CPU 1 |  :    | A=1  |      \     --->| C->&Y |  V
	|       |       +------+       |        +-------+
	|       |   wwwwwwwwwwwwwwww   |        :       :
	|       |       +------+       |        :       :
	|       |  :    | C=&B |---    |        :       :       +-------+
	|       |  :    +------+   \   |        +-------+       |       |
	|       |------>| D=4  |    ----------->| C->&B |------>|       |
	|       |       +------+       |        +-------+       |       |
	+-------+       :      :       |        :       :       |       |
	                               |        :       :       |       |
	                               |        :       :       | CPU 2 |
	                               |        +-------+       |       |
	    Apparently incorrect --->  |        | B->7  |------>|       |
	    perception of B (!)        |        +-------+       |       |
	                               |        :       :       |       |
	                               |        +-------+       |       |
	    The load of X holds --->    \       | X->9  |------>|       |
	    up the maintenance           \      +-------+       |       |
	    of coherence of B             ----->| B->2  |       +-------+
	                                        +-------+
	                                        :       :
\end{VerbatimU}

In the above example, CPU~2 perceives that \co{B} is~7, despite the load of~\co{*C}
(which would be~\co{B}) coming after the LOAD of~\co{C}.

If, however, an address-dependency barrier were to be placed between the load
of~\co{C} and the load of~\co{*C} (ie: \co{B}) on CPU~2:

\begin{VerbatimU}
	CPU 1                   CPU 2
	=======================	=======================
		{ B = 7; X = 9; Y = 8; C = &Y }
	STORE A = 1
	STORE B = 2
	<write barrier>
	STORE C = &B            LOAD X
	STORE D = 4             LOAD C (gets &B)
	                        <AD barrier>
	                        LOAD *C (reads B)
\end{VerbatimU}

\noindent%
then the following will occur:

\begin{VerbatimU}
	+-------+       :      :                :       :
	|       |       +------+                +-------+
	|       |------>| B=2  |-----       --->| Y->8  |
	|       |  :    +------+     \          +-------+
	| CPU 1 |  :    | A=1  |      \     --->| C->&Y |
	|       |       +------+       |        +-------+
	|       |   wwwwwwwwwwwwwwww   |        :       :
	|       |       +------+       |        :       :
	|       |  :    | C=&B |---    |        :       :       +-------+
	|       |  :    +------+   \   |        +-------+       |       |
	|       |------>| D=4  |    ----------->| C->&B |------>|       |
	|       |       +------+       |        +-------+       |       |
	+-------+       :      :       |        :       :       |       |
	                               |        :       :       |       |
	                               |        :       :       | CPU 2 |
	                               |        +-------+       |       |
	                               |        | X->9  |------>|       |
	                               |        +-------+       |       |
	  Makes sure all effects --->   \   aaaaaaaaaaaaaaaaa   |       |
	  prior to the store of C        \      +-------+       |       |
	  are perceptible to              ----->| B->2  |------>|       |
	  subsequent loads                      +-------+       |       |
	                                        :       :       +-------+
\end{VerbatimU}

And thirdly, a read barrier acts as a partial order on loads.
Consider the following sequence of events:

\begin{VerbatimU}
	CPU 1                   CPU 2
	=======================	=======================
		{ A = 0, B = 9 }
	STORE A=1
	<write barrier>
	STORE B=2
	                        LOAD B
	                        LOAD A
\end{VerbatimU}

Without intervention, CPU~2 may then choose to perceive the events on CPU~1 in
some effectively random order, despite the write barrier issued by CPU~1:

  \begin{VerbatimU}
	+-------+       :      :                :       :
	|       |       +------+                +-------+
	|       |------>| A=1  |------      --->| A->0  |
	|       |       +------+      \         +-------+
	| CPU 1 |   wwwwwwwwwwwwwwww   \    --->| B->9  |
	|       |       +------+        |       +-------+
	|       |------>| B=2  |---     |       :       :
	|       |       +------+   \    |       :       :       +-------+
	+-------+       :      :    \   |       +-------+       |       |
	                             ---------->| B->2  |------>|       |
	                                |       +-------+       | CPU 2 |
	                                |       | A->0  |------>|       |
	                                |       +-------+       |       |
	                                |       :       :       +-------+
	                                 \      :       :
	                                  \     +-------+
	                                   ---->| A->1  |
	                                        +-------+
	                                        :       :
\end{VerbatimU}

If, however, a read barrier were to be placed between the load of~\co{B} and the
load of~\co{A} on CPU~2:

\begin{VerbatimU}
	CPU 1                   CPU 2
	=======================	=======================
		{ A = 0, B = 9 }
	STORE A=1
	<write barrier>
	STORE B=2
	                        LOAD B
	                        <read barrier>
	                        LOAD A
\end{VerbatimU}

\noindent%
then the partial ordering imposed by CPU~1 will be perceived correctly by
CPU~2:

\begin{VerbatimU}
	+-------+       :      :                :       :
	|       |       +------+                +-------+
	|       |------>| A=1  |------      --->| A->0  |
	|       |       +------+      \         +-------+
	| CPU 1 |   wwwwwwwwwwwwwwww   \    --->| B->9  |
	|       |       +------+        |       +-------+
	|       |------>| B=2  |---     |       :       :
	|       |       +------+   \    |       :       :       +-------+
	+-------+       :      :    \   |       +-------+       |       |
	                             ---------->| B->2  |------>|       |
	                                |       +-------+       | CPU 2 |
	                                |       :       :       |       |
	                                |       :       :       |       |
	  At this point the read ---->   \  rrrrrrrrrrrrrrrrr   |       |
	  barrier causes all effects      \     +-------+       |       |
	  prior to the storage of B        ---->| A->1  |------>|       |
	  to be perceptible to CPU 2            +-------+       |       |
	                                        :       :       +-------+
\end{VerbatimU}

To illustrate this more completely, consider what could happen if the code
contained a load of A either side of the read barrier:

\begin{VerbatimU}
	CPU 1                   CPU 2
	=======================	=======================
		{ A = 0, B = 9 }
	STORE A=1
	<write barrier>
	STORE B=2
	                        LOAD B
	                        LOAD A [first load of A]
	                        <read barrier>
	                        LOAD A [second load of A]
\end{VerbatimU}

Even though the two loads of~\co{A} both occur after the load of~\co{B},
they may both come up with different values:

\begin{VerbatimU}
	+-------+       :      :                :       :
	|       |       +------+                +-------+
	|       |------>| A=1  |------      --->| A->0  |
	|       |       +------+      \         +-------+
	| CPU 1 |   wwwwwwwwwwwwwwww   \    --->| B->9  |
	|       |       +------+        |       +-------+
	|       |------>| B=2  |---     |       :       :
	|       |       +------+   \    |       :       :       +-------+
	+-------+       :      :    \   |       +-------+       |       |
	                             ---------->| B->2  |------>|       |
	                                |       +-------+       | CPU 2 |
	                                |       :       :       |       |
	                                |       :       :       |       |
	                                |       +-------+       |       |
	                                |       | A->0  |------>| 1st   |
	                                |       +-------+       |       |
	  At this point the read ---->   \  rrrrrrrrrrrrrrrrr   |       |
	  barrier causes all effects      \     +-------+       |       |
	  prior to the storage of B        ---->| A->1  |------>| 2nd   |
	  to be perceptible to CPU 2            +-------+       |       |
	                                        :       :       +-------+
\end{VerbatimU}

But it may be that the update to~\co{A} from CPU~1 becomes perceptible to CPU~2
before the read barrier completes anyway:

\begin{VerbatimU}
	+-------+       :      :                :       :
	|       |       +------+                +-------+
	|       |------>| A=1  |------      --->| A->0  |
	|       |       +------+      \         +-------+
	| CPU 1 |   wwwwwwwwwwwwwwww   \    --->| B->9  |
	|       |       +------+        |       +-------+
	|       |------>| B=2  |---     |       :       :
	|       |       +------+   \    |       :       :       +-------+
	+-------+       :      :    \   |       +-------+       |       |
	                             ---------->| B->2  |------>|       |
	                                |       +-------+       | CPU 2 |
	                                |       :       :       |       |
	                                 \      :       :       |       |
	                                  \     +-------+       |       |
	                                   ---->| A->1  |------>| 1st   |
	                                        +-------+       |       |
	                                    rrrrrrrrrrrrrrrrr   |       |
	                                        +-------+       |       |
	                                        | A->1  |------>| 2nd   |
	                                        +-------+       |       |
	                                        :       :       +-------+
\end{VerbatimU}

The guarantee is that the second load will always come up with \co{A == 1}
if the load of~\co{B} came up with \co{B == 2}.
No such guarantee exists for the first load of~\co{A}; that may come up with
either \co{A == 0} or \co{A == 1}.


\subsection{Read memory barriers vs load speculation}

Many CPUs speculate with loads:
That is they see that they will need to load an item from memory, and
they find a time where they're not using the bus for any other loads,
and so do the load in advance---even though they haven't actually
got to that point in the instruction execution flow yet.
This permits the actual load instruction to potentially complete
immediately because the CPU already has the value to hand.

It may turn out that the CPU didn't actually need the value---perhaps
because a branch circumvented the load---in which case it can discard
the value or just cache it for later use.

Consider:

\begin{VerbatimU}
	CPU 1    CPU 2
	======== =======================
	         LOAD B
	         DIVIDE  } Divide instructions generally
	         DIVIDE  } take a long time to perform
	         LOAD A
\end{VerbatimU}

Which might appear as this:

\begin{VerbatimU}
	                                        :       :       +-------+
	                                        +-------+       |       |
	                                    --->| B->2  |------>|       |
	                                        +-------+       | CPU 2 |
	                                        :       :DIVIDE |       |
	                                        +-------+       |       |
	The CPU being busy doing a --->     --->| A->0  |~~~~   |       |
	division speculates on the              +-------+   ~   |       |
	LOAD of A                               :       :   ~   |       |
	                                        :       :DIVIDE |       |
	                                        :       :   ~   |       |
	Once the divisions are complete -->     :       :   ~-->|       |
	the CPU can then perform the            :       :       |       |
	LOAD with immediate effect              :       :       +-------+
\end{VerbatimU}

Placing a read barrier or an address-dependency barrier just before the second
load:

\begin{VerbatimU}
	CPU 1	CPU 2
	=======	=======================
	        LOAD B
	        DIVIDE
	        DIVIDE
	        <read barrier>
	        LOAD A
\end{VerbatimU}

\noindent%
will force any value speculatively obtained to be reconsidered to an extent
dependent on the type of barrier used.
If there was no change made to the speculated memory location, then the
speculated value will just be used:

\begin{VerbatimU}
	                                        :       :       +-------+
	                                        +-------+       |       |
	                                    --->| B->2  |------>|       |
	                                        +-------+       | CPU 2 |
	                                        :       :DIVIDE |       |
	                                        +-------+       |       |
	The CPU being busy doing a --->     --->| A->0  |~~~~   |       |
	division speculates on the              +-------+   ~   |       |
	LOAD of A                               :       :   ~   |       |
	                                        :       :DIVIDE |       |
	                                        :       :   ~   |       |
	                                        :       :   ~   |       |
	                                    rrrrrrrrrrrrrrrr~   |       |
	                                        :       :   ~   |       |
	                                        :       :   ~-->|       |
	                                        :       :       |       |
	                                        :       :       +-------+
\end{VerbatimU}

\noindent%
but if there was an update or an invalidation from another CPU pending, then
the speculation will be cancelled and the value reloaded:

\begin{VerbatimU}
	                                        :       :       +-------+
	                                        +-------+       |       |
	                                    --->| B->2  |------>|       |
	                                        +-------+       | CPU 2 |
	                                        :       :DIVIDE |       |
	                                        +-------+       |       |
	The CPU being busy doing a --->     --->| A->0  |~~~~   |       |
	division speculates on the              +-------+   ~   |       |
	LOAD of A                               :       :   ~   |       |
	                                        :       :DIVIDE |       |
	                                        :       :   ~   |       |
	                                        :       :   ~   |       |
	                                    rrrrrrrrrrrrrrrrr   |       |
	                                        +-------+       |       |
	The speculation is discarded --->   --->| A->1  |------>|       |
	and an updated value is                 +-------+       |       |
	retrieved                               :       :       +-------+
\end{VerbatimU}


\subsection{Multicopy atomicity}
\label{sec:Multicopy atomicity}

Multicopy atomicity is a deeply intuitive notion about ordering that is
not always provided by real computer systems, namely that a given store
becomes visible at the same time to all CPUs, or, alternatively, that all
CPUs agree on the order in which all stores become visible.
However, support of full multicopy atomicity would rule out valuable
hardware optimizations, so a weaker form called ``other multicopy atomicity''
instead guarantees only that a given store becomes visible at the same
time to all \emph{other} CPUs.
The remainder of this document discusses this weaker form, but for brevity
will call it simply ``multicopy atomicity''.

The following example demonstrates multicopy atomicity:

\begin{VerbatimU}
	CPU 1                   CPU 2                 CPU 3
	======================= ===================== ==================
		{ X = 0, Y = 0 }
	STORE X=1               r1=LOAD X (reads 1)   LOAD Y (reads 1)
	                        <general barrier>     <read barrier>
	                        STORE Y=r1            LOAD X
\end{VerbatimU}

Suppose that CPU~2's load from~\co{X} returns~1, which it then stores to~\co{Y},
and CPU~3's load from~\co{Y} returns~1.
This indicates that CPUd~1's store to~\co{X} precedes CPU~2's load from~\co{X} and
that CPU~2's store to~\co{Y} precedes CPU~3's load from~\co{Y}.
In addition, the memory barriers guarantee that CPU~2 executes its load
before its store, and CPU~3 loads from~\co{Y} before it loads from~\co{X}.
The question is then ``Can CPU~3's load from~\co{X} return~0?''

Because CPU~3's load from~\co{X} in some sense comes after CPU~2's load, it
is natural to expect that CPU~3's load from~\co{X} must therefore return~1.
This expectation follows from multicopy atomicity:
If a load executing on CPU~B follows a load from the same variable
executing on CPU~A (and CPU~A did not originally store the value which
it read), then on multicopy-atomic systems, CPU~B's load must return
either the same value that CPU~A's load did or some later value.
However, the Linux kernel does not require systems to be multicopy atomic.

The use of a general memory barrier in the example above compensates
for any lack of multicopy atomicity.
In the example, if CPU~2's load from~\co{X} returns~1 and CPU~3's load from~\co{Y}
returns~1, then CPU~3's load from~\co{X} must indeed also return~1.

However, dependencies, read barriers, and write barriers are not always
able to compensate for non-multicopy atomicity.
For example, suppose that CPU~2's general barrier is removed from the
above example, leaving only the data dependency shown below:

\begin{VerbatimU}
	CPU 1       CPU 2                CPU 3
	=========== ==================== =================
		{ X = 0, Y = 0 }
	STORE X=1   r1=LOAD X (reads 1)  LOAD Y (reads 1)
	            <data dependency>    <read barrier>
	            STORE Y=r1           LOAD X (reads 0)
\end{VerbatimU}

This substitution allows non-multicopy atomicity to run rampant:
In this example, it is perfectly legal for CPU~2's load from~\co{X} to return~1,
CPU~3's load from~\co{Y} to return~1, and its load from~\co{X} to return~0.

The key point is that although CPU~2's data dependency orders its load
and store, it does not guarantee to order CPU~1's store.
Thus, if this example runs on a non-multicopy-atomic system where CPUs~1
and~2 share a store buffer or a level of cache, CPU~2 might have early
access to CPU~1's writes.
General barriers are therefore required to ensure that all CPUs agree
on the combined order of multiple accesses.

General barriers can compensate not only for non-multicopy atomicity,
but can also generate additional ordering that can ensure that \emph{all}
CPUs will perceive the same order of \emph{all} operations.
In contrast, a chain of release-acquire pairs do not provide this
additional ordering, which means that only those CPUs on the chain are
guaranteed to agree on the combined order of the accesses.
For example, switching to C code in deference to the ghost of Herman
Hollerith:

\begin{VerbatimU}
	int u, v, x, y, z;

	void cpu0(void)
	{
		r0 = smp_load_acquire(&x);
		WRITE_ONCE(u, 1);
		smp_store_release(&y, 1);
	}

	void cpu1(void)
	{
		r1 = smp_load_acquire(&y);
		r4 = READ_ONCE(v);
		r5 = READ_ONCE(u);
		smp_store_release(&z, 1);
	}

	void cpu2(void)
	{
		r2 = smp_load_acquire(&z);
		smp_store_release(&x, 1);
	}

	void cpu3(void)
	{
		WRITE_ONCE(v, 1);
		smp_mb();
		r3 = READ_ONCE(u);
	}
\end{VerbatimU}

Because \co{cpu0()}, \co{cpu1()}, and \co{cpu2()} participate in a chain of
\co{smp_store_release()}/\co{smp_load_acquire()} pairs, the following outcome
is prohibited:

\begin{VerbatimU}
	r0 == 1 && r1 == 1 && r2 == 1
\end{VerbatimU}

Furthermore, because of the release-acquire relationship between \co{cpu0()}
and \co{cpu1()}, \co{cpu1()} must see \co{cpu0()}'s writes, so that the
following outcome is prohibited:

\begin{VerbatimU}
	r1 == 1 && r5 == 0
\end{VerbatimU}

However, the ordering provided by a release-acquire chain is local
to the CPUs participating in that chain and does not apply to \co{cpu3()},
at least aside from stores.
Therefore, the following outcome is possible:

\begin{VerbatimU}
	r0 == 0 && r1 == 1 && r2 == 1 && r3 == 0 && r4 == 0
\end{VerbatimU}

As an aside, the following outcome is also possible:

\begin{VerbatimU}
	r0 == 0 && r1 == 1 && r2 == 1 && r3 == 0 && r4 == 0 && r5 == 1
\end{VerbatimU}

Although \co{cpu0()}, \co{cpu1()}, and \co{cpu2()} will see their
respective reads and writes in order, CPUs not involved in the
release-acquire chain might well disagree on the order.
This disagreement stems from the fact that the weak memory-barrier
instructions used to implement \co{smp_load_acquire()} and
\co{smp_store_release()} are not required to order prior stores against
subsequent loads in all cases.
This means that \co{cpu3()} can see \co{cpu0()}'s store to~u as happening
\emph{after} \co{cpu1()}'s load from~v, even though both \co{cpu0()} and
\co{cpu1()} agree that these two operations occurred in the intended order.

However, please keep in mind that \co{smp_load_acquire()} is not magic.
In particular, it simply reads from its argument with ordering.
It does \emph{not} ensure that any particular value will be read.
Therefore, the following outcome is possible:

\begin{VerbatimU}
	r0 == 0 && r1 == 0 && r2 == 0 && r5 == 0
\end{VerbatimU}

Note that this outcome can happen even on a mythical sequentially
consistent system where nothing is ever reordered.

To reiterate, if your code requires full ordering of all operations,
use general barriers throughout.


\section{Explicit kernel barriers}

The Linux kernel has a variety of different barriers that act at different
levels:

\begin{itemize}[itemsep=0pt]
 \item Compiler barrier.
 \item CPU memory barriers.
\end{itemize}


\subsection{Compiler barrier}
\label{sec:Compiler barrier}

The Linux kernel has an explicit compiler barrier function that prevents the
compiler from moving the memory accesses either side of it to the other side:

\begin{VerbatimU}
	barrier();
\end{VerbatimU}

This is a general barrier---there are no read-read or write-write
variants of \co{barrier()}.
However, \co{READ_ONCE()} and \co{WRITE_ONCE()} can be thought of as
weak forms of \co{barrier()} that affect only the specific accesses
flagged by the \co{READ_ONCE()} or \co{WRITE_ONCE()}.

The \co{barrier()} function has the following effects:

\begin{itemize}
 \item
     Prevents the compiler from reordering accesses following the
     \co{barrier()} to precede any accesses preceding the \co{barrier()}.
     One example use for this property is to ease communication between
     interrupt-handler code and the code that was interrupted.

 \item
     Within a loop, forces the compiler to load the variables used
     in that loop's conditional on each pass through that loop.
\end{itemize}

The \co{READ_ONCE()} and \co{WRITE_ONCE()} functions can prevent any number of
optimizations that, while perfectly safe in single-threaded code, can
be fatal in concurrent code.
Here are some examples of these sorts of optimizations:

\begin{itemize}
 \item
     The compiler is within its rights to reorder loads and stores
     to the same variable, and in some cases, the CPU is within its
     rights to reorder loads to the same variable.
     This means that the following code:

\begin{VerbatimU}
	a[0] = x;
	a[1] = x;
\end{VerbatimU}

     Might result in an older value of~\co{x} stored in \co{a[1]} than in \co{a[0]}.
     Prevent both the compiler and the CPU from doing this as follows:

\begin{VerbatimU}
	a[0] = READ_ONCE(x);
	a[1] = READ_ONCE(x);
\end{VerbatimU}

     In short, \co{READ_ONCE()} and \co{WRITE_ONCE()} provide cache coherence
     for accesses from multiple CPUs to a single variable.

 \item
     The compiler is within its rights to merge successive loads from
     the same variable.
     Such merging can cause the compiler to ``optimize'' the following code:

\begin{VerbatimU}
	while (tmp = a)
		do_something_with(tmp);
\end{VerbatimU}

     into the following code, which, although in some sense legitimate
     for single-threaded code, is almost certainly not what the developer
     intended:

\begin{VerbatimU}
	if (tmp = a)
		for (;;)
			do_something_with(tmp);
\end{VerbatimU}

     Use \co{READ_ONCE()} to prevent the compiler from doing this to you:

\begin{VerbatimU}
	while (tmp = READ_ONCE(a))
		do_something_with(tmp);
\end{VerbatimU}

 \item
     The compiler is within its rights to reload a variable, for example,
     in cases where high register pressure prevents the compiler from
     keeping all data of interest in registers.
     The compiler might therefore optimize the variable~\qco{tmp} out of
     our previous example:

\begin{VerbatimU}
	while (tmp = a)
		do_something_with(tmp);
\end{VerbatimU}

     This could result in the following code, which is perfectly safe in
     single-threaded code, but can be fatal in concurrent code:

\begin{VerbatimU}
	while (a)
		do_something_with(a);
\end{VerbatimU}

     For example, the optimized version of this code could result in
     passing a zero to \co{do_something_with()} in the case where the variable
     a was modified by some other CPU between the \qco{while} statement and
     the call to \co{do_something_with()}.

     Again, use \co{READ_ONCE()} to prevent the compiler from doing this:

\begin{VerbatimU}
	while (tmp = READ_ONCE(a))
		do_something_with(tmp);
\end{VerbatimU}

     Note that if the compiler runs short of registers, it might save
     tmp onto the stack.
     The overhead of this saving and later restoring is why compilers
     reload variables.
     Doing so is perfectly safe for single-threaded code, so you need
     to tell the compiler about cases where it is not safe.

 \item
     The compiler is within its rights to omit a load entirely if it knows
     what the value will be.
     For example, if the compiler can prove that the value of variable~\qco{a}
     is always zero, it can optimize this code:

\begin{VerbatimU}
	while (tmp = a)
		do_something_with(tmp);
\end{VerbatimU}

     Into this:

\begin{VerbatimU}
	do { } while (0);
\end{VerbatimU}

     This transformation is a win for single-threaded code because it
     gets rid of a load and a branch.
     The problem is that the compiler will carry out its proof assuming
     that the current CPU is the only one updating variable~\qco{a}.
     If variable~\qco{a} is shared, then the compiler's proof will be erroneous.
     Use \co{READ_ONCE()} to tell the compiler that it doesn't know as
     much as it thinks it does:

\begin{VerbatimU}
	while (tmp = READ_ONCE(a))
		do_something_with(tmp);
\end{VerbatimU}

     But please note that the compiler is also closely watching what you
     do with the value after the \co{READ_ONCE()}.
     For example, suppose you do the following and \co{MAX} is a preprocessor
     macro with the value~1:

\begin{VerbatimU}
	while ((tmp = READ_ONCE(a)) % MAX)
		do_something_with(tmp);
\end{VerbatimU}

     Then the compiler knows that the result of the \qco{\%} operator applied
     to \co{MAX} will always be zero, again allowing the compiler to optimize
     the code into near-nonexistence.
     (It will still load from the variable~\qco{a}.)

 \item
     Similarly, the compiler is within its rights to omit a store entirely
     if it knows that the variable already has the value being stored.
     Again, the compiler assumes that the current CPU is the only one
     storing into the variable, which can cause the compiler to do the
     wrong thing for shared variables.
     For example, suppose you have the following:

\begin{VerbatimU}
	a = 0;
	... Code that does not store to variable a ...
	a = 0;
\end{VerbatimU}

     The compiler sees that the value of variable~\qco{a} is already zero, so
     it might well omit the second store.
     This would come as a fatal surprise if some other CPU might have
     stored to variable~\qco{a} in the meantime.

     Use \co{WRITE_ONCE()} to prevent the compiler from making this sort of
     wrong guess:

\begin{VerbatimU}
	WRITE_ONCE(a, 0);
	... Code that does not store to variable a ...
	WRITE_ONCE(a, 0);
\end{VerbatimU}

\item
     The compiler is within its rights to reorder memory accesses unless
     you tell it not to.
     For example, consider the following interaction between process-level
     code and an interrupt handler:

\begin{VerbatimU}
	void process_level(void)
	{
		msg = get_message();
		flag = true;
	}

	void interrupt_handler(void)
	{
		if (flag)
			process_message(msg);
	}
\end{VerbatimU}

     There is nothing to prevent the compiler from transforming
     \co{process_level()} to the following, in fact, this might well be a
     win for single-threaded code:

\begin{VerbatimU}
	void process_level(void)
	{
		flag = true;
		msg = get_message();
	}
\end{VerbatimU}

     If the interrupt occurs between these two statement, then
     \co{interrupt_handler()} might be passed a garbled \co{msg}.
     Use \co{WRITE_ONCE()} to prevent this as follows:

\begin{VerbatimU}
	void process_level(void)
	{
		WRITE_ONCE(msg, get_message());
		WRITE_ONCE(flag, true);
	}

	void interrupt_handler(void)
	{
		if (READ_ONCE(flag))
			process_message(READ_ONCE(msg));
	}
\end{VerbatimU}

     Note that the \co{READ_ONCE()} and \co{WRITE_ONCE()} wrappers in
     \co{interrupt_handler()} are needed if this interrupt handler can itself
     be interrupted by something that also accesses \qco{flag} and \qco{msg},
     for example, a nested interrupt or an NMI\@.
     Otherwise, \co{READ_ONCE()} and \co{WRITE_ONCE()} are not needed
     in \co{interrupt_handler()} other than for documentation purposes.
     (Note also that nested interrupts do not typically occur in modern
     Linux kernels, in fact, if an interrupt handler returns with
     interrupts enabled, you will get a \co{WARN_ONCE()} splat.)

     You should assume that the compiler can move \co{READ_ONCE()} and
     \co{WRITE_ONCE()} past code not containing \co{READ_ONCE()},
     \co{WRITE_ONCE()}, \co{barrier()}, or similar primitives.

     This effect could also be achieved using \co{barrier()}, but
     \co{READ_ONCE()} and \co{WRITE_ONCE()} are more selective:
     With \co{READ_ONCE()} and \co{WRITE_ONCE()}, the compiler need only
     forget the contents of the indicated memory locations, while with
     \co{barrier()} the compiler must discard the value of all memory
     locations that it has currently cached in any machine registers.
     Of course, the compiler must also respect the order in which the
     \co{READ_ONCE()}s and \co{WRITE_ONCE()}s occur, though the CPU of
     course need not do so.

 \item
     The compiler is within its rights to invent stores to a variable,
     as in the following example:

\begin{VerbatimU}
	if (a)
		b = a;
	else
		b = 42;
\end{VerbatimU}

     The compiler might save a branch by optimizing this as follows:

\begin{VerbatimU}
	b = 42;
	if (a)
		b = a;
\end{VerbatimU}

     In single-threaded code, this is not only safe, but also saves
     a branch.
     Unfortunately, in concurrent code, this optimization could cause
     some other CPU to see a spurious value of~42---even if variable~\qco{a}
     was never zero---when loading variable~\qco{b}.
     Use \co{WRITE_ONCE()} to prevent this as follows:

\begin{VerbatimU}
	if (a)
		WRITE_ONCE(b, a);
	else
		WRITE_ONCE(b, 42);
\end{VerbatimU}

     The compiler can also invent loads.
     These are usually less damaging, but they can result in cache-line
     bouncing and thus in poor performance and scalability.
     Use \co{READ_ONCE()} to prevent invented loads.

 \item
     For aligned memory locations whose size allows them to be accessed
     with a single memory-reference instruction, prevents ``load tearing''
     and ``store tearing,'' in which a single large access is replaced by
     multiple smaller accesses.
     For example, given an architecture having 16-bit store instructions
     with 7-bit immediate fields, the compiler might be tempted to use
     two 16-bit store-immediate instructions to implement the following
     32-bit store:

\begin{VerbatimU}
	p = 0x00010002;
\end{VerbatimU}

     Please note that GCC really does use this sort of optimization,
     which is not surprising given that it would likely take more
     than two instructions to build the constant and then store it.
     This optimization can therefore be a win in single-threaded code.
     In fact, a recent bug (since fixed) caused GCC to incorrectly use
     this optimization in a volatile store.
     In the absence of such bugs, use of \co{WRITE_ONCE()} prevents
     store tearing in the following example:

\begin{VerbatimU}
	WRITE_ONCE(p, 0x00010002);
\end{VerbatimU}

     Use of packed structures can also result in load and store tearing,
     as in this example:

\begin{VerbatimU}
	struct __attribute__((__packed__)) foo {
		short a;
		int b;
		short c;
	};
	struct foo foo1, foo2;
	...

	foo2.a = foo1.a;
	foo2.b = foo1.b;
	foo2.c = foo1.c;
\end{VerbatimU}

     Because there are no \co{READ_ONCE()} or \co{WRITE_ONCE()} wrappers and no
     volatile markings, the compiler would be well within its rights to
     implement these three assignment statements as a pair of 32-bit
     loads followed by a pair of 32-bit stores.
     This would result in load tearing on \qco{foo1.b} and store tearing
     on \qco{foo2.b}.
     \co{READ_ONCE()} and \co{WRITE_ONCE()} again prevent tearing in
     this example:

\begin{VerbatimU}
	foo2.a = foo1.a;
	WRITE_ONCE(foo2.b, READ_ONCE(foo1.b));
	foo2.c = foo1.c;
\end{VerbatimU}
\end{itemize}

All that aside, it is never necessary to use \co{READ_ONCE()} and
\co{WRITE_ONCE()} on a variable that has been marked volatile.
For example, because \qco{jiffies} is marked volatile, it is never necessary to
say \co{READ_ONCE(jiffies)}.
The reason for this is that \co{READ_ONCE()} and \co{WRITE_ONCE()} are
implemented as volatile casts, which has no effect when its argument is
already marked volatile.

Please note that these compiler barriers have no direct effect on the CPU,
which may then reorder things however it wishes.


\subsection{CPU memory barriers}

The Linux kernel has seven basic CPU memory barriers:

\begin{VerbatimU}
	TYPE                  MANDATORY       SMP CONDITIONAL
	===================== =============== ===============
	GENERAL               mb()            smp_mb()
	WRITE                 wmb()           smp_wmb()
	READ                  rmb()           smp_rmb()
	ADDRESS DEPENDENCY                    READ_ONCE()
\end{VerbatimU}

All memory barriers except the address-dependency barriers imply a compiler
barrier.
Address dependencies do not impose any additional compiler ordering.

Aside:
In the case of address dependencies, the compiler would be expected
to issue the loads in the correct order (eg., \qco{a[b]} would have to load
the value of~\co{b} before loading \co{a[b]}), however there is no guarantee in
the C specification that the compiler may not speculate the value of~\qco{b}
(eg., is equal to~1) and load \co{a[b]} before~\qco{b} (eg.,
\co{tmp = a[1]; if (b != 1) tmp = a[b];}).
There is also the problem of a compiler reloading~\qco{b} after having loaded
\co{a[b]}, thus having a newer copy of~\qco{b} than \co{a[b]}.
A consensus has not yet been reached about these problems, however the
\co{READ_ONCE()} macro is a good place to start looking.

SMP memory barriers are reduced to compiler barriers on uniprocessor compiled
systems because it is assumed that a CPU will appear to be self-consistent,
and will order overlapping accesses correctly with respect to itself.
However, see \cref{sec:Virtual machine guests} below.

\begin{Note}
Note that SMP memory barriers \emph{must} be used to control the ordering of
references to shared memory on SMP systems, though the use of locking instead
is sufficient.
\end{Note}

Mandatory barriers should not be used to control SMP effects, since mandatory
barriers impose unnecessary overhead on both SMP and UP systems.
They may, however, be used to control MMIO effects on accesses through
relaxed memory I/O windows.
These barriers are required even on non-SMP systems as they affect the order
in which memory operations appear to a device by prohibiting both the
compiler and the CPU from reordering them.


There are some more advanced barrier functions:

\begin{description}[style=nextline]
 \item[\tco{smp_store_mb(var, value)}]
     This assigns the value to the variable and then inserts a full memory
     barrier after it.  It isn't guaranteed to insert anything more than a
     compiler barrier in a UP compilation.


 \item[\tco{smp_mb__before_atomic()} and \tco{smp_mb__after_atomic()}]
     These are for use with atomic RMW functions that do not imply memory
     barriers, but where the code needs a memory barrier.
     Examples for atomic RMW functions that do not imply a memory barrier
     are e.g., add, subtract, (failed) conditional operations,
     \co{_relaxed} functions, but not \co{atomic_read} or \co{atomic_set}.
     A common example where a memory barrier may be required is when
     atomic ops are used for reference counting.

     These are also used for atomic RMW bitop functions that do not imply a
     memory barrier (such as \co{set_bit} and \co{clear_bit}).

     As an example, consider a piece of code that marks an object as being dead
     and then decrements the object's reference count:

\begin{VerbatimU}
	obj->dead = 1;
	smp_mb__before_atomic();
	atomic_dec(&obj->ref_count);
\end{VerbatimU}

     This makes sure that the death mark on the object is perceived to be set
     \emph{before} the reference counter is decremented.

     See \path{Documentation/atomic_{t,bitops}.txt} for more information.

   \item[\tco{dma_wmb()}, \tco{dma_rmb()}, and \tco{dma_mb()}]
     These are for use with consistent memory to guarantee the ordering
     of writes or reads of shared memory accessible to both the CPU and a
     DMA capable device.
     See \path{Documentation/core-api/dma-api.rst} file for more
     information about consistent memory.

     For example, consider a device driver that shares memory with a device
     and uses a descriptor status value to indicate if the descriptor belongs
     to the device or the CPU, and a doorbell to notify it when new
     descriptors are available:

\begin{VerbatimU}
	if (desc->status != DEVICE_OWN) {
		/* do not read data until we own descriptor */
		dma_rmb();

		/* read/modify data */
		read_data = desc->data;
		desc->data = write_data;

		/* flush modifications before status update */
		dma_wmb();

		/* assign ownership */
		desc->status = DEVICE_OWN;

		/* Make descriptor status visible to the device followed by
		 * notify device of new descriptor
		 */
		writel(DESC_NOTIFY, doorbell);
	}
\end{VerbatimU}

     The \co{dma_rmb()} allows us to guarantee that the device has released
     ownership before we read the data from the descriptor, and the
     \co{dma_wmb()} allows us to guarantee the data is written to the
     descriptor before the device can see it now has ownership.
     The \co{dma_mb()} implies both a \co{dma_rmb()} and a \co{dma_wmb()}.

     Note that the \co{dma_*()} barriers do not provide any ordering
     guarantees for accesses to MMIO regions.
     See the later \cref{sec:Kernel I/O barrier effects} for more
     information about I/O accessors and MMIO ordering.

 \item[\tco{pmem_wmb()}]
     This is for use with persistent memory to ensure that stores for which
     modifications are written to persistent storage reached a platform
     durability domain.

     For example, after a non-temporal write to pmem region, we use
     \co{pmem_wmb()} to ensure that stores have reached a platform
     durability domain.
     This ensures that stores have updated persistent storage before any
     data access or data transfer caused by subsequent instructions is
     initiated.
     This is in addition to the ordering done by \co{wmb()}.

     For load from persistent memory, existing read memory barriers are
     sufficient to ensure read ordering.

 \item[\tco{io_stop_wc()}]
     For memory accesses with write-combining attributes (e.g., those returned
     by \co{ioremap_wc()}), the CPU may wait for prior accesses to be merged
     with subsequent ones.
     \co{io_stop_wc()} can be used to prevent the merging of write-combining
     memory accesses before this macro with those after it when such wait
     has performance implications.
\end{description}


\section{Implicit kernel memory barriers}

Some of the other functions in the linux kernel imply memory barriers, amongst
which are locking and scheduling functions.

This specification is a \emph{minimum} guarantee; any particular architecture
may provide more substantial guarantees, but these may not be relied upon
outside of arch specific code.


\subsection{Lock acquisition functions}

The Linux kernel has a number of locking constructs:

\begin{itemize}[itemsep=0pt]
 \item spin locks
 \item R/W spin locks
 \item mutexes
 \item semaphores
 \item R/W semaphores
\end{itemize}

In all cases there are variants on ``ACQUIRE'' operations and ``RELEASE'' operations
for each construct.
These operations all imply certain barriers:

\begin{description}[style=nextline]
 \item[ACQUIRE operation implication:]
     Memory operations issued after the ACQUIRE will be completed after the
     ACQUIRE operation has completed.

     Memory operations issued before the ACQUIRE may be completed after
     the ACQUIRE operation has completed.

 \item[RELEASE operation implication:]
     Memory operations issued before the RELEASE will be completed before the
     RELEASE operation has completed.

     Memory operations issued after the RELEASE may be completed before the
     RELEASE operation has completed.

 \item[ACQUIRE vs ACQUIRE implication:]
     All ACQUIRE operations issued before another ACQUIRE operation will be
     completed before that ACQUIRE operation.

 \item[ACQUIRE vs RELEASE implication:]
     All ACQUIRE operations issued before a RELEASE operation will be
     completed before the RELEASE operation.

 \item[Failed conditional ACQUIRE implication:]
     Certain locking variants of the ACQUIRE operation may fail, either due to
     being unable to get the lock immediately, or due to receiving an unblocked
     signal while asleep waiting for the lock to become available.
     Failed locks do not imply any sort of barrier.
\end{description}

\begin{Note}
One of the consequences of lock ACQUIREs and RELEASEs being only
one-way barriers is that the effects of instructions outside of a critical
section may seep into the inside of the critical section.
\end{Note}

An ACQUIRE followed by a RELEASE may not be assumed to be full memory barrier
because it is possible for an access preceding the ACQUIRE to happen after the
ACQUIRE, and an access following the RELEASE to happen before the RELEASE, and
the two accesses can themselves then cross:

\begin{VerbatimU}
	*A = a;
	ACQUIRE M
	RELEASE M
	*B = b;
\end{VerbatimU}

\noindent%
may occur as:

\begin{VerbatimU}
	ACQUIRE M, STORE *B, STORE *A, RELEASE M
\end{VerbatimU}

When the ACQUIRE and RELEASE are a lock acquisition and release,
respectively, this same reordering can occur if the lock's ACQUIRE and
RELEASE are to the same lock variable, but only from the perspective of
another CPU not holding that lock.
In short, a ACQUIRE followed by an RELEASE may \emph{not} be assumed to
be a full memory barrier.

Similarly, the reverse case of a RELEASE followed by an ACQUIRE does
not imply a full memory barrier.
Therefore, the CPU's execution of the critical sections corresponding
to the RELEASE and the ACQUIRE can cross, so that:

\begin{VerbatimU}
	*A = a;
	RELEASE M
	ACQUIRE N
	*B = b;
\end{VerbatimU}

\noindent%
could occur as:

\begin{VerbatimU}
	ACQUIRE N, STORE *B, STORE *A, RELEASE M
\end{VerbatimU}

It might appear that this reordering could introduce a deadlock.
However, this cannot happen because if such a deadlock threatened,
the RELEASE would simply complete, thereby avoiding the deadlock.

\begin{quote}
	Why does this work?

	One key point is that we are only talking about the CPU doing
	the reordering, not the compiler.
	If the compiler (or, for that matter, the developer) switched
	the operations, deadlock \emph{could} occur.

	But suppose the CPU reordered the operations.
	In this case, the unlock precedes the lock in the assembly code.
	The CPU simply elected to try executing the later lock operation
	first.
	If there is a deadlock, this lock operation will simply spin (or
	try to sleep, but more on that later).
	The CPU will eventually execute the unlock operation (which
	preceded the lock operation in the assembly code), which will
	unravel the potential deadlock, allowing the lock operation to
	succeed.

	But what if the lock is a sleeplock?
	In that case, the code will try to enter the scheduler, where it
	will eventually encounter a memory barrier, which will force the
	earlier unlock operation to complete, again unraveling the deadlock.
	There might be a sleep-unlock race, but the locking primitive
	needs to resolve such races properly in any case.

\end{quote}

Locks and semaphores may not provide any guarantee of ordering on UP compiled
systems, and so cannot be counted on in such a situation to actually achieve
anything at all---especially with respect to I/O accesses---unless combined
with interrupt disabling operations.

See also \cref{sec:Inter-CPU acquiring barrier effects}.


As an example, consider the following:

\begin{VerbatimU}
	*A = a;
	*B = b;
	ACQUIRE
	*C = c;
	*D = d;
	RELEASE
	*E = e;
	*F = f;
\end{VerbatimU}

The following sequence of events is acceptable:

\begin{VerbatimU}
	ACQUIRE, {*F,*A}, *E, {*C,*D}, *B, RELEASE

	[+] Note that {*F,*A} indicates a combined access.
\end{VerbatimU}

But none of the following are:

\begin{VerbatimU}
	{*F,*A}, *B,   ACQUIRE, *C, *D,   RELEASE, *E
	*A, *B, *C,    ACQUIRE, *D,       RELEASE, *E, *F
	*A, *B,        ACQUIRE, *C,       RELEASE, *D, *E, *F
	*B,            ACQUIRE, *C, *D,   RELEASE, {*F,*A}, *E
\end{VerbatimU}


\subsection{Interrupt disabling functions}

Functions that disable interrupts (ACQUIRE equivalent) and enable interrupts
(RELEASE equivalent) will act as compiler barriers only.
So if memory or I/O barriers are required in such a situation, they must be
provided from some other means.


\subsection{Sleep and wake-up functions}

Sleeping and waking on an event flagged in global data can be viewed as an
interaction between two pieces of data:
The task state of the task waiting for the event and the global data used
to indicate the event.
To make sure that these appear to happen in the right order, the primitives
to begin the process of going to sleep, and the primitives to initiate a
wake up imply certain barriers.

Firstly, the sleeper normally follows something like this sequence of events:

\begin{VerbatimU}
	for (;;) {
		set_current_state(TASK_UNINTERRUPTIBLE);
		if (event_indicated)
			break;
		schedule();
	}
\end{VerbatimU}

A general memory barrier is interpolated automatically by
\co{set_current_state()} after it has altered the task state:

\begin{VerbatimU}
	CPU 1
	===============================
	set_current_state();
	  smp_store_mb();
	    STORE current->state
	    <general barrier>
	LOAD event_indicated
\end{VerbatimU}

\co{set_current_state()} may be wrapped by:

\begin{VerbatimU}
	prepare_to_wait();
	prepare_to_wait_exclusive();
\end{VerbatimU}

\noindent%
which therefore also imply a general memory barrier after setting the state.
The whole sequence above is available in various canned forms, all of which
interpolate the memory barrier in the right place:

\begin{VerbatimU}
	wait_event();
	wait_event_interruptible();
	wait_event_interruptible_exclusive();
	wait_event_interruptible_timeout();
	wait_event_killable();
	wait_event_timeout();
	wait_on_bit();
	wait_on_bit_lock();
\end{VerbatimU}

Secondly, code that performs a wake up normally follows something like this:

\begin{VerbatimU}
	event_indicated = 1;
	wake_up(&event_wait_queue);
\end{VerbatimU}

\noindent%
or:

\begin{VerbatimU}
	event_indicated = 1;
	wake_up_process(event_daemon);
\end{VerbatimU}

A general memory barrier is executed by \co{wake_up()} if it wakes something up.
If it doesn't wake anything up then a memory barrier may or may not be
executed; you must not rely on it.
The barrier occurs before the task state is accessed, in particular, it sits
between the STORE to indicate the event and the STORE to set \co{TASK_RUNNING}:

\begin{VerbatimU}
	CPU 1 (Sleeper)             CPU 2 (Waker)
	=========================== ===============================
	set_current_state();        STORE event_indicated
	  smp_store_mb();           wake_up();
	    STORE current->state      ...
	    <general barrier>         <general barrier>
	LOAD event_indicated          if ((LOAD task->state) & TASK_NORMAL)
	                                STORE task->state
\end{VerbatimU}

\noindent%
where \qco{task} is the thread being woken up and it equals CPU~1's \qco{current}.

To repeat, a general memory barrier is guaranteed to be executed by
\co{wake_up()} if something is actually awakened, but otherwise there is
no such guarantee.
To see this, consider the following sequence of events, where~\co{X} and~\co{Y} are both
initially zero:

\begin{VerbatimU}
	CPU 1           CPU 2
	=============== ==============
	X = 1;          Y = 1;
	smp_mb();       wake_up();
	LOAD Y          LOAD X
\end{VerbatimU}

If a wakeup does occur, one (at least) of the two loads must see~1.
If, on the other hand, a wakeup does not occur, both loads might see~0.

\co{wake_up_process()} always executes a general memory barrier.
The barrier again occurs before the task state is accessed.
In particular, if the \co{wake_up()} in the previous snippet were replaced
by a call to \co{wake_up_process()} then one of the two loads would be
guaranteed to see~1.

The available waker functions include:

\begin{VerbatimU}
	complete();
	wake_up();
	wake_up_all();
	wake_up_bit();
	wake_up_interruptible();
	wake_up_interruptible_all();
	wake_up_interruptible_nr();
	wake_up_interruptible_poll();
	wake_up_interruptible_sync();
	wake_up_interruptible_sync_poll();
	wake_up_locked();
	wake_up_locked_poll();
	wake_up_nr();
	wake_up_poll();
	wake_up_process();
\end{VerbatimU}

In terms of memory ordering, these functions all provide the same guarantees of
a \co{wake_up()} (or stronger).

\begin{Note}
Note that the memory barriers implied by the sleeper and the waker do
\emph{not} order multiple stores before the wake-up with respect to loads
of those stored values after the sleeper has called \co{set_current_state()}.
For instance, if the sleeper does:

\begin{VerbatimU}
	set_current_state(TASK_INTERRUPTIBLE);
	if (event_indicated)
		break;
	__set_current_state(TASK_RUNNING);
	do_something(my_data);
\end{VerbatimU}

\noindent%
and the waker does:

\begin{VerbatimU}
	my_data = value;
	event_indicated = 1;
	wake_up(&event_wait_queue);
\end{VerbatimU}

\noindent%
there's no guarantee that the change to \co{event_indicated} will be
perceived by the sleeper as coming after the change to \co{my_data}.
In such a circumstance, the code on both sides must interpolate its own
memory barriers between the separate data accesses.
Thus the above sleeper ought to do:

\begin{VerbatimU}
	set_current_state(TASK_INTERRUPTIBLE);
	if (event_indicated) {
		smp_rmb();
		do_something(my_data);
	}
\end{VerbatimU}

\noindent%
and the waker should do:

\begin{VerbatimU}
	my_data = value;
	smp_wmb();
	event_indicated = 1;
	wake_up(&event_wait_queue);
\end{VerbatimU}
\end{Note}

\subsection{Miscellaneous functions}

Other functions that imply barriers:

\begin{itemize}
 \item \co{schedule()} and similar imply full memory barriers.
\end{itemize}


\section{Inter-CPU acquiring barrier effects}
\label{sec:Inter-CPU acquiring barrier effects}

On SMP systems locking primitives give a more substantial form of barrier:
One that does affect memory access ordering on other CPUs, within the
context of conflict on any particular lock.


\subsection{Acquires vs memory accesses}

Consider the following:
The system has a pair of spinlocks~\co{M} and~\co{Q}, and three CPUs; then
should the following sequence of events occur:

\begin{VerbatimU}
	CPU 1                   CPU 2
	======================= =======================
	WRITE_ONCE(*A, a);      WRITE_ONCE(*E, e);
	ACQUIRE M               ACQUIRE Q
	WRITE_ONCE(*B, b);      WRITE_ONCE(*F, f);
	WRITE_ONCE(*C, c);      WRITE_ONCE(*G, g);
	RELEASE M               RELEASE Q
	WRITE_ONCE(*D, d);      WRITE_ONCE(*H, h);
\end{VerbatimU}

Then there is no guarantee as to what order CPU~3 will see the accesses
to~\co{*A} through~\co{*H} occur in, other than the constraints imposed by
the separate locks on the separate CPUs.
It might, for example, see:

\begin{VerbatimU}[tabsize=1]
	*E, ACQUIRE M, ACQUIRE Q, *G, *C, *F, *A, *B, RELEASE Q, *D, *H, RELEASE M
\end{VerbatimU}

But it won't see any of:

\begin{VerbatimU}
	*B, *C or *D preceding ACQUIRE M
	*A, *B or *C following RELEASE M
	*F, *G or *H preceding ACQUIRE Q
	*E, *F or *G following RELEASE Q
\end{VerbatimU}


\section{Where are memory barriers needed?}

Under normal operation, memory operation reordering is generally not going to
be a problem as a single-threaded linear piece of code will still appear to
work correctly, even if it's in an SMP kernel.
There are, however, four circumstances in which reordering definitely
\emph{could} be a problem:

\begin{itemize}[itemsep=0pt]
 \item Interprocessor interaction.
 \item Atomic operations.
 \item Accessing devices.
 \item Interrupts.
\end{itemize}

\subsection{Interprocessor interaction}

When there's a system with more than one processor, more than one CPU in the
system may be working on the same data set at the same time.
This can cause synchronisation problems, and the usual way of dealing with
them is to use locks.
Locks, however, are quite expensive, and so it may be preferable to operate
without the use of a lock if at all possible.
In such a case, operations that affect both CPUs may have to be carefully
ordered to prevent a malfunction.

Consider, for example, the R/W semaphore slow path.
Here a waiting process is queued on the semaphore, by virtue of it having
a piece of its stack linked to the semaphore's list of waiting processes:

\begin{VerbatimU}
	struct rw_semaphore {
		...
		spinlock_t lock;
		struct list_head waiters;
	};

	struct rwsem_waiter {
		struct list_head list;
		struct task_struct *task;
	};
\end{VerbatimU}

To wake up a particular waiter, the \co{up_read()} or \co{up_write()}
functions have to:

\begin{enumerate}
 \item
     read the next pointer from this waiter's record to know as to where the
     next waiter record is;

 \item
     read the pointer to the waiter's task structure;

 \item
     clear the task pointer to tell the waiter it has been given the semaphore;

 \item
     call \co{wake_up_process()} on the task; and

 \item
     release the reference held on the waiter's task struct.
\end{enumerate}

In other words, it has to perform this sequence of events:

\begin{VerbatimU}
	LOAD waiter->list.next;
	LOAD waiter->task;
	STORE waiter->task;
	CALL wakeup
	RELEASE task
\end{VerbatimU}

\noindent%
and if any of these steps occur out of order, then the whole thing may
malfunction.

Once it has queued itself and dropped the semaphore lock, the waiter does not
get the lock again; it instead just waits for its task pointer to be cleared
before proceeding.
Since the record is on the waiter's stack, this means that if the task
pointer is cleared \emph{before} the next pointer in the list is read,
another CPU might start processing the waiter and might clobber the waiter's
stack before the \co{up*()} function has a chance to read the next pointer.

Consider then what might happen to the above sequence of events:

\begin{VerbatimU}
	CPU 1                           CPU 2
	===============================	===============================
	                                down_xxx()
	                                Queue waiter
	                                Sleep
	up_yyy()
	LOAD waiter->task;
	STORE waiter->task;
	                                Woken up by other event
	<preempt>
	                                Resume processing
	                                down_xxx() returns
	                                call foo()
	                                foo() clobbers *waiter
	</preempt>
	LOAD waiter->list.next;
	--- OOPS ---
\end{VerbatimU}

This could be dealt with using the semaphore lock, but then the \co{down_xxx()}
function has to needlessly get the spinlock again after being woken up.

The way to deal with this is to insert a general SMP memory barrier:

\begin{VerbatimU}
	LOAD waiter->list.next;
	LOAD waiter->task;
	smp_mb();
	STORE waiter->task;
	CALL wakeup
	RELEASE task
\end{VerbatimU}

In this case, the barrier makes a guarantee that all memory accesses before the
barrier will appear to happen before all the memory accesses after the barrier
with respect to the other CPUs on the system.
It does \emph{not} guarantee that all the memory accesses before the barrier
will be complete by the time the barrier instruction itself is complete.

On a UP system---where this wouldn't be a problem---the \co{smp_mb()} is just a
compiler barrier, thus making sure the compiler emits the instructions in the
right order without actually intervening in the CPU.
Since there's only one CPU, that CPU's dependency ordering logic will take
care of everything else.

\subsection{Atomic operations}

While they are technically interprocessor interaction considerations, atomic
operations are noted specially as some of them imply full memory barriers and
some don't, but they're very heavily relied on as a group throughout the
kernel.

See \path{Documentation/atomic_t.txt} for more information.

\subsection{Accessing devices}

Many devices can be memory mapped, and so appear to the CPU as if they're just
a set of memory locations.
To control such a device, the driver usually has to make the right memory
accesses in exactly the right order.

However, having a clever CPU or a clever compiler creates a potential problem
in that the carefully sequenced accesses in the driver code won't reach the
device in the requisite order if the CPU or the compiler thinks it is more
efficient to reorder, combine or merge accesses---something that would cause
the device to malfunction.

Inside of the Linux kernel, I/O should be done through the appropriate accessor
routines---such as \co{inb()} or \co{writel()}---which know how to make such
accesses appropriately sequential.
While this, for the most part, renders the explicit use of memory barriers
unnecessary, if the accessor functions are used to refer to an I/O memory
window with relaxed memory access properties, then \emph{mandatory} memory
barriers are required to enforce ordering.

See \path{Documentation/driver-api/device-io.rst} for more information.


\subsection{Interrupts}

A driver may be interrupted by its own interrupt service routine, and thus the
two parts of the driver may interfere with each other's attempts to control or
access the device.

This may be alleviated---at least in part---by disabling local interrupts (a
form of locking), such that the critical operations are all contained within
the interrupt-disabled section in the driver.
While the driver's interrupt routine is executing, the driver's core may
not run on the same CPU, and its interrupt is not permitted to happen again
until the current interrupt has been handled, thus the interrupt handler
does not need to lock against that.

However, consider a driver that was talking to an ethernet card that sports an
address register and a data register.
If that driver's core talks to the card under interrupt-disablement and then
the driver's interrupt handler is invoked:

\begin{VerbatimU}
	LOCAL IRQ DISABLE
	writew(ADDR, 3);
	writew(DATA, y);
	LOCAL IRQ ENABLE
	<interrupt>
	writew(ADDR, 4);
	q = readw(DATA);
	</interrupt>
\end{VerbatimU}

The store to the data register might happen after the second store to the
address register if ordering rules are sufficiently relaxed:

\begin{VerbatimU}
	STORE *ADDR = 3, STORE *ADDR = 4, STORE *DATA = y, q = LOAD *DATA
\end{VerbatimU}

If ordering rules are relaxed, it must be assumed that accesses done inside an
interrupt disabled section may leak outside of it and may interleave with
accesses performed in an interrupt---and vice versa---unless implicit or
explicit barriers are used.

Normally this won't be a problem because the I/O accesses done inside such
sections will include synchronous load operations on strictly ordered I/O
registers that form implicit I/O barriers.


A similar situation may occur between an interrupt routine and two routines
running on separate CPUs that communicate with each other.
If such a case is likely, then interrupt-disabling locks should be used to
guarantee ordering.


\section{Kernel I/O barrier effects}
\label{sec:Kernel I/O barrier effects}

Interfacing with peripherals via I/O accesses is deeply architecture and device
specific.
Therefore, drivers which are inherently non-portable may rely on specific
behaviours of their target systems in order to achieve synchronization
in the most lightweight manner possible.
For drivers intending to be portable between multiple architectures and
bus implementations, the kernel offers a series of accessor functions that
provide various degrees of ordering guarantees:

\begin{description}[style=nextline]
 \item[\tco{readX()}, \tco{writeX()}:]
	The \co{readX()} and \co{writeX()} MMIO accessors take a pointer to
	the peripheral being accessed as an \co{__iomem *} parameter.
	For pointers mapped with the default I/O attributes (e.g., those
        returned by \co{ioremap()}), the ordering guarantees are as follows:

	\begin{enumerate}
	 \item
	   All \co{readX()} and \co{writeX()} accesses to the same peripheral
	   are ordered with respect to each other.
	   This ensures that MMIO register accesses by the same CPU thread to
	   a particular device will arrive in program order.

	 \item
	   A \co{writeX()} issued by a CPU thread holding a spinlock is ordered
	   before a \co{writeX()} to the same peripheral from another CPU thread
	   issued after a later acquisition of the same spinlock.
	   This ensures that MMIO register writes to a particular device
	   issued while holding a spinlock will arrive in an order consistent
	   with acquisitions of the lock.

	 \item
	   A \co{writeX()} by a CPU thread to the peripheral will first wait
	   for the completion of all prior writes to memory either issued by, or
	   propagated to, the same thread.
	   This ensures that writes by the CPU to an outbound DMA buffer
	   allocated by \co{dma_alloc_coherent()} will be visible to a DMA
	   engine when the CPU writes to its MMIO control register to trigger
	   the transfer.

	 \item
	   A \co{readX()} by a CPU thread from the peripheral will complete
	   before any subsequent reads from memory by the same thread can begin.
	   This ensures that reads by the CPU from an incoming DMA buffer
	   allocated by \co{dma_alloc_coherent()} will not see stale data
	   after reading from the DMA engine's MMIO status register to
	   establish that the DMA transfer has completed.

	 \item
	   A \co{readX()} by a CPU thread from the peripheral will complete
	   before any subsequent \co{delay()} loop can begin execution on the
	   same thread.
	   This ensures that two MMIO register writes by the CPU to a peripheral
	   will arrive at least 1us apart if the first write is immediately read
	   back with \co{readX()} and \co{udelay(1)} is called prior to the
	   second \co{writeX()}:

\begin{VerbatimU}[tabsize=1]
		writel(42, DEVICE_REGISTER_0); // Arrives at the device...
		readl(DEVICE_REGISTER_0);
		udelay(1);
		writel(42, DEVICE_REGISTER_1); // ...at least 1us before this.
\end{VerbatimU}
	\end{enumerate}

	The ordering properties of \co{__iomem} pointers obtained with
	non-default attributes (e.g., those returned by \co{ioremap_wc()})
	are specific to the underlying architecture and therefore the
	guarantees listed above cannot generally be relied upon for accesses
	to these types of mappings.

 \item[\tco{readX_relaxed()}, \tco{writeX_relaxed()}:]

	These are similar to \co{readX()} and \co{writeX()}, but provide
	weaker memory ordering guarantees.
	Specifically, they do not guarantee ordering with respect to locking,
	normal memory accesses or \co{delay()} loops (i.e.,
	bullets 2--5 above) but they are still guaranteed to be ordered with
	respect to other accesses from the same CPU thread to the same
	peripheral when operating on \co{__iomem} pointers mapped with the
        default I/O attributes.

 \item[\tco{readsX()}, \tco{writesX()}:]

	The \co{readsX()} and \co{writesX()} MMIO accessors are designed for
	accessing register-based, memory-mapped FIFOs residing on peripherals
	that are not capable of performing DMA\@.
	Consequently, they provide only the ordering guarantees of
	\co{readX_relaxed()} and \co{writeX_relaxed()}, as documented above.

 \item[\tco{inX()}, \tco{outX()}:]

	The \co{inX()} and \co{outX()} accessors are intended to access legacy
	port-mapped I/O peripherals, which may require special instructions
	on some architectures (notably x86).
	The port number of the peripheral being accessed is passed as an
	argument.

	Since many CPU architectures ultimately access these peripherals via an
	internal virtual memory mapping, the portable ordering guarantees
	provided by \co{inX()} and \co{outX()} are the same as those provided
	by \co{readX()} and \co{writeX()} respectively when accessing a
	mapping with the default I/O attributes.

	Device drivers may expect \co{outX()} to emit a non-posted write
	transaction that waits for a completion response from the I/O
	peripheral before returning.
	This is not guaranteed by all architectures and is therefore not
	part of the portable ordering semantics.

 \item[\tco{insX()}, \tco{outsX()}:]

	As above, the \co{insX()} and \co{outsX()} accessors provide the same
	ordering guarantees as \co{readsX()} and \co{writesX()} respectively
	when accessing a mapping with the default I/O attributes.

 \item[\tco{ioreadX()}, \tco{iowriteX()}:]

	These will perform appropriately for the type of access they're actually
	doing, be it \co{inX()}/\co{outX()} or \co{readX()}/\co{writeX()}.
\end{description}

With the exception of the string accessors (\co{insX()}, \co{outsX()},
\co{readsX()} and \co{writesX()}), all of the above assume that the underlying
peripheral is little-endian and will therefore perform byte-swapping
operations on big-endian architectures.


\section{Assumed minimum execution ordering model}

It has to be assumed that the conceptual CPU is weakly-ordered but that it will
maintain the appearance of program causality with respect to itself.
Some CPUs (such as i386 or \co{x86_64}) are more constrained than others
(such as powerpc or frv), and so the most relaxed case (namely DEC Alpha)
must be assumed outside of arch-specific code.

This means that it must be considered that the CPU will execute its instruction
stream in any order it feels like---or even in parallel---provided that if an
instruction in the stream depends on an earlier instruction, then that
earlier instruction must be sufficiently complete\footnote{
	Some instructions have more than one effect---such as changing the
	condition codes, changing registers or changing memory---and
	different instructions may depend on different effects.}
before the later instruction may proceed; in other words, provided that the
appearance of causality is maintained.

A CPU may also discard any instruction sequence that winds up having no
ultimate effect.
For example, if two adjacent instructions both load an immediate value
into the same register, the first may be discarded.


Similarly, it has to be assumed that compiler might reorder the instruction
stream in any way it sees fit, again provided the appearance of causality is
maintained.


\section{The effects of the CPU cache}

The way cached memory operations are perceived across the system is affected to
a certain extent by the caches that lie between CPUs and memory, and by the
memory coherence system that maintains the consistency of state in the system.

As far as the way a CPU interacts with another part of the system through the
caches goes, the memory system has to include the CPU's caches, and memory
barriers for the most part act at the interface between the CPU and its cache
(memory barriers logically act on the dotted line in the following diagram):

\begin{VerbatimU}[tabsize=1]
	    <--- CPU --->         :       <----------- Memory ----------->
	                          :
	+--------+    +--------+  :   +--------+    +-----------+
	|        |    |        |  :   |        |    |           |    +--------+
	|  CPU   |    | Memory |  :   | CPU    |    |           |    |        |
	|  Core  |--->| Access |----->| Cache  |<-->|           |    |        |
	|        |    | Queue  |  :   |        |    |           |--->| Memory |
	|        |    |        |  :   |        |    |           |    |        |
	+--------+    +--------+  :   +--------+    |           |    |        |
	                          :                 | Cache     |    +--------+
	                          :                 | Coherency |
	                          :                 | Mechanism |    +--------+
	+--------+    +--------+  :   +--------+    |           |    |        |
	|        |    |        |  :   |        |    |           |    |        |
	|  CPU   |    | Memory |  :   | CPU    |    |           |--->| Device |
	|  Core  |--->| Access |----->| Cache  |<-->|           |    |        |
	|        |    | Queue  |  :   |        |    |           |    |        |
	|        |    |        |  :   |        |    |           |    +--------+
	+--------+    +--------+  :   +--------+    +-----------+
	                          :
	                          :
\end{VerbatimU}

Although any particular load or store may not actually appear outside of the
CPU that issued it since it may have been satisfied within the CPU's own cache,
it will still appear as if the full memory access had taken place as far as the
other CPUs are concerned since the cache coherency mechanisms will migrate the
cacheline over to the accessing CPU and propagate the effects upon conflict.

The CPU core may execute instructions in any order it deems fit, provided the
expected program causality appears to be maintained.  Some of the instructions
generate load and store operations which then go into the queue of memory
accesses to be performed.  The core may place these in the queue in any order
it wishes, and continue execution until it is forced to wait for an instruction
to complete.

What memory barriers are concerned with is controlling the order in which
accesses cross from the CPU side of things to the memory side of things, and
the order in which the effects are perceived to happen by the other observers
in the system.

\begin{Note}
Memory barriers are \emph{not} needed within a given CPU, as CPUs always see
their own loads and stores as if they had happened in program order.
\end{Note}

\begin{Note}
MMIO or other device accesses may bypass the cache system.
This depends on the properties of the memory window through which devices
are accessed and/or the use of any special device communication instructions
the CPU may have.
\end{Note}


\subsection{Cache coherency vs DMA}

Not all systems maintain cache coherency with respect to devices doing DMA\@.
In such cases, a device attempting DMA may obtain stale data from RAM because
dirty cache lines may be resident in the caches of various CPUs, and may not
have been written back to RAM yet.
To deal with this, the appropriate part of the kernel must flush the
overlapping bits of cache on each CPU (and maybe invalidate them as well).

In addition, the data DMA'd to RAM by a device may be overwritten by dirty
cache lines being written back to RAM from a CPU's cache after the device has
installed its own data, or cache lines present in the CPU's cache may simply
obscure the fact that RAM has been updated, until at such time as the cacheline
is discarded from the CPU's cache and reloaded.
To deal with this, the appropriate part of the kernel must invalidate the
overlapping bits of the cache on each CPU.

See \path{Documentation/core-api/cachetlb.rst} for more information
on cache management.


\subsection{Cache coherency vs MMIO}

Memory mapped I/O usually takes place through memory locations that are part of
a window in the CPU's memory space that has different properties assigned than
the usual RAM directed window.

Amongst these properties is usually the fact that such accesses bypass the
caching entirely and go directly to the device buses.
This means MMIO accesses may, in effect, overtake accesses to cached memory
that were emitted earlier.
A memory barrier isn't sufficient in such a case, but rather the cache must be
flushed between the cached memory write and the MMIO access if the two are in
any way dependent.


\section{The things CPUs get up to}

A programmer might take it for granted that the CPU will perform memory
operations in exactly the order specified, so that if the CPU is, for example,
given the following piece of code to execute:

\begin{VerbatimU}
	a = READ_ONCE(*A);
	WRITE_ONCE(*B, b);
	c = READ_ONCE(*C);
	d = READ_ONCE(*D);
	WRITE_ONCE(*E, e);
\end{VerbatimU}

\noindent%
they would then expect that the CPU will complete the memory operation for each
instruction before moving on to the next one, leading to a definite sequence of
operations as seen by external observers in the system:

\begin{VerbatimU}
	LOAD *A, STORE *B, LOAD *C, LOAD *D, STORE *E.
\end{VerbatimU}

Reality is, of course, much messier.
With many CPUs and compilers, the above assumption doesn't hold because:

\begin{itemize}
 \item
     loads are more likely to need to be completed immediately to permit
     execution progress, whereas stores can often be deferred without a
     problem;

 \item
     loads may be done speculatively, and the result discarded should it prove
     to have been unnecessary;

 \item
     loads may be done speculatively, leading to the result having been fetched
     at the wrong time in the expected sequence of events;

 \item
     the order of the memory accesses may be rearranged to promote better use
     of the CPU buses and caches;

 \item
     loads and stores may be combined to improve performance when talking to
     memory or I/O hardware that can do batched accesses of adjacent locations,
     thus cutting down on transaction setup costs (memory and PCI devices may
     both be able to do this); and

 \item
     the CPU's data cache may affect the ordering, and while cache-coherency
     mechanisms may alleviate this---once the store has actually hit the
     cache---there's no guarantee that the coherency management will be
     propagated in order to other CPUs.
\end{itemize}

So what another CPU, say, might actually observe from the above piece of code
is:

\begin{VerbatimU}
	LOAD *A, ..., LOAD {*C,*D}, STORE *E, STORE *B

	(Where "LOAD {*C,*D}" is a combined load)
\end{VerbatimU}

However, it is guaranteed that a CPU will be self-consistent: it will see its
\emph{own} accesses appear to be correctly ordered, without the need for a
memory barrier.
For instance with the following code:

\begin{VerbatimU}
	U = READ_ONCE(*A);
	WRITE_ONCE(*A, V);
	WRITE_ONCE(*A, W);
	X = READ_ONCE(*A);
	WRITE_ONCE(*A, Y);
	Z = READ_ONCE(*A);
\end{VerbatimU}

\noindent%
and assuming no intervention by an external influence, it can be assumed that
the final result will appear to be:

\begin{VerbatimU}
	U == the original value of *A
	X == W
	Z == Y
	*A == Y
\end{VerbatimU}

The code above may cause the CPU to generate the full sequence of memory
accesses:

\begin{VerbatimU}
	U=LOAD *A, STORE *A=V, STORE *A=W, X=LOAD *A, STORE *A=Y, Z=LOAD *A
\end{VerbatimU}

\noindent%
in that order, but, without intervention, the sequence may have almost any
combination of elements combined or discarded, provided the program's view
of the world remains consistent.
Note that \co{READ_ONCE()} and \co{WRITE_ONCE()} are \emph{not} optional
in the above example, as there are architectures where a given CPU might
reorder successive loads to the same location.
On such architectures, \co{READ_ONCE()} and \co{WRITE_ONCE()} do whatever is
necessary to prevent this, for example, on Itanium the volatile casts
used by \co{READ_ONCE()} and \co{WRITE_ONCE()} cause GCC to emit the special
\co{ld.acq} and \co{st.rel} instructions (respectively) that prevent such
reordering.

The compiler may also combine, discard or defer elements of the sequence before
the CPU even sees them.

For instance:

\begin{VerbatimU}
	*A = V;
	*A = W;
\end{VerbatimU}

\noindent%
may be reduced to:

\begin{VerbatimU}
	*A = W;
\end{VerbatimU}

\noindent%
since, without either a write barrier or an \co{WRITE_ONCE()}, it can be
assumed that the effect of the storage of~\co{V} to~\co{*A} is lost.
Similarly:

\begin{VerbatimU}
	*A = Y;
	Z = *A;
\end{VerbatimU}

\noindent%
may, without a memory barrier or an \co{READ_ONCE()} and \co{WRITE_ONCE()}, be
reduced to:

\begin{VerbatimU}
	*A = Y;
	Z = Y;
\end{VerbatimU}

\noindent%
and the LOAD operation never appear outside of the CPU.


\subsection{And then there's the Alpha}

The DEC Alpha CPU is one of the most relaxed CPUs there is.
Not only that, some versions of the Alpha CPU have a split data cache,
permitting them to have two semantically-related cache lines updated at
separate times.
This is where the address-dependency barrier really becomes necessary as
this synchronises both caches with the memory coherence system, thus making
it seem like pointer changes vs new data occur in the right order.

The Alpha defines the Linux kernel's memory model, although as of v4.15
the Linux kernel's addition of \co{smp_mb()} to \co{READ_ONCE()} on Alpha
greatly reduced its impact on the memory model.


\subsection{Virtual machine guests}
\label{sec:Virtual machine guests}

Guests running within virtual machines might be affected by SMP effects even if
the guest itself is compiled without SMP support.
This is an artifact of interfacing with an SMP host while running an UP kernel.
Using mandatory barriers for this use-case would be possible but is often
suboptimal.

To handle this case optimally, low-level \co{virt_mb()} etc macros are
available.
These have the same effect as \co{smp_mb()} etc when SMP is enabled, but
generate identical code for SMP and non-SMP systems.
For example, virtual machine guests should use \co{virt_mb()} rather than
\co{smp_mb()} when synchronizing against a (possibly SMP) host.

These are equivalent to \co{smp_mb()} etc counterparts in all other respects,
in particular, they do not control MMIO effects:
To control MMIO effects, use mandatory barriers.


\section{Example uses}

\subsection{Circular buffers}

Memory barriers can be used to implement circular buffering without the need
of a lock to serialise the producer with the consumer.
See \path{Documentation/core-api/circular-buffers.rst} for details.


\section{References}

\begingroup
\footnotesize
\begin{itemize}
\item
Alpha AXP Architecture Reference Manual, Second Edition (Sites \& Witek,
Digital Press)
  \begin{itemize}
    \item
	Chapter 5.2: Physical Address Space Characteristics
    \item
	Chapter 5.4: Caches and Write Buffers
    \item
	Chapter 5.5: Data Sharing
    \item
	Chapter 5.6: Read/Write Ordering
  \end{itemize}

\item
AMD64 Architecture Programmer's Manual Volume 2: System Programming
  \begin{itemize}
    \item
	Chapter 7.1: Memory-Access Ordering
    \item
	Chapter 7.4: Buffering and Combining Memory Writes
  \end{itemize}
       
\item
ARM Architecture Reference Manual (ARMv8, for ARMv8-A architecture profile)
  \begin{itemize}
    \item
	Chapter B2: The AArch64 Application Level Memory Model
  \end{itemize}

\item
IA-32 Intel Architecture Software Developer's Manual, Volume 3:
System Programming Guide
  \begin{itemize}
    \item
	Chapter 7.1: Locked Atomic Operations
    \item
	Chapter 7.2: Memory Ordering
    \item
	Chapter 7.4: Serializing Instructions
  \end{itemize}

\item
The SPARC Architecture Manual, Version 9
  \begin{itemize}
    \item
	Chapter 8: Memory Models
    \item
	Appendix D: Formal Specification of the Memory Models
    \item
	Appendix J: Programming with the Memory Models
  \end{itemize}

\item
Storage in the PowerPC (Stone and Fitzgerald)

\item
UltraSPARC Programmer Reference Manual
  \begin{itemize}
    \item
	Chapter 5: Memory Accesses and Cacheability
    \item
	Chapter 15: Sparc-V9 Memory Models
  \end{itemize}

\item
UltraSPARC III Cu User's Manual
  \begin{itemize}
    \item
	Chapter 9: Memory Models
  \end{itemize}

\item
UltraSPARC IIIi Processor User's Manual
  \begin{itemize}
    \item
	Chapter 8: Memory Models
  \end{itemize}

\item
UltraSPARC Architecture 2005
  \begin{itemize}
    \item
	Chapter 9: Memory
    \item
	Appendix D: Formal Specifications of the Memory Models
  \end{itemize}

\item
UltraSPARC T1 Supplement to the UltraSPARC Architecture 2005
  \begin{itemize}
    \item
	Chapter 8: Memory Models
    \item
	Appendix F: Caches and Cache Coherency
  \end{itemize}

\item
Solaris Internals, Core Kernel Architecture, p63--68:
  \begin{itemize}
    \item
	Chapter 3.3: Hardware Considerations for Locks and
			Synchronization
  \end{itemize}

\item
Unix Systems for Modern Architectures, Symmetric Multiprocessing and Caching
for Kernel Programmers:
  \begin{itemize}
    \item
	Chapter 13: Other Memory Models
  \end{itemize}

\item
Intel Itanium Architecture Software Developer's Manual: Volume 1:
  \begin{itemize}
    \item
	Section 2.6: Speculation
    \item
	Section 4.4: Memory Access
  \end{itemize}
\end{itemize}
\endgroup
